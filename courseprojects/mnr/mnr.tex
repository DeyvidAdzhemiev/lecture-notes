\documentclass[12pt,a4paper]{article}
\usepackage[utf8]{inputenc}
\usepackage{amsmath}
\usepackage{amsfonts}
\usepackage{amssymb}
\usepackage{listings}
\usepackage{url}
\usepackage[bulgarian]{babel}
\usepackage{listings}
\usepackage{enumerate}
\usepackage[framemethod=tikz]{mdframed}
\usepackage{enumitem}
\usepackage{relsize}
\usepackage{hyperref}
\usepackage{caption}
\usepackage{tikz}
\usetikzlibrary{shapes,arrows,positioning,calc,positioning,fit,chains}
\usepackage{forest}


\captionsetup{font=footnotesize}

\lstset{breaklines=true}

\setenumerate[1]{label=\thesection.\arabic*.}
\setenumerate[2]{label*=\arabic*.}

\newcommand{\code}[1]{\texttt{#1}}

\hypersetup{
    colorlinks,
    citecolor=black,
    filecolor=black,
    linkcolor=black,
    urlcolor=black
}

\tikzset{
block/.style = {draw, fill=white, rectangle,align = center},
entry/.style = {draw, fill=black, circle, radius=3em},
condition/.style = {draw, fill=white, diamond, align = center,node distance=3cm},
fork/.style = {draw, fill=black, circle,inner sep=1pt},
}
\tikzset{
block/.style = {draw, fill=white, rectangle,align = center},
entry/.style = {draw, fill=black, circle, radius=3em},
condition/.style = {draw, fill=white, diamond, align = center,node distance=3cm},
fork/.style = {draw, fill=black, circle,inner sep=1pt},
lnode/.style={rectangle split, rectangle split parts=3,draw, rectangle split horizontal},
treenode/.style = {align=center, inner sep=0pt, text centered, circle, font=\sffamily\bfseries, draw=black, fill=white, text width=1.5em}
}



\author{\textit{Калин Георгиев}\\
\small{kalin@fmi.uni-sofia.bg}}
\title{\textsc{Курсов проект за Машини с неограничени регистри}}


\begin{document}
\maketitle

\section {Дефинция на МНР и примери}

\small{Дефиницията на Машина с неогрничени регистри по-долу е взаимствана от учебника \cite{tprog}\textit{А. Дичев, И. Сосков, ``Теория на програмите'', Издателство на СУ, София, 1998}.

\vspace{20px}

\begin{mdframed}[hidealllines=true,backgroundcolor=gray!20]

	``Машина с неограничени регистри'' (или МНР) наричаме абстрактна машина, разполагаща с неограничена памет. Паметта на машината се представя с безкрайна редица от естествени числа $m[0],m[1],...$, където $m[i] \in \mathcal{N}$. Елементите $m[i]$ на редицата наричаме ``клетки'' на паметта на машината, а числото $i$ наричаме ``адрес'' на клетката $m[i]$.

	 МНР разполага с набор от инструцкии за работа с паметта. Всяка инструкция получава един или повече параметри (операнди) и може да предизвика промяна в стойността на някоя от клетките на паметта. Инструкциите на МНР за работа с паметта са:

	\begin{enumerate}[label=\arabic*)]
		\item \code{ZERO n}: Записва стойността 0 в клетката с адрес $n$
		\item \code{INC n}: Увеличава с единица стойността, записана в клетката с адрес $n$
		\item \code{MOVE x y}: Присвоява на клетката с адрес $y$ стойността на клетката с адрес $x$
	\end{enumerate}

	``Програма'' за МНР наричаме всяка последователност от инструкции на МНР и съответните им операнди. Всяка инструкция от програмата индексираме с поредния ѝ номер. Изпълнението на програмата започва от първата инструкция и преминава през всички инструкции последователно, освен в някои случаи, описани по-долу. Изпълнението на програмата се прекратвя след изпълнението на последната ѝ инструкция. Например, след изпълнението на следната програма:

	\begin{verbatim}
	0: ZERO 0
	1: ZERO 1
	2: ZERO 2
	3: INC 1
	4: INC 2
	5: INC 2
	\end{verbatim}

	Първите три клетки на машината ще имат стойност 0, 1, 2, независимо от началните им стойности.

	Освен инструкциите за работа с паметта, МНР притежават и една инструкция за промяна на последователноста на изпълнение на програмата:

	\begin{enumerate}[label=\arabic*)]
	\setcounter{enumi}{3}
		\item \code{JUMP x}: Изпълнението на програмата ``прескача'' и продължава от инструкцията с пореден номер $x$. Ако програмата има по-малко от $x+1$ инструкции, изпълнението ѝ се прекратява
		\item \code{JUMP x y z}: Ако съдържанията на клетките  $x$ и $y$ съвпадат, изпълнението на програмата ``прескача'' и продължава от инструкцията с пореден номер $z$. В противен случай, програмата продължава със следващата инструкция. Ако програмата има по-малко от $z+1$ инструкции, изпълнението ѝ се прекратява
	\end{enumerate}

	Например, нека изпълнето на следната програма започва при стойности на клиетките на паметта 10,0,0,...:

	\begin{verbatim}
	0: JUMP 0 1 5
	1: INC 1
	2: INC 2
	3: INC 2
	4: JUMP 0
	\end{verbatim}

	След приключване на програмата, първите три клетки на машината ще имат стойности 10, 10, 20.

\end{mdframed}

\begin{mdframed}[hidealllines=true,backgroundcolor=gray!20]
\textbf{Примери:} На Фигура \ref{fig:mnr}~(a) е показана блок схема на програма, изпозлваща само операторите \code{=}, \code{==}, \code{++} и \code{if}, която намира в променливата \code{result} сумата на променливите $a_0$ и $a_1$. $a_0$ и $a_1$ считаме за дадени. Променливата \code{count} се иницилиаира с 0, а \code{result} - с $a_0$. В цикъл се добавя по една единица към \code {count} и \code{result} дотогава, докато \code{count} достигне стойността на $a_1$. По този начин, към \code{result} се добавят $a_1$ на брой единици, т.е. стойността ѝ се увеличава с $a_1$ спрямо налчалната ѝ стойност $a_0$.

На Фигура \ref{fig:mnr}~(b) е показана същата програма, като операторите от първата са заменени със сътответните им инструцкии на МНР. Резултатът от програмата се получава в клетката $m[2]$, а за брояч се ползва клетката $m[3]$. На блок схемата са дадени поредните номера на инструкциите в окончателната програмата на МНР:
\begin{verbatim}
0:MOVE 0 2
1:ZERO 3
2:JUMP 1 3 6
3:INC 2
4:INC 3
5:JUMP 3
\end{verbatim}
\end{mdframed}

\begin{figure}
\relscale{0.7}
  \begin{tabular}{p{7cm} p{7cm}}
      \begin{tikzpicture}[auto, node distance=1.5cm,>=latex']
      \node [entry, name=start](start){};
      \node [block,name=init, below of = start] (init)
         {\code{result:=$a_0$}\\\code{counter:=0}};
      \node [fork,name=test1fork,below of = init,node distance = 1cm]{};
      \node [condition,name=test1, below of = test1fork,node distance = 2cm] (test1) {\code{counter==$a_1$}};
      \node [block,name=inc,right of = test1, node distance = 3cm] (inc) {\code{$a_0$++}\\\code{counter++}};
      \node [entry, name=end, below of = test1, node distance = 2.5cm](end){};
      \draw [->] (start) -- (init);
      \draw [-] (init) -- (test1fork);
      \draw [->] (test1fork) -- (test1);
      \draw [->] (test1) -- node{no} (inc);
      \draw [->] (inc) |- (test1fork);
      \draw [->] (test1) -- node []{yes}(end);
      \end{tikzpicture}

      &

      \begin{tikzpicture}[auto, node distance=1.5cm,>=latex']
      \node [entry, name=start](start){};
      \node [block,name=init, below of = start, align = left] (init)
         {\code{0:MOVE 0 2}\\\code{1:ZERO 3}};
      \node [fork,name=test1fork,below of = init,node distance = 1cm]{};
      \node [condition,name=test1, below of = test1fork,node distance = 2cm] (test1) {\code{2:JUMP 1 3 6}};
      \node [block,name=inc,right of = test1, node distance = 3cm,align = left] (inc) {\code{3:INC 2}\\\code{4:INC 3}\\\code{5:JUMP 3}};
      \node [entry, name=end, below of = test1, node distance = 2.5cm](end){};
      \draw [->] (start) -- (init);
      \draw [-] (init) -- (test1fork);
      \draw [->] (test1fork) -- (test1);
      \draw [->] (test1) -- node{} (inc);
      \draw [->] (inc) |- (test1fork);
      \draw [->] (test1) -- node []{}(end);
      \end{tikzpicture}

      \\
      (a)Програма за сумиране на числата $a_0$ и $a_1$ с изпозлване само на операторите \code{=}, \code{==}, \code{++} и \code{if}.
      &
      (b)Програма за сумиране на клетките $m[0]$ и $m[1]$ с инструкции на МНР.
  \end{tabular}

  \caption{Блок схеми на програма за сумиране на числа}
  \label{fig:mnr}
\end{figure}

\subsection{Примерни задачи за програми за МНР}


\begin{enumerate}[resume]
	\item Нека паметта на МНР е инициалирана с редицата $m,n,0,0,...$. Да се напише програма на МНР, след изпълнението на която клетката с адрес 2 съдържа числото $m+n$.
	\item Нека паметта на МНР е инициалирана с редицата $m,n,0,0,...$. Да се напише програма на МНР, след изпълнението на която клетката с адрес 2 съдържа числото $m \times n$.
	\item Нека паметта на МНР е инициалирана с редицата $m,n,0,0,...$. Да се напише програма на МНР, след изпълнението на която клетката с адрес 2 съдържа числото 1 тогава и само тогава, когато $m>n$ и числото 0 във всички останали случаи.
\end{enumerate}

\section{Условие на проекта}

Да се реализира интерпретатор за програми на МНР. Интепретаторът да работи в диалогов режим като приема инструкции за МНР и команди от стандартния вход. Във всеки момент интерпретарорът поддържа в паметта ``заредена програма'', състояща се от всички въведени или заредени от файл инструкции и команди на интерпретатора. Инструцкиите са номерирани с последователни ествествени числа спрямо реда на въвеждането им. Командите нямат номера, но също са част от програмата и са подредени заедно с инструкциите по реда на въвеждането им. 

Интерпретаторът да поддържа следните команди (каманди, които не са инструкции на МНР, започват със символа "/"):

\begin{enumerate}
    \item \code{/load <file name>}: Зарежда програма за МНР от текстов файл. Програмата може да съдържа инструкции за МНР и команди на интепретатора. Програмата не се изпълнява при зареждането на файла.
    \item \code{/zero x y}: Нулира клетките на паметта с адреси от \code{x} до \code{y}. 
    \item \code{/run}: Изпълнява заредената програмата, както и всички команди на интерпретатора от заредената програма. Инструкциите и командите се изпълняват в реда, в който са зададени в изходния фйал.
    \item \code{/call x}: Изпълнява заредената програмата, както и всички команди на интерпретатора от заредената програма, като започва изпълнението от инструкцията с пореден номер \code{x}, а не от началото. Инструкциите и командите се изпълняват в реда, в който са зададени в изходния фйал.
    \item \code{/mem x y}: Извежда на стандартния изход съдържанието на клетките с адреси от \code{x} до \code{y}.
    \item \code{/set x y}: Променя на \code{y} съдържанието на клетката с адрес \code{x}.
    \item \code{/add <file name>}: Зарежда програма за МНР от текстов файл. 
    
    \emph{Обхват} на програма наричаме наричаме целочисления интервал $[a,b]$, където $a$ е най-малкия адрес на клетка от паметта, който се използва в някоя инструкция на интерпретатора (но не и в командите!), а $b$ e най-големия такъв адрес. 
    
    Ако преди изпълнението на командата \code{/add} в паметта има заредена програма с обхват $[a,b]$, а новата програма има обхват $[a',b']$, новата програма да се преработи така, че да има обхват $[a'+b,b'+b]$. Инструкциите и командите на новата програмата да се добавят към края на заредената в паметта програма.
    \item \code{/code}: Извежда на стандартния изход заредена в паметта програма (заедно с командите на изтерпретатора в нея в съответния ред). 
    \end{enumerate}

\emph{Внимание:} За реализация на паметта на МНР да се използва ``разреден масив'' (\emph{sparce array}) по алгоритъм, избран от вас. 

Да се демонстрират командите \code{/add} и \code{/call} така, че една програма да ползва друга като подпрограма (например програма за умножение на числа да използва програма за събиране на числа като подпрограма).

\begin{thebibliography}{99}
    \bibitem{tprog} А. Дичев, И. Сосков, ``Теория на програмите'', Издателство на СУ, София, 1998
\end{thebibliography}


\end{document}