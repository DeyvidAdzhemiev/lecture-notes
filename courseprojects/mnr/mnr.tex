\documentclass[12pt,a4paper]{article}
\usepackage[utf8]{inputenc}
\usepackage{amsmath}
\usepackage{amsfonts}
\usepackage{amssymb}
\usepackage{listings}
\usepackage{url}
\usepackage[bulgarian]{babel}
\usepackage{listings}
\usepackage{enumerate}
\usepackage[framemethod=tikz]{mdframed}
\usepackage{enumitem}
\usepackage{relsize}
\usepackage{hyperref}
\usepackage{caption}
\usepackage{tikz}
\usetikzlibrary{shapes,arrows,positioning,calc,positioning,fit,chains}
\usepackage{forest}


\captionsetup{font=footnotesize}

\lstset{breaklines=true}

\setenumerate[1]{label=\thesection.\arabic*.}
\setenumerate[2]{label*=\arabic*.}

\newcommand{\code}[1]{\texttt{#1}}

\hypersetup{
    colorlinks,
    citecolor=black,
    filecolor=black,
    linkcolor=black,
    urlcolor=black
}

\tikzset{
block/.style = {draw, fill=white, rectangle,align = center},
entry/.style = {draw, fill=black, circle, radius=3em},
condition/.style = {draw, fill=white, diamond, align = center,node distance=3cm},
fork/.style = {draw, fill=black, circle,inner sep=1pt},
}
\tikzset{
block/.style = {draw, fill=white, rectangle,align = center},
entry/.style = {draw, fill=black, circle, radius=3em},
condition/.style = {draw, fill=white, diamond, align = center,node distance=3cm},
fork/.style = {draw, fill=black, circle,inner sep=1pt},
lnode/.style={rectangle split, rectangle split parts=3,draw, rectangle split horizontal},
treenode/.style = {align=center, inner sep=0pt, text centered, circle, font=\sffamily\bfseries, draw=black, fill=white, text width=1.5em}
}



\author{\textit{Калин Георгиев}\\
\small{kalin@fmi.uni-sofia.bg}}
\title{\textsc{Курсов проект за Машини с неограничени регистри}}


\begin{document}
\maketitle

\section {Дефиниция на МНР и примери}

\small{Дефиницията на Машина с неограничени регистри по-долу е взаимствана от учебника \cite{tprog} \textit{А. Дичев, И. Сосков, ``Теория на програмите'', Издателство на СУ, София, 1998}.

\vspace{20px}

\begin{mdframed}[hidealllines=true,backgroundcolor=gray!20]

	``Машина с неограничени регистри'' (или МНР) наричаме абстрактна машина, разполагаща с неограничена памет. Паметта на машината се представя с безкрайна редица от естествени числа $m[0], m[1], \ldots$, където $m[i] \in \mathcal{N}$. Елементите $m[i]$ на редицата наричаме ``клетки'' на паметта на машината, а числото $i$ наричаме ``адрес'' на клетката $m[i]$.

	 МНР разполага с набор от инструкции за работа с паметта. Всяка инструкция получава един или повече параметри (операнди) и може да предизвика промяна в стойността на някоя от клетките на паметта. Инструкциите на МНР за работа с паметта са:

	\begin{enumerate}[label=\arabic*)]
		\item \code{ZERO n}: Записва стойността 0 в клетката с адрес $n$
		\item \code{INC n}: Увеличава с единица стойността, записана в клетката с адрес $n$
		\item \code{MOVE x y}: Присвоява на клетката с адрес $y$ стойността на клетката с адрес $x$
	\end{enumerate}

	``Програма'' за МНР наричаме всяка последователност от инструкции на МНР и съответните им операнди. Всяка инструкция от програмата индексираме с поредния ѝ номер. Изпълнението на програмата започва от първата инструкция и преминава през всички инструкции последователно, освен в някои случаи, описани по-долу. Изпълнението на програмата се прекратява след изпълнението на последната ѝ инструкция. Например, след изпълнението на следната програма:

	\begin{verbatim}
	0: ZERO 0
	1: ZERO 1
	2: ZERO 2
	3: INC 1
	4: INC 2
	5: INC 2
	\end{verbatim}

	Първите три клетки на машината ще имат стойност 0, 1, 2, независимо от началните им стойности.

	Освен инструкциите за работа с паметта, МНР притежават и една инструкция за промяна на последователността на изпълнение на програмата:

	\begin{enumerate}[label=\arabic*)]
	\setcounter{enumi}{3}
		\item \code{JUMP z}: Изпълнението на програмата ``прескача'' и продължава от инструкцията с пореден номер $z$. Ако програмата има по-малко от $z+1$ инструкции, изпълнението ѝ се прекратява
		\item \code{JUMP x y z}: Ако съдържанията на клетките  $x$ и $y$ съвпадат, изпълнението на програмата ``прескача'' и продължава от инструкцията с пореден номер $z$. В противен случай, програмата продължава със следващата инструкция. Ако програмата има по-малко от $z+1$ инструкции, изпълнението ѝ се прекратява. Забележете, че горният вариант на инструкцията \code{JUMP} с един операнд (за безусловен преход) е частен случай на инструкцията \code{JUMP} с три операнда (за условен преход). Така \code{JUMP z} и може да се ``симулира'', например, с \code{JUMP 0 0 z}.
	\end{enumerate}

	Например, нека изпълнението на следната програма започва при стойности на клетките на паметта $10, 0, 0, \ldots$:

	\begin{verbatim}
	0: JUMP 0 1 5
	1: INC 1
	2: INC 2
	3: INC 2
	4: JUMP 0
	\end{verbatim}

	След приключване на програмата, първите три клетки на машината ще имат стойности 10, 10, 20.

\end{mdframed}

\begin{mdframed}[hidealllines=true,backgroundcolor=gray!20]
\textbf{Примери:} На Фигура \ref{fig:mnr}~(a) е показана блок схема на програма, използваща само операторите \code{=}, \code{==}, \code{++} и \code{if}, която намира в променливата \code{result} сумата на променливите $a_0$ и $a_1$. $a_0$ и $a_1$ считаме за дадени. Променливата \code{count} се инициализира с 0, а \code{result} - с $a_0$. В цикъл се добавя по една единица към \code {count} и \code{result} дотогава, докато \code{count} достигне стойността на $a_1$. По този начин, към \code{result} се добавят $a_1$ на брой единици, т.е. стойността ѝ се увеличава с $a_1$ спрямо налчалната ѝ стойност $a_0$.

На Фигура \ref{fig:mnr}~(b) е показана същата програма, като операторите от първата са заменени със съответните им инструкции на МНР. Резултатът от програмата се получава в клетката $m[2]$, а за брояч се ползва клетката $m[3]$. На блок схемата са дадени поредните номера на инструкциите в окончателната програмата на МНР:
\begin{verbatim}
0:MOVE 0 2
1:ZERO 3
2:JUMP 1 3 6
3:INC 2
4:INC 3
5:JUMP 2
\end{verbatim}
\end{mdframed}

\begin{figure}
\relscale{0.7}
  \begin{tabular}{p{7cm} p{7cm}}
      \begin{tikzpicture}[auto, node distance=1.5cm,>=latex']
      \node [entry, name=start](start){};
      \node [block,name=init, below of = start] (init)
         {\code{result:=$a_0$}\\\code{counter:=0}};
      \node [fork,name=test1fork,below of = init,node distance = 1cm]{};
      \node [condition,name=test1, below of = test1fork,node distance = 2cm] (test1) {\code{counter==$a_1$}};
      \node [block,name=inc,right of = test1, node distance = 3cm] (inc) {\code{$a_0$++}\\\code{counter++}};
      \node [entry, name=end, below of = test1, node distance = 2.5cm](end){};
      \draw [->] (start) -- (init);
      \draw [-] (init) -- (test1fork);
      \draw [->] (test1fork) -- (test1);
      \draw [->] (test1) -- node{no} (inc);
      \draw [->] (inc) |- (test1fork);
      \draw [->] (test1) -- node []{yes}(end);
      \end{tikzpicture}

      &

      \begin{tikzpicture}[auto, node distance=1.5cm,>=latex']
      \node [entry, name=start](start){};
      \node [block,name=init, below of = start, align = left] (init)
         {\code{0:MOVE 0 2}\\\code{1:ZERO 3}};
      \node [fork,name=test1fork,below of = init,node distance = 1cm]{};
      \node [condition,name=test1, below of = test1fork,node distance = 2cm] (test1) {\code{2:JUMP 1 3 6}};
      \node [block,name=inc,right of = test1, node distance = 3cm,align = left] (inc) {\code{3:INC 2}\\\code{4:INC 3}\\\code{5:JUMP 2}};
      \node [entry, name=end, below of = test1, node distance = 2.5cm](end){};
      \draw [->] (start) -- (init);
      \draw [-] (init) -- (test1fork);
      \draw [->] (test1fork) -- (test1);
      \draw [->] (test1) -- node{} (inc);
      \draw [->] (inc) |- (test1fork);
      \draw [->] (test1) -- node []{}(end);
      \end{tikzpicture}

      \\
      (a) Програма за сумиране на числата $a_0$ и $a_1$ с използване само на операторите \code{=}, \code{==}, \code{++} и \code{if}.
      &
      (b) Програма за сумиране на клетките $m[0]$ и $m[1]$ с инструкции на МНР.
  \end{tabular}

  \caption{Блок схеми на програма за сумиране на числа}
  \label{fig:mnr}
\end{figure}

\subsection{Примерни задачи за програми за МНР}


\begin{enumerate}[resume]
	\item Нека паметта на МНР е инициализирана с редицата $m,n,0,0,...$. Да се напише програма на МНР, след изпълнението на която клетката с адрес 2 съдържа числото $m+n$.
	\item Нека паметта на МНР е инициализирана с редицата $m,n,0,0,...$. Да се напише програма на МНР, след изпълнението на която клетката с адрес 2 съдържа числото $m \times n$.
	\item Нека паметта на МНР е инициализирана с редицата $m,n,0,0,...$. Да се напише програма на МНР, след изпълнението на която клетката с адрес 2 съдържа числото 1 тогава и само тогава, когато $m>n$ и числото 0 във всички останали случаи.
\end{enumerate}

\section{Условие на проекта}

``Програма за МНР'' наричаме последователност от два вида оператори: \emph{инструкции} и \emph{команди}.

Инструкциите за МНР са от петте вида, описани по-горе: \code{ZERO x}, \code{INC n}, \code{MOVE x y}, \code{JUMP z}, \code{JUMP x y z}. Инструкциите са номерирани с последователни естествени числа спрямо реда им в програмата.

Командите на МНР започват със символа \code{/} и нямат поредни номера. Командите в програмата не влияят върху номерацията на инструкциите. Видовете команди, които могат да бъдат включвани в програмите за МНР, са описани по-долу. 

Да се реализира интерпретатор за програми на МНР. Интепретаторът да работи в диалогов режим като приема команди от стандартния вход и ги изпълнява веднага след въвеждането им. Във всеки момент интерпретаторът поддържа в паметта ``заредена програма'', състояща се от всички въведени инструкции и команди на интерпретатора.

Интерпретаторът да поддържа следните команди:

\begin{enumerate}
    \item \code{/zero x y}: Нулира клетките на паметта с адреси от \code{x} до \code{y} включително.
    \item \code{/set x y}: Променя на \code{y} съдържанието на клетката с адрес \code{x}.
    \item \code{/copy x y z}: Копира съдържанието на \code{z} последователни клетки, започващи от адрес \code{x} в съответните \code{z} последователни клетки, започващи от адрес \code{y} 
    \item \code{/mem x y}: Извежда на стандартния изход съдържанието на клетките с адреси от \code{x} до \code{y} включително.
    \item \code{/load <filename>}: Зарежда програма за МНР от текстов файл. Заредената до момента програма за МНР в паметта на интерпретатора се изтрива и се замества с новата програма, заредена от файла. Програмата не се изпълнява при зареждането на файла. Ако командата е част от текущо изпълняваща се програма, изпълнението на програмата се прекратява.
    \item \code{/run}: Изпълнява заредената програма, като спазва последователността на инструкциите и командите и започва изпълнението от първия оператор на програмата. След изпълнението на даден оператор (инструкция или команда), различен от инструкцията \code{JUMP}, се преминава към изпълнението на следващия в последователността. Изпълнението на програмата се прекратява при опит за преминаване извън заредената последователност от оператори. Ако командата е част от текущо изпълняваща се програма, се преминава към изпълнението на първия оператор в програмата, без да се правят промени по паметта на машината.
    \item \code{/call x}: Изпълнява заредената програма, като започва изпълнението от инструкцията с пореден номер \code{x}, а не от началото. Инструкциите и командите се изпълняват в реда, в който са зададени в изходния файл.
    \item \code{/add <filename>}: Разширява заредената програма за МНР с оператори, прочетени от текстов файл.

    \emph{Обхват} на програма наричаме наричаме целочисления интервал $[a,b]$, където $a$ е най-малкия адрес на клетка от паметта, който се използва в някоя инструкция на интерпретатора (но не и в командите!), а $b$ e най-големия такъв адрес.

    Ако преди изпълнението на командата \code{/add} в паметта има заредена програма с обхват $[a,b]$ и $N$ на брой инструкции, а новата програма има обхват $[a',b']$, новата програма $P$ да се преработи така, че:
    \begin{itemize}
    \item да има обхват $[a'+b+1,b'+b+1]$, т.е. всички адреси на клетки, срещащи се в инструкции на $P$, да се увеличат с числото $b+1$;
    \item всички номера на инструкции, които се споменават в инструкции \code{JUMP} в $P$, да бъдат увеличени с $N$.
    \end{itemize}

    Операторите на $P$ да се добавят последователно към края на заредената в паметта програма.
    \item \code{/quote <string>}: Добавя нов оператор (инструкция или команда) \code{<string>} на края на заредената програма. Командата връща греша, в случай, че \code{<string>} не е валиден оператор.
    \item \code{/code}: Извежда на стандартния изход заредена в паметта програма.
    \item \code{/comment <string>}: Няма ефект, служи за добавяне на коментар
    \end{enumerate}

    \textbf{Примерна програма:}

Файл \code{fib.urm}:
\begin{verbatim}
/comment Програма за пресмятане на n-тото число на Fibonacci
/comment Числото n е записано в клетка с адрес 0
/comment Резултатът се записва в клетка с адрес 1
/comment Помощни клетки:
/comment Адрес 2: предишното число на Fibonacci
/comment Адрес 3: брояч, който се инкрементира до n
/comment Обхват: [0, 3]
/comment Брой инструкции: 10
/comment Нулиране на помощните клетки
ZERO 1
ZERO 2
ZERO 3
/comment Зареждане на подпрограма за добавяне на числа
/add add.urm
/comment Добавяне на оператор за връщане в главната програма
/comment при приключване на работа на подпрограмата
/quote JUMP 7
/comment Ако n = 0, край
JUMP 0 3 16
/comment Инициализация на числото на Fibonacci с номер 1 на 1 в клетка с адрес 1
INC 1
/comment Инициализация на брояча с 1
INC 3
/comment Ако n (записано на адрес 0) съвпада с брояча, край
JUMP 0 3 16
/comment Събиране на последните две числа на Fibonacci
/comment Подготовка на входа на подпрограмата add.urm
/copy 1 2 4
/comment Извикване на подпрограмата
/call 10
/comment Връщане от подпрограмата на инструкция с номер 7
/comment Запазване на текущото число на Fibonacci на адрес 1 като предишно на адрес 2
MOVE 2 1
/comment Копиране на новото число на Fibonacci, сметнато от подпрограмата, на адрес 1
/copy 6 1 1
/comment Увеличаване на брояча
INC 2
/comment Връщане в началото на цикъла
JUMP 3
\end{verbatim}

Файл \code{add.urm}:
\begin{verbatim}
/comment Програма за събиране на две числа в клетки с адреси 0 и 1
/comment Резултатът се записва в клетка с адрес 2
/comment Помощна клетка с адрес 3: брояч, който се инкрементира до числото на адрес 1
/comment Обхват: [0, 3]
/comment Брой инструкции: 6
/comment Нулиране на брояча
ZERO 3
/comment Прехвърляне на стойността на клетката с адрес 0 в резултата на адрес 2
MOVE 2 0
/comment Проверка за край
JUMP 1 3 6
/comment Инкрементиране на брояча и резултата
INC 2
INC 3
/comment Връщане в началото на цикъла
JUMP 2 
\end{verbatim}

Примерно изпълнение от стандартния вход:
\begin{verbatim}
/load fib.urm
/set 0 9
/run
/mem 1 1
/comment Извежда се 34
/mem 0 6
/comment Извежда се 9 34 21 9 21 13 34 13
\end{verbatim}

\emph{Внимание:} За реализация на паметта на МНР да се използва ``разреден масив'' (\emph{sparse array}) по алгоритъм, избран от вас.

\begin{thebibliography}{99}
    \bibitem{tprog} А. Дичев, И. Сосков, ``Теория на програмите'', Издателство на СУ, София, 1998
\end{thebibliography}


\end{document}