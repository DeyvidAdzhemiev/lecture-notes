\documentclass[12pt,a4paper]{article}
\usepackage[utf8]{inputenc}
\usepackage{amsmath}
\usepackage{amsfonts}
\usepackage{amssymb}
\usepackage{listings}
\usepackage{url}
\usepackage[bulgarian]{babel}
\usepackage{listings}
\usepackage{enumerate}


\lstset{breaklines=true} 


\author{\textit{email: kalin@fmi.uni-sofia.bg}}
\title{\textsc{Задачи за задължителна самоподготовка} \\
по \\
Увод в програмирането\\
\textit{масиви и низове}}



\begin{document}
\maketitle


\begin{enumerate}

	\item Задача 3.1.

	 Да се напише програма, която въвежда редица от n цели числа $(1 \leq n \leq 50)$ и намира и извежда минималното от тях.

	 \item Задача 3.2. 

	 Да се напише програма, която въвежда редицата от n $(1 \leq n \leq 50)$ цели числа $a_0, a_1, ..., a_{n-1}$ и намира и извежда сумата на тези елементи на редицата, които се явяват удвоени нечетни числа.

	\item Задача 3.3. 

	Да се напише програма, която намира и извежда сумата от положителните и произведението на отрицателните елементи на редицата от реални числа $a_0, a_1, ..., a_{n-1}$ $(1 \leq n \leq 20)$.

	\item Задача 3.7. 

	Да се напише програма, която изяснява има ли в редицата от цели числа $a_0, a_1, ..., a_{n-1}$ $(1 \leq n \leq 100)$ поне два последователни елемента с равни стойности.

	\item Задача 3.8. 

	Да се напише програма, която проверява дали редицата от реални числа $a_0, a_1, ..., a_{n-1}$ $(1 \leq n \leq 100)$ е монотонно растяща.




	\item Задача 3.10. 

	Да се напише програма, която за дадена числова редица $a_0, a_1, ..., a_{n-1}$ $(1 \leq n \leq 100)$ намира дължината на най-дългата ú ненамаляваща подредица $a_i, a_{i+1}, ..., a_{i+k}$ $(a_i \leq a_{i+1} \leq ... \leq a_{i+k})$.


	\item Задача 3.11. 

	Дадена е редицата от символи $s_0, s_1, ..., s_{n-1}$ $(1 \leq n \leq 100)$. Да се напише програма, която извежда отначало всички символи, които са цифри, след това всички символи, които са малки латински букви и накрая всички останали символи от редицата, запазвайки реда им в редицата.

	\item Задача 3.13. 

	Да се напише програма, която определя дали редицата от символи $s_0, s_1, ..., s_{n-1}$ $(1 \leq n \leq 100)$ е симетрична, т.е. четена отляво надясно и отдясно наляво е една и съща.


	\item Задача 3.15. 

	Да се напише програма, която въвежда реланите вектори $a_0, a_1, ..., a_{n-1}$ и $b_0, b_1, ..., b_{n-1}$ $(1 \leq n \leq 100)$,  намира скаларното им произведение  и го извежда на екрана. 

	\item Задача 3.26. (хистограма на символите). 

	Символен низ е съставен единствено от малки латински букви. Да се напише програма, която намира и извежда на екрана броя на срещанията на всяка от буквите на низа.

	\item Задача 3.28. (търсене на функция). 

	Дадени са два символни низа с еднаква дължина $s_1$ и $s_2$, съставени от малки латински букви. Да се напише програма, която проверява дали съществува функция $f:char \rightarrow char$, изобразяваща $s_1$ в $s_2$, така че $f(s_1[i])$ = $f(s_2[i])$ и $i=1..$дължината на $s_1$ и $s_2$.















	








\end{enumerate}


	\vspace{20px}

	\small{Някои от задачите са от сборника \textit{Магдалина Тодорова, Петър Армянов, Дафина Петкова, Калин Николов, ``Сборник от задачи по програмиране на C++. Първа част. Увод в програмирането''}. За тези задачи е запазена номерацията в сборника.}


\end{document}

