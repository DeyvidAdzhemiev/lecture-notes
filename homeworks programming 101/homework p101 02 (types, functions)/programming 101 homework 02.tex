\documentclass[12pt,a4paper]{article}
\usepackage[utf8]{inputenc}
\usepackage{amsmath}
\usepackage{amsfonts}
\usepackage{amssymb}
\usepackage{listings}
\usepackage{url}
\usepackage[bulgarian]{babel}
\usepackage{listings}
\usepackage{enumerate}


\lstset{breaklines=true} 


\author{\textit{email: kalin@fmi.uni-sofia.bg}}
\title{\textsc{Задачи за задължителна самоподготовка} \\
по \\
Увод в програмирането\\
\textit{типове и функции}}



\begin{document}
\maketitle


\begin{enumerate}
	\item Задача 4.10.

	Да се напише програма, която намира лице на кръг с даден радиус. За целта да се дефинира и използва подходяща функция.

	\item Задача 4.11. Да се напише функция, която намира площта на фигурата, заключена между $f(x) = ax^2 + bx + c$, абсцисната ос и правите x = p и x = q, където a, b, c, p и q са дадени реални параметру, p < q.

	\item Задача 4.12. Да се напише булева функция, която проверява дали дата, зададена в следния формат: dd.mm.yyyy е коректна дата от грегорианския календар. 

	\item Задача 4.13. 

	Дадено е естествено число $n (1 < n \leq 10000)$. Да се напише програма, която намира и извежда на екрана всички  прости числа, по-малки от n.

	\item Задача 4.25. 

	Да се дефинира процедура, която получава целочислен параметър $n$ и база на бройна система $k$. Процедурата да отпечатва на екрана представянето на числото $n$ в системата $k$.

\item Задача 2.57. 

Да се напише булева функция, която проверява дали сумата от цифрите на дадено като параметър положително цяло число е кратна на 3.

\item Задача 2.55. 

Да се напише булева функция, която проверява дали дадено естествено число е степен на 2.

\item Задача 2.64. 

Да се напише целочислена функция с параметри n и k, която намира цялото число, което се получава от положителното цяло число n като се задраска k–тата му отдясно наляво цифра. Например ако n е 31245 и k е 4, функцията трябва да намери числото 3245; ако k е 2, функцията трябва да намери числото 3125, а ако n е 5 и k е 1, функцията трябва да намери числото 0.

\item Задача 2.81. 

Едно положително цяло число е съвършено, ако е равно на сумата от своите делители (без самото число). Например, 6 е съвършено, защото 6 = 1+2+3; числото 1 не е съвършено. Да се напише процедура, която намира и отпечатва на екрана  всички съвършени числа, ненадминаващи дадено положително цяло число в параметър n.

\item Задача 5.15. 

Да се дефинира функция, която заменя всяко срещане на цифрата 5 в дадено неотрицателно цяло число с 8.




\end{enumerate}


	\vspace{20px}

	\small{Задачите са от сборника \textit{Магдалина Тодорова, Петър Армянов, Дафина Петкова, Калин Николов, ``Сборник от задачи по програмиране на C++. Първа част. Увод в програмирането''}. Запазена е номерацията в сборника.}


\end{document}

