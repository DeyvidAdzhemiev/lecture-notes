\documentclass[12pt,a4paper]{article}
\usepackage[utf8]{inputenc}
\usepackage{amsmath}
\usepackage{amsfonts}
\usepackage{amssymb}
\usepackage{listings}
\usepackage{url}
\usepackage[bulgarian]{babel}
\usepackage{listings}
\usepackage{enumerate}


\lstset{breaklines=true} 


\author{\textit{email: kalin@fmi.uni-sofia.bg}}
\title{\textsc{Задачи за задължителна самоподготовка} \\
по \\
Увод в програмирането\\
\textit{матрици и вложени цикли}}



\begin{document}
\maketitle


\begin{enumerate}
	
	\item
	

		Да се напише програма, която въвежда от клавиатурата матрица от цели числа $A_{N\times N}$ и я транспонира:


	\item Задача 3.18. 

	Дадени са числовите редици $a_0, a_1, ..., a_{n-1}$ и $b_0, b_1, ..., b_{n-1}$ $(1 \leq n \leq 50)$. Да се напише програма, която въвежда от клавиатурата двете редици и намира броя на равенствата от вида $a_i = b_j$ $(i = 0, ..., n-1, j = 0, ..., n-1)$.


	\item Задача 3.21. 

	Две числови редици си приличат, ако съвпадат множествата от числата, които ги съставят. Да се напише програма, която въвежда числовите редици $a_0, a_1, ..., a_{n-1}$ и $b_0, b_1, ..., b_{n-1}$ $(1 \leq n \leq 50)$ и установява дали си приличат.

	\item Задача 3.27. 

	Дадени са две цели неотрицателни числа, представени с масиви от символи. Да се напише програма, която установява дали първото от двете числа е по-голямо от второто.


	\item Задача 3.29. 

Дадена е квадратна целочислена матрица A от n-ти ред $(1 \leq n \leq 50)$. Да се напише програма, която намира сумата от нечетните числа под главния диагонал на A (без него).

  \item Задача 3.45. 

  Матрицата А има седлова точка в $a_{i,j}$, ако $a_{i,j}$ е минимален елемент в i-тия ред и максимален елемент в j-тия стълб на A. Да се напише програма, която извежда всички седлови точки на дадена матрица А с размерност $n \times m (1 \leq n \leq 20, 1 \leq m \leq 30)$.


 \item Задача 3.58. (сортирано сливане). 

  Да се напише програма, която извежда на екрана резултата от сливането на два символни низа с максимум 20 символа, елементите на всеки от които са сортирани във възходящ ред. Например, ако низовете са ``aadgghxyyz'' и ``abcfijkl'', в резултат от сортирането сливане се получава низът ``aaabcdfgghijklxyyz''.

	\item Задача 3.113. (периодичност на масив). 

	Да се напише програма, която проверява дали в едномерен масив от цели числа съществува период. Например, ако масивът е с елементи 1, 2, 3, 1, 2, 3, 1, 2, 3, 1, периодът е 3. Ако период съществува, да се изведе.

	\item

	Да се напише програма, която въвежда от клавиатурата матрица от цели числа $A_{N\times N}$ и сортира:

	\begin{enumerate}
		\item всеки от редовете на матрицата;
		\item всяка от колоните на матрицата.
	\end{enumerate}

	Така получените матрици да се отпечатат на екрана в табличен вид.

	\item 
	Да се напише програма, която въвежда от клавиатурата матриците от цели числа $A_{N\times N}$ и $B_{N\times N}$ и извежда на екрана резултатът от умножението на двете матрици.



\end{enumerate}


	\vspace{20px}

	\small{Някои от задачите са от сборника \textit{Магдалина Тодорова, Петър Армянов, Дафина Петкова, Калин Николов, ``Сборник от задачи по програмиране на C++. Първа част. Увод в програмирането''}. За тези задачи е запазена номерацията в сборника.}



\end{document}

