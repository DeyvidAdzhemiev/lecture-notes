\documentclass[12pt,a4paper]{article}
\usepackage[utf8]{inputenc}
\usepackage{amsmath}
\usepackage{amsfonts}
\usepackage{amssymb}
\usepackage{listings}
\usepackage{url}
\usepackage[bulgarian]{babel}
\usepackage{listings}
\usepackage{enumerate}


\lstset{breaklines=true} 


\author{\textit{email: kalin@fmi.uni-sofia.bg}}
\title{\textsc{Задачи за задължителна самоподготовка} \\
по \\
Увод в програмирането\\
\textit{Рекурсия}}



\begin{document}
\maketitle


\begin{enumerate}
	
	\item Задача 5.2. 

	Да се дефинира рекурсивна функция за намиране на стойността на полинома на Ермит $Hn(x)$ (x е реална променлива, а n неотрицателна цяла променлива), дефиниран по следния начин:

	$H_0(x)=1$

	$H_1(x)=2x$

	$H_n(x)=2xH_{n-1}(x)+2(n-1)H_{n-2}(x), n>1$


	\item Задача 5.3. 

	Произведението на две положителни цели числа може да се дефинира по
следния начин:

	$mult (m,n) = m$, ако n = 1

	$mult (m,n) = m + mult (m,n-1)$, иначе.

	Да се дефинира рекурсивна функция, която намира произведението на две положителни цели числа по описания по-горе начин.

	\item Задача 5.5. 

	Да се дефинира функция, която намира най-големия общ делител на две неотрицателни цели числа, поне едното от които е различно от 0.

	\item Задача 5.7. 

	Дадени са естествените числа n и k $(n \ge 1, k > 1)$. Да се дефинира рекурсивна функция, която намира произведението на естествените числа от 1 до n със стъпка k.

	\item Задача 5.10. 

	Дадено е неотрицателно цяло число n в десетична бройна система. Да се дефинира рекурсивна функция, която намира сумата от цифрите на n в бройна система с основа k $(k > 1)$.

	\item Задача 5.11. 

	Да се дефинира рекурсивна функция, която установява дали в записа на неотрицателното цяло числo n, записано в десетична бройна система, се съдържа цифрата k.

	\item Задача 5.19. 

	Да се дефинира рекурсивна функция, която проверява дали дадено положително цяло число е елемент на редицата на Фибоначи.

	\item Задача 5.28. 

	Да се дефинира рекурсивна функция, която намира максималния елемент на редицата от цели числа $a_0, a_1, a_2, ..., a_{n-1}$, където $n \ge 1$.

	Забележка: Редицата е представена като масив.

	\item Задача 5.30. 

	Да се напише програма, която въвежда от клавиатурата n цели числа (n > 0) и след това ги извежда в обратен ред. За целта да се дефинира подходяща рекурсивна функция.

	\item Задача 5.31. 

	Да се напише функция 

	\texttt{void insertSorted (long x, long arr[], long n)}, 

	която включва цялото число \texttt{x} число в сортирана във възходящ ред редица от цели числа \texttt{arr}, в която има записани \texttt{n} елемента. Вмъкването да запазва наредбата на елементите. Предлолага се, че за редицата е заделена достатъчно памер за допълване с още едно число.

	\item Задача 5.34. Да се дефинира рекурсивна функция, която сравнява лексикографски два символни низа.


\end{enumerate}


	\vspace{20px}

	\small{Някои от задачите са от сборника \textit{Магдалина Тодорова, Петър Армянов, Дафина Петкова, Калин Николов, ``Сборник от задачи по програмиране на C++. Първа част. Увод в програмирането''}. За тези задачи е запазена номерацията в сборника.}



\end{document}

