\documentclass[12pt,a4paper]{article}
\usepackage[utf8]{inputenc}
\usepackage{amsmath}
\usepackage{amsfonts}
\usepackage{amssymb}
\usepackage{listings}
\usepackage{url}
\usepackage[bulgarian]{babel}
\usepackage{listings}
\usepackage{enumerate}
\usepackage{graphicx}
\usepackage{relsize}
\usepackage[framemethod=tikz]{mdframed}

\lstset{breaklines=true} 


\author{\textit{email: kalin@fmi.uni-sofia.bg}}
\title{\textsc{Задачи за задължителна самоподготовка} \\
по \\
Увод в програмирането\\
\textit{Структури и текстови файлове}}



\begin{document}
\maketitle


\begin{enumerate}
	
	\item Задача 1.1.

	Нека е дефинирана структурата Product:
	\begin{verbatim}
		struct Product
		{ 
		  char description[32];
		  //описание на изделие
		  int partNum;
		  //номер на изделие
		  double cost;
		  //цена на изделие
		};
		
	\end{verbatim}

	\begin{enumerate}
		\item Да се създадат две изделия и се инициализират чрез следните данни:

		\begin{tabular}{c | c | c}
			description & partNum & cost \\\hline
			screw driver & 456 & 5.50 \\\hline
			hammer & 324 & 8.2-0
		\end{tabular}

		\item Да се изведат на екрана компонентите на двете изделия;
		\item Да се дефинира масив от 5 структури Product. Елементите на масива да не се инициализират;
		\item Да се реализира цикъл, който инициализира масива чрез нулевите за съответния тип на полетата стойности;
		\item Да се променят елементите на масива така, че да съдържат следните стойности:

		\begin{tabular}{c | c | c}
			description & partNum & cost \\\hline
			screw driver & 456 & 5.50 \\\hline
			hammer & 324 & 8.20 \\\hline
			socket & 777 & 6.80 \\\hline
			plier & 123 & 10.80 \\\hline
			hand-saw & 555 & 12.80
		\end{tabular}
		\item Да се изведат елементите на масива на конзолата с подходящо форматиране;
		\item Да се изведат елементите на масива в текстов файл;
		\item Да се дефинира втори масив и неговите елементи да се инициализират чрез прочитане на записаните от предната точка данни в съответния файл;
		\item Да се изведат на конзолата елементите на втория масив и да се сравнят с елементите на първия. 

		\end{enumerate}


	\item Задача 1.4. 

	Да се дефинират структурите polar и rect, задаващи вектор с полярни и с правоъгълни координати съответно. Да се дефинират функции, които преобразуват вектор, зададен чрез правоъгълни координати, в полярни координати и обратно, както и функции, които извеждат вектор, зададен чрез полярните си и чрез правоъгълните си координати. 

	В главната функция да се дава възможност за избор на режим на въвеждане: r – за въвеждане в правоъгълни и p – в полярни координати. За всеки избран режим да се въведат произволен брой вектори, да се преобразуват в другия режим и да се изведат.

	\item Задача 1.5. 

	Да се дефинират структурите: Person, определяща лице по собствено име и фамилия и Client, определяща клиент като лице, притежаващо банкова сметка с дадена сума.

	Да се дефинират функции, които въвеждат и извеждат данни за лице и клиент. Да се напише програма, която:

	\begin{enumerate}
		\item въвежда от файл имената и банковите сметки на едномерен масив от клиенти;
		\item извежда на екрана имената и банковите сметки на клиентите от масива; 
		\item намира сумата от задълженията на клиентите от масива.

	\end{enumerate}


	\item Задача 1.8. 

	Да се дефинира функция, която сортира лексикографски във възходящ ред редица от точки в равнината. За целта да се дефинира структура Point, описваща точка от равнината с декартови координати.

	\item Задача 1.Б.5. 

	Да се дефинира структура Planet, определяща планета по име (символен низ), разстояние от слънцето, диаметър и маса (реални числа). Да се дефинират функции, изпълняващи следните действия:

	\begin{enumerate}
		\item въвежда данни за планета от клавиатурата;
		\item извежда данните за планета;
		\item  връща като резултат броя секунди, които са необходими на светлината да достигне от слънцето до планетата (да се приеме, че светлината има скорост 299792 km/s и че разстоянието на планетата до слънцето е зададено в километри.
		\item въвежда от файл множество от планети, реализирано чрез едномерен масив;
		\item извежда данните за планетите от масив, подаден на функцията като параметър;
		\item отпечатва данните за планетата с най-голям диаметър от масив, подаден на функцията като параметър;
	\end{enumerate}





\end{enumerate}

	\vspace{20px}

	\small{Някои от задачите са от сборника \textit{Магдалина Тодорова, Петър Армянов, Дафина Петкова, Калин Николов, ``Сборник от задачи по програмиране на C++. Втора част. Обектно ориентирано програмиране''}. За тези задачи е запазена номерацията в сборника.}

%\begin{thebibliography}{99}

%\bibitem{cindy}	David Matuszek, ``Backtracking'', https://www.cis.upenn.edu/~matuszek/cit594-2012/Pages/backtracking.html.

%\end{thebibliography}

\end{document}

