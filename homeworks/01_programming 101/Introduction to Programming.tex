\documentclass[12pt,a4paper]{article}
\usepackage[utf8]{inputenc}
\usepackage{amsmath}
\usepackage{amsfonts}
\usepackage{amssymb}
\usepackage{listings}
\usepackage{url}
\usepackage[bulgarian]{babel}
\usepackage{listings}
\usepackage{enumerate}
\usepackage[framemethod=tikz]{mdframed}
\usepackage{enumitem}


\lstset{breaklines=true}
\setenumerate[1]{label=\thesection.\arabic*.}
\setenumerate[2]{label*=\arabic*.}


\author{\textit{email: kalin@fmi.uni-sofia.bg}}
\title{\textsc{Задачи за задължителна самоподготовка} \\
по \\
Увод в програмирането}


\begin{document}
\maketitle


\small{Някои от задачите по-долу са решени в сборника \cite{sbornik}\textit{Магдалина Тодорова, Петър Армянов, Дафина Петкова, Калин Георгиев, ``Сборник от задачи по програмиране на C++. Първа част. Увод в програмирането''}. За задачите от сборника е посочена номерацията им от сборника.}

\vspace{20px}


\section {Увод, основи и примери}

\subsection {Основни примери}

\begin{enumerate}

	\item Превърнете рожденната си дата шестнадесетична, в осмична и в двоична бройни системи.

	\item Как бихте кодирали вашето име само с числа? Измислете собствено представяне на символни константи чрез редици от числа и запишете името си в това представяне.

	Разгледайте стандартната ASCii таблица (\texttt{http://www.asciitable.com/}) и запишете името си чрез серия от ASCii кодове.

\end{enumerate}

\subsection {Променливи, вход и изход, логически и аритметични операции, условен оператор}

\begin{enumerate}[resume]

	\item Задача 1.6.\cite{sbornik} Да се напише програма, която по зададени навършени години намира приблизително броя на дните, часовете, минутите и секундите, които е живял човек до навършване на зададените години.

	\item Задача 1.7.\cite{sbornik} Да се напише програма, която намира лицето на триъгълник по дадени: а) дължини на страна и височина към нея; б) три страни.


	\item Задача 2.7.\cite{sbornik} Да се напише програма, която въвежда координатите на точка от равнина и извежда на кой квадрант принадлежи тя. Да се разгледат случаите, когато точката принадлежи на някоя от координатните оси или съвпада с центъра на координатната система.

	\item Задача 1.14.\cite{sbornik} Да се запише булев израз, който да има стойност истина, ако посоченото условие е вярно и стойност - лъжа, в противен случай:

	\renewcommand{\theenumii}{\Alph{enumii}}

	\begin{enumerate}[label=\alph*)]%[a)] % a), b), c), ...
			 \item цялото число p се дели на 4 или на 7;
			 \item уравнението $ax^2 + bx + c = 0 (a \neq 0)$ няма реални корени;
			 \item точка с координати (a, b) лежи във вътрешността на кръг с радиус 5 и център (0, 1); г) точка с координати (a, b) лежи извън кръга с център (c, d) и радиус f;
			 \item точка принадлежи на частта от кръга с център (0, 0) и радиус 5 в трети квадрант;
			 \item точка принадлежи на венеца с център (0, 0) и радиуси 5 и 10;
			 \item x принадлежи на отсечката [0, 1];
			 \item x е равно на max \{a, b, c\};
			 \item x е различно от max \{ a, b, c\};
			 \item поне една от булевите променливи x и y има стойност true;
			 \item и двете булеви променливи x и y имат стойност true;
			 \item нито едно от числата a, b и c не е положително;
			 \item цифрата 7 влиза в записа на положителното трицифрено число p;
			 \item цифрите на трицифреното число m са различни;
			 \item поне две от цифрите на трицифреното число m са равни помежду си;
			 \item цифрите на трицифреното естествено число x образуват строго растяща или строго намаляваща редица;
			 \item десетичните записи на трицифрените естествени числа x и y са симетрични;
			 \item естественото число x, за което се знае, че е по-малко от 23, е просто.
  \end{enumerate}


\item Задача 2.12.\cite{sbornik} Да се напише програма, която проверява дали дадена година е високосна.

\end{enumerate}

\subsection {Цикли}

\begin{enumerate}[resume]


	\item Задача 1.20.\cite{sbornik} Да се напише програма, която по въведени от клавиатурата цели числа x и k ($k \geq 1$) намира и извежда на екрана k-тата цифра на х. Броенето да е отдясно наляво.

	\item Задача 2.40.\cite{sbornik} Да се напише програма, която (чрез цикъл for) намира сумата на всяко трето цяло число, започвайки от 2 и ненадминавайки n (т.е. сумата 2 + 5 + 8 + 11 + ...).

	\item Задача 2.44.\cite{sbornik} Дадено е естествено число n ($n \geq 1$). Да се напише програма, която намира броя на тези елементи от серията числа $i^3 + 13 \times i \times n + n
	^3$ , $i = 1, 2, ..., n$, които са кратни на 5 или на 9.

	\item За въведени от клавиатурата естествени числа $n$ и $k$, да се провери и изпише на екрана дали $n$ е точна степен на числото $k$.

	\textit{Упътване: Разделете променливата $n$ на променливата $k$ ``колкото пъти е възможно'' и проверете дали $n$ достига единица или някое друго число след края на процеса. Използвайте добре подбрано условие за for цикъл, оператора \% за намиране на остатък при целочислено деление, и оператора за целочислено деление /.}

\end{enumerate}


\subsection {Машини с неограничени регистри}


\small{Дефиницията на Машина с неогрничени регистри по-долу е взаимствана от учебника \cite{tprog}\textit{А. Дичев, И. Сосков, ``Теория на програмите'', Издателство на СУ, София, 1998}.

\vspace{20px}

\begin{mdframed}[hidealllines=true,backgroundcolor=gray!20]

	``Машина с неограничени регистри'' (или МНР) наричаме абстрактна машина, разполагаща с неограничена памет. Паметта на машината се представя с безкрайна редица от естествени числа $m[0],m[1],...$, където $m[i] \in \mathcal{N}$. Елементите $m[i]$ на редицата наричаме ``клетки'' на паметта на машината, а числото $i$ наричаме ``адрес'' на клетката $m[i]$.

	 МНР разполага с набор от инструцкии за работа с паметта. Всяка инструкция получава един или повече параметри (операнди) и може да предизвика промяна в стойността на някоя от клетките на паметта. Инструкциите на МНР за работа с паметта са:

	\begin{enumerate}[label=\arabic*)]
		\item \texttt{ZERO n}: Записва стойността 0 в клетката с адрес $n$
		\item \texttt{INC n}: Увеличава с единица стойността, записана в клетката с адрес $n$
		\item \texttt{MOVE x y}: Присвоява на клетката с адрес $y$ стойността на клетката с адрес $x$
	\end{enumerate}

	``Програма'' за МНР наричаме всяка последователност от инструкции на МНР и съответните им операнди. Всяка инструкция от програмата индексираме с поредния ѝ номер. Изпълнението на програмата започва от първата инструкция и преминава през всички инструкции последователно, освен в някои случаи, опиани по-долу. Изпълнението на програмата се прекратвя след изпълнението на последната ѝ инструкция. Например, след изпълнението на следната програма:

	\begin{verbatim}
	0: ZERO 0
	1: ZERO 1
	2: ZERO 2
	3: INC 1
	4: INC 2
	5: INC 2
	\end{verbatim}

	Първите три клетки на машината ще имат стойност 0, 1, 2, независимо от началните им стойности.

	Освен инструкциите за работа с паметта, МНР притежават и една инструкция за промяна на последователноста на изпълнение на програмата:

	\begin{enumerate}[label=\arabic*)]
	\setcounter{enumi}{3}
		\item \texttt{JUMP x}: Изпълнението на програмата ``прескача'' и продължава от инструкцията с пореден номер $x$. Ако програмата има по-малко от $x+1$ инструкции, изпълнението ѝ се прекратява
		\item \texttt{JUMP x y z}: Ако съдържанията на клетките  $x$ и $y$ съвпадат, изпълнението на програмата ``прескача'' и продължава от инструкцията с пореден номер $z$. В противен случай, програмата продължава със следващата инструкция. Ако програмата има по-малко от $z+1$ инструкции, изпълнението ѝ се прекратява
	\end{enumerate}

	Например, нека изпълнето на следната програма започва при стойности на клиетките на паметта 10,0,0,...:

	\begin{verbatim}
	0: JUMP 0 1 5
	1: INC 1
	2: INC 2
	3: INC 2
	4: JUMP 0

	\end{verbatim}

	След приключване на програмата, първите три клетки на машината ще имат стойности 10, 10, 20.

\end{mdframed}

Задачи:

\begin{enumerate}[resume]
	\item Нека паметта на МНР е инициалирана с редицата $m,n,0,0,...$. Да се напише програма на МНР, след изпълнението на която клетката с адрес 2 съдържа числото $m+n$.
	\item Нека паметта на МНР е инициалирана с редицата $m,n,0,0,...$. Да се напише програма на МНР, след изпълнението на която клетката с адрес 2 съдържа числото $m \times n$.
	\item Нека паметта на МНР е инициалирана с редицата $m,n,0,0,...$. Да се напише програма на МНР, след изпълнението на която клетката с адрес 2 съдържа числото 1 тогава и само тогава, когато $m>n$ и числото 0 във всички останали случаи.
\end{enumerate}


\subsection {Типове и функции}

\begin{enumerate}
	\item Задача 4.12.\cite{sbornik} Да се напише булева функция, която проверява дали дата, зададена в следния формат: dd.mm.yyyy е коректна дата от грегорианския календар.
	\item Задача 4.25.\cite{sbornik} Да се дефинира процедура, която получава целочислен параметър $n$ и база на бройна система $k \leq 16$. Процедурата да отпечатва на екрана представянето на числото $n$ в системата с база $k$.
	\item Задача 2.57. \cite{sbornik}	Да се напише булева функция, която проверява дали сумата от цифрите на дадено като параметър положително цяло число е кратна на 3.
	\item Задача 2.81. \cite{sbornik} Едно положително цяло число е съвършено, ако е равно на сумата от своите делители (без самото число). Например, 6 е съвършено, защото 6 = 1+2+3; числото 1 не е съвършено. Да се напише процедура, която намира и отпечатва на екрана  всички съвършени числа, ненадминаващи дадено положително цяло число в параметър n.

\end{enumerate}


\begin{thebibliography}{99}

\bibitem{sbornik}	Магдалина Тодорова, Петър Армянов, Дафина Петкова, Калин Георгиев, ``Сборник от задачи по програмиране на C++. Първа част. Увод в програмирането''
\bibitem{tprog} А. Дичев, И. Сосков, ``Теория на програмите'', Издателство на СУ, София, 1998

\end{thebibliography}

\end{document}
