\documentclass[12pt,a4paper]{article}
\usepackage[utf8]{inputenc}
\usepackage{amsmath}
\usepackage{amsfonts}
\usepackage{amssymb}
\usepackage{listings}
\usepackage{url}
\usepackage[bulgarian]{babel}
\usepackage{listings}
\usepackage{enumerate}
\usepackage[framemethod=tikz]{mdframed}
\usepackage{enumitem}
\usepackage{relsize}
\usepackage{hyperref}
\usepackage{tikz}
\usetikzlibrary{shapes,arrows,positioning,calc}

\lstset{breaklines=true}

\setenumerate[1]{label=\thesection.\arabic*.}
\setenumerate[2]{label*=\arabic*.}

\newcommand{\code}[1]{\texttt{#1}}

\hypersetup{
    colorlinks,
    citecolor=black,
    filecolor=black,
    linkcolor=black,
    urlcolor=black
}


\tikzset{
block/.style = {draw, fill=white, rectangle,align = center},
entry/.style = {draw, fill=black, circle, radius=3em},
condition/.style = {draw, fill=white, diamond, align = center,node distance=3cm},
fork/.style = {draw, fill=black, circle,inner sep=1pt},
cut/.style = {draw, fill=white, circle,inner sep=3pt, font = \tiny}
}



\author{\textit{email: kalin@fmi.uni-sofia.bg}}
\title{\textsc{Задачи за задължителна самоподготовка} \\
по \\
Увод в програмирането}


\begin{document}
\maketitle

\section{Инварианта на цикъл}

\emph{Изложението в тази секция е силно опростена версия на метода на Флойд (Floyd) за верификация на блок схеми, описан в монументалната му статия \cite{floyd}. Редица идеи и задачи са взаимствани от книгата \cite{tpsbornik} на А. Скоскова и С. Николова ``Теория на програмите в задачи''. За по-подробно изложение и допълнителни примери и задачи препоръчваме тази книга. Изложението по-долу е нарочно опростено, като на места е жертвана прецизността му.}

\begin{mdframed}[hidealllines=true,backgroundcolor=gray!20]


\bigskip
\textbf{Инварианта на цикъл}
\bigskip


Нека е дадена програма или част от програма, кято се се състои от единствен \code{while} цикъл. Нека имаме някакъв набор от работни променливи $v_1,..,v_n$, които се инициализрат непосредствено преди \code {while} цикъла. Условието на цикъла зависи изцяло от работните променливи, а тялото на цикъла използва или променя само тях. Приемаме, че програмата не проивежда странични ефекти и не зависи от такива.

Тоест, за програмата знаем, че: (1) не извършва входни и изходни операции, (2) поведението ѝ зависи изцяло от началните стойности на работните променливи $v_1,..,v_n$ и (3) не модифицира никакви други променливи, освен работните. Такава програма можем да изобразим чрез блок схемата на Фигура \ref{fig:1while}.
\end{mdframed}


\begin{figure}
    \begin{tikzpicture}[auto, node distance=1.5cm,>=latex']
    \node [entry, name=start](start){};
    \node [block,name=init, below of = start] (init)
       {Инициализация на\\ $v_0,..,v_n$};
    \node [fork,name=test1fork,below of = init,node distance = 2cm]{};
    \node [condition,name=test1, below of = test1fork,node distance = 3cm] (test1) {Условие \\ $C(v_0,..,v_n)$};
    \node [block,name=inc,right of = test1, node distance = 6.5cm] (inc) {Операции \\$(v_0^i,..,v_n^i) \rightarrow (v_0^{i+1},..,v_n^{i+1})$};
    \node [fork,name=endfork,below of = test1, node distance = 3cm]{};
    \node [entry, name=end, below of = endfork, node distance = 1cm](end){};

    \node [cut, name=cut1, left of = test1fork, node distance = 2cm](cut1){1};
    \node [cut, name=cut2, left of = endfork, node distance = 2cm](cut2){2};

    \draw [->] (start) -- (init);
    \draw [-] (init) -- (test1fork);
    \draw [->] (test1fork) -- (test1);
    \draw [->] (test1) -- node{true} (inc);
    \draw [->] (inc) |- (test1fork);
    \draw [-] (test1) -- node []{false}(endfork);
    \draw [->] (endfork) -- (end);

    \draw [-,dashed] (cut1) -- (test1fork);
    \draw [-,dashed] (cut2) -- (endfork);
    \end{tikzpicture}

  \caption{Блок схемa на част от програма с \code{while} цикъл}
  \label{fig:1while}
\end{figure}


\begin{mdframed}[hidealllines=true,backgroundcolor=gray!20]
На фигурата, $C(v_1,..,v_n)$ е някакъв логически израз, зависещ само от $v_1,..,v_n$. Да допуснем, че сме ``паузирали'' изпълнението на програмата в точката, обозначена с ``1'', точно преди да започне $i$-тото поред ($i\geq 0$) изпълнение на тялото на цикъла, т.е. непосредствено преди да бъдат изпълнени операторите в него за $i$-ти път. В този момент, $n$-торката работни променливи $(v_1,..,v_n)$ имат някакви конкретни стойности. Да ги обозначим с $(v_0^{i},..,v_n^{i})$. След изпълнение на всички оператори в тялото на цикъла, работните променливи получават някакви нови стойности. Това са точно стойностите $(v_0^{i+1},..,v_n^{i+1})$, които работните променливи ще имат в началото на $i+1$-вата итерация. Това е изобразено на фигурата чрез прехода $(v_0^i,..,v_n^i) \rightarrow (v_0^{i+1},..,v_n^{i+1})$.

Нека приемем, че при някакво конкретно изпълнение на програмата, при конкретни дадени начални стойности на работните променливи  $(v_0^{0},..,v_n^{0})$, тялото на цикъла ще бъде изпълнено точно  $k\geq 0$ пъти, след което условието $C(v_1,..,v_n)$ на цикъла ще се наруши и той ще приключи. Изпълнението на програмата достига точката, обозначена с ``2''. По този начин се получава редицата от $n$-торки $(v_0^{0},..,v_n^{0}),..,(v_0^{k},..,v_n^{k})$, като за всички нейни членове $0\leq i < k$, освен за последния, знаем че е вярно $C(v_0^{i},..,v_n^{i})$, а за последния, $k$-ти член, е вярно обратното -- $\neg C(v_0^{k},..,v_n^{k})$.

Освен свойството $C(v_0^{i},..,v_n^{i})$ (или неговото отрицание), което знаем за всички последователни стойности на работните променливи, можем да въведем още едно свойство $I(v_0^{i},..,v_n^{i})$, наречено ``инварианта на цикъла''. За свойството $I$ искаме да е вярно за всички възможни стойности на работните променливи, дори и за последния член на редицата им. От там идва името ``инварианта'', т.е. факт, което е винаги верен.

Това свойство може да е произволно, например тъждествената истина \code{true} изпълнява условието за инварианта, тъй като е вярна за всички членове на редицата на работните променливи. Инвариантата обаче е най-полезна, когато закодира ``смисъла'' на работните променливи, и ако от верността на $\neg C(v_0^{k},..,v_n^{k}) \&\& I(v_0^{k},..,v_n^{k})$ можем да изведем нещо полезно за ``крайният резултат'' от изпълнението на цикъла, съдържащ се в стойностите $(v_0^{k},..,v_n^{k})$.
\end{mdframed}

\begin{figure}
    \begin{tikzpicture}[auto, node distance=1.5cm,>=latex']
    \node [entry, name=start](start){};
    \node [block,name=init, below of = start, align=left] (init)
       {\code{candidate = 0;}\\\code{current = 1;}};

    \node [fork,name=test1fork,below of = init,node distance = 2cm]{};

    \node [block,name=inc2, right of = test1fork, node distance = 5cm,minimum height=1.5cm]{\code{current++;}};

    \node [condition,name=test1, below of = test1fork,node distance = 3cm] (test1) {\code{current < m}};

    \node [condition, name=test2, right of = test1, node distance = 5cm,font=\tiny] (test2) {\code{A[current]}\\\code{<A[candidate]}};

    \node [block,name=inc,right of = test2, node distance = 5cm, minimum height=1.5cm] (inc) {\code{candidate = current};};

    \node [fork,name=endfork,below of = test1, node distance = 3cm]{};
    \node [entry, name=end, below of = endfork, node distance = 1cm](end){};

    \node [cut, name=cut1, left of = test1fork, node distance = 2cm](cut1){1};
    \node [cut, name=cut2, left of = endfork, node distance = 2cm](cut2){2};

    \draw [->] (start) -- (init);
    \draw [-] (init) -- (test1fork);
    \draw [->] (test1fork) -- (test1);

    \draw [->] (test1) -- node{true} (test2);
    \draw [->] (test2) -- node{true} (inc);

    \draw [->] (inc) |- (inc2);
    \draw [->] (inc2) -- (test1fork);
    \draw [->] (test2) -- node{false} (inc2);


    \draw [-] (test1) -- node []{false}(endfork);
    \draw [->] (endfork) -- (end);

    \draw [-,dashed] (cut1) -- (test1fork);
    \draw [-,dashed] (cut2) -- (endfork);
    \end{tikzpicture}

  \caption{Блок схемa на програмата за намиране на най-малък елемент в масив}
  \label{fig:minel}
\end{figure}



\begin{mdframed}[hidealllines=true,backgroundcolor=gray!20]

\bigskip
\textbf{Верификация на програми}
\bigskip

Понятието ``Инварианта'' ще илюстрираме чрез едно нейно приложение: ``Верификация на програма''. Верификацията на програма е доказателство, че при необходимите начални условия дадена програма изчислява стойност, удовлетворяваща някакво желано свойство. Като пример да разгледаме следната  програма, за която ще се уверим, че намира най-малък елемент на едномерен масив $A$ с $m>0$ елемента $A[0],..,A[m-1]$.


\begin{verbatim}
size_t candidate = 0, current = 1;
while (current < m)
{
  if (A[candidate] < a[current])
  {
    candidate = current;
  }
  current++;
}
\end{verbatim}
Можем да приоложим метода на ивариантата, за да докажем строго, че когато цикълът приключи, то елементът $A[candidate]$ е гарантирано по-малък или равен на всички останали елементи на масива.
Като първа стъпка, за улеснение изобразяваме програмата като блок схемата на Фигура \ref{fig:minel}.


Работните променливи на програмата, освен масива $A$, са целочислените променливи \code{candidate} и \code{current}. Какъв е смисълът на тези променливи? Веднага се вижда, че \code{current} е просто брояч - служи за обхождане на елементите на масива последователно от $A[0]$ до $A[m-1]$, като на всяка итерация от цикъла се разглежда стойността на $A[current]$.

Интуитивно се вижда, че \code{candidate} е намерения най-малък елемент на масива до текущия момент от обхождането. Тоест можем да се надяваме, че ако сме разгледали първите $i$ елемента на масива, то правилно сме определили, че най-малкият от тях е $A[current]$.

Тези размишления може да запишем чрез инвариантата:

$I ::= A[candidate]=min(A[0],..,A[current-1])$.

Тази инварианта очевидно е вярна в началото на изпълнението на програмата, когато $current=1$, а $candidate=0$. Как да се уверим, че инвариантата е валидна за всички стойности, през които преминават работните променливи?

Нека допуснем, че инвариантата е вярна за някаква стъпка $i$ от изпълнението на цикъла. Тоест, допускаме $I(candidate^i, current^i)$, или $A[candidate^i]=min(A[0],..,A[current^i-1])$.

На итерация $i+1$ имаме  $candidate^{i+1}=current^{i}$ и $current^{i+1}=current^{i}+1$, и следователно трябва да покажем, че $A[candidate^{i+1}]=min(A[0],..,A[current^{i+1}-1])$.

След навлизане в тялото на цикъла, имаме две възможности:
\begin{enumerate}[label=\arabic*)]
  \item $A[current^i]\geq A[candidate^i]$. От допускането следва, че добавянето на $A[current^i]$ към $min(A[0],..,A[current^i-1])$ не променя стойността на минимума, или $min(A[0],..,A[current^i-1])=min(A[0],..,A[current^i-1],A[current^i])$. От тук директно се вижда, че инвариантата е вярна за интерация $i+1$.
  \item $A[current^i] < A[candidate^i$]. Тъй като по допускане $A[candidate^i] = min(A[0], .., A[current^i-1])$, от тук следва, че $A[current^i] < min(A[0], .., A[current^i-1])$, следователно

   $A[current^i] = min(A[0], .., A[current^i-1], A[current^i])$.

   Но след присвояването имаме $candidate^{i+1} = current^i$, от където $A[candidate^{i+1}] = min(A[0],..,A[current^i])$. Но $current^i = current^{i+1}-1$, от където получаваме верността на $I$ за итерация $i+1$: $A[candidate^{i+1}] = min(A[0], .., A[current^{i+1}-1])$.

\end{enumerate}
  Доказателството протича по индукция. Уверяваме се, че инвариантата е вярна за цялата редица от стойности на \code{candidate} и \code{current}.

  Какво се случва в края на цикъла, когато имаме $\neg(current<m)$? Тъй като числата са цели, а \code{current} се увеличава само с единица, можем да заключим, че $current=m$. Замествайки това равенствио в инвариантата получаваме твърдението

  $A[candidate] = min(A[0], .., A[m-1])$.

  Тоест, намерили сме минималния елемент на масива.
\end{mdframed}

\begin{enumerate}
  \item Да се докаже строго, че за следната функция е вярно $pow(x,y)=x^y$:
  \begin{enumerate}[label=\alph*)]%[a)] % a), b), c), ...

    \item
\begin{verbatim}
unsigned int pow (unsigned int x, usnigned int y)
{
  unsigned int p = 1, i = 0;
  while (i < y)
  {
    p *= x;
    i++;
  }
  return p;
}
\end{verbatim}
    \item
\begin{verbatim}
unsigned int pow (unsigned int x, usnigned int y)
{
  unsigned int p = 1;
  while (y > 0)
  {
    p *= x;
    y--;
  }
  return p;
}
\end{verbatim}
      \item
\begin{verbatim}
unsigned int pow (unsigned int x, usnigned int y)
{
  unsigned int z = x, t = y, p = 1;
  while (t > 0)
  {
    if (t%2 == 0)
    {
      z *= z;
      t /= 2;
    }else{
      t = t - 1;
      p *= z;
    }
  }
  return p;
}
\end{verbatim}
  \end{enumerate}

  \item Да се докаже строго, че за следната функция е вярно $sqrt(n)=[\sqrt{n}]$.

\begin{verbatim}
unsigned int sqrt (unsigned int n)
{
  unsigned int x = 0, y = 1, s = 1;
  while (s <= n)
  {
    x++;
    y += 2;
    s += y;
  }
  return x;
}
\end{verbatim}

\item Да се напише програма, която проверява дали дадени два масива $A$ и $B$ с еднакъв брой елементи $m$ са еднакви, т.е. съдържат същите елементи в същия ред. Да се докаже строго, че програмата работи правилно.

\item Да се докаже, че следната функция връща истина тогава и само тогава, когато масивът \code{A} с \code{n} елемента съдържа елемента \code{x}.

\begin{verbatim}
bool member (int A[], size_t n, int x)
{
  size_t i = 0;
  while (i < n && A[i] != x)
  {
    i++;
  }
  return i < n;
}
\end{verbatim}

\item Да се напише програма, която проверява дали даден масив $A$ с $m$ елемента е сортиран, т.е. дали елементите му са наредени в нарастващ ред. Да се докаже строго, че програмата работи правилно.

\end{enumerate}

\begin{mdframed}[hidealllines=true,backgroundcolor=gray!20]
Съществуват редица съвременни методи за верификация на програми. Пълната версия на метода, използван по-горе, се нарича ``логика на Флойд-Хоар'' (Floyd–Hoare logic) и е само един представител на този клас методи.

Също така, в компютърните науки има направление, наречено ``Синтез на програми'' (Program synthesis). При синтеза на програми се решава обратната задача: по спецификация на входа и изхода да се генерира програма, която удовлетворява спецификацията.

Препоръчваме на любознателния читател да се запознае с логиката на Floyd–Hoare и методите за синтез на програми.
\end{mdframed}


\pagebreak

\begin{thebibliography}{99}

\bibitem{tpsbornik}	Александра Соскова, Стела Николова, ``Теория на програмите в задачи'', Софтех, 2003
\bibitem{floyd} R. W. Floyd. ``Assigning meanings to programs.'' Proceedings of the American Mathematical Society, Symposia on Applied Mathematics. Vol. 19, pp. 19–31. 1967.

\end{thebibliography}

\end{document}
