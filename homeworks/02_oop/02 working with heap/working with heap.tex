\documentclass[12pt,a4paper]{article}
\usepackage[utf8]{inputenc}
\usepackage{amsmath}
\usepackage{amsfonts}
\usepackage{amssymb}
\usepackage{listings}
\usepackage{url}
\usepackage[bulgarian]{babel}
\usepackage{listings}
\usepackage{enumerate}


\lstset{breaklines=true} 


\author{\textit{email: kalin@fmi.uni-sofia.bg}}
\title{\textsc{Задачи за задължителна самоподготовка} \\
по \\
Обектно-ориентирано програмиране\\
\textit{Работа с heap}}



\begin{document}
\maketitle


\begin{enumerate}


\item Задача 1.4.24. Да се дефинира функция \texttt{strduplicate}, която създава копие на символен низ. Функцията да се грижи за заделянето на памет за новия низ.

\item Задача 1.4.25. Да се дефинира функция, която преобразува положително цяло число в съответния му символен низ и връща така построения символен низ.

\item Задача 1.4.30. Обединение на два символни низа $s_1$ и $s_2$ наричаме всеки символен низ, който съдържа без повторение всички символи на $s_1$ и $s_2$. Да се дефинира функция, която намира и връща обединението на два символни низа.

\item Задача 1.4.36. За работа със символни низове могат да бъдат използвани следните основни функции:

\begin{itemize}
  
\item \texttt{char car(const char* x)}, която връща първия символ (елемент) на низа x;
\item \texttt{char* cdr(char* x)}, която връща останалата част от низа x след отделянето на първия елемент на низа x;
\item \texttt{char* cons(char x, const char* y)}, която връща указател към символен низ, разположен в динамичната памет и съдържащ конкатенацията на символa x със символния низ y;

\item \texttt{bool eq(const char* x, const char* y)}, която връща true тогава и само тогава, когато низовете съвпадат.


\end{itemize}
Да се дефинират описаните функции. Като се използват тези функции да се дефинират следните функции:

\begin{itemize}

\item \texttt{char* reverse(char* x)}, която връща указател към символен низ, разположен в динамичната памет и съдържащ символите на x, записани в обратен ред;
\item  \texttt{char* copy(char* x)}, която връща указател към символен низ, разположен в динамичната памет и съдържащ копие на символния низ x;
\item  \texttt{char* car\_n( char* x, int n)}, която връща указател към символен низ, разположен в динамичната памет и съдържащ първите n символа на символния низ x;
\item \texttt{char* cdr\_n(char* x, int n)}, която връща останалата част от низа x след отделянето на първите n символа. Предварително е известно, че x притежава поне n символа;
\item \texttt{int number\_of\_char( char* x, char ch)}, която намира колко пъти символът ch се среща в символния низ x;
\item \texttt{int number\_of\_substr( char* x, char* y)}, която намира колко пъти символният низ y се среща в символния низ x;
\item \texttt{char* delete\_substr(char* x, char* y)}, която връща указател към символен низ, разположен в динамичната памет и съдържащ символите на низа x, от който са изтрити всички срещания на символния низ y.

\end{itemize}


\end{enumerate}


	\vspace{20px}

	\small{Някои от задачите са от сборника \textit{Магдалина Тодорова, Петър Армянов, Калин Николов, ``Сборник от задачи по програмиране на C++. Част първа. Увод в програмирането''}. За тези задачи е запазена номерацията в сборника.}


\end{document}

