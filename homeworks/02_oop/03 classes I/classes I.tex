\documentclass[12pt,a4paper]{article}
\usepackage[utf8]{inputenc}
\usepackage{amsmath}
\usepackage{amsfonts}
\usepackage{amssymb}
\usepackage{listings}
\usepackage{url}
\usepackage[bulgarian]{babel}
\usepackage{listings}
\usepackage{enumerate}


\newcommand{\code}[1]{\texttt{#1}}

\lstset{breaklines=true}


\author{\textit{email: kalin@fmi.uni-sofia.bg}}
\title{\textsc{Задачи за задължителна самоподготовка} \\
по \\
Обектно-ориентирано програмиране\\
\textit{Класове и оператори}}



\begin{document}
\maketitle


\begin{enumerate}

\item Задача 2.2.31. Да се дефинира клас \texttt{BankAccount}, определящ банкова сметка на клиент, състояща се от: име на клиент (символен низ до 50 символа), номер на банкова сметка (символен низ до 20 символа) и налична сума на клиент (реално число). Класът да притежава методи, чрез които може да:

\begin{itemize}
	\item инициализира банкова сметка;
	\item извежда на екрана информация за банкова сметка;
	\item внася пари в банкова сметка;
	\item тегли пари от банкова сметка.
\end{itemize}

Да се дефинира оператор за сравнение на две сметки по сумите в тях.

Да се дефинира главна функция, която създава две банкови сметки, извежда информацията в сметките, внася сума в една от сметките и тегли сума от другата сметка.

\item Задача 2.2.39. Да се дефинира клас \texttt{Time}, който определя момент от денонощието по зададени час и минути. Класът да съдържа подходящи методи за:

\begin{itemize}

 \item достъп и промяна на часа и минутите с проверки за коректност;
 \item добавящ към времето цяло число минути;
 \item достъп до боря минути, изминали от началото на денонощието;
 \item оператор за сравнение (казваме, че $t_1 < t_2$, ако $t_2$ е по-късно в денонощието от $t_1$).

\end{itemize}


Да се предефинират операторите +, - и *, така че да могат да се събират и изваждат две времена, както и да се умножават време с цяло число и цяло число с време. Да се включи дефинираният клас в програма и направят обръщения към член-функциите му и предефинираните оператори.


\item Задача 2.2.44. (асоциативен масив) Да се дефинира клас \texttt{Dictionary}, който създава тълковен речник. Тълковният речник се състои от не повече от 500 двойки дума–тълкувание, като думата е символен низ с не повече от 100 символа, а тълкованието е символен низ с не повече от 500 символа.

\begin{itemize}
	\item Да се дефинира подходяща структура, описваща една двойка дума-тълкувание;
	\item Да се дефинират подходящи член-данни на клас \texttt{Dictionary};
\end{itemize}

Клас \texttt{Dictionary} да съдържа методи, с които може да се извършват следните операции над речника:

\begin{itemize}

\item Инициализация на празен речник;
\item извеждане на всички думи в речника и техните тълкуания;
\item включване на нова двойка дума–тълкуване в речника;
\item изключване на двойка дума–тълкуване от речника (по дадена дума);
\item търсене на значението на дадена дума в речник.
\item извеждане на всички думи в речника и техните тълкуания по азбучен ред на думите;
\end{itemize}

Да се дефинира оператор +, обединяващ два речника, такъв че:

\begin{itemize}
 \item Ако някои думи имат значение и в двата речника, значенията да се конкатенират в резултатния сумарен речник;
 \item Ако общият брой на думите в двата речника надхвърля 500, да се използват само първите 500 думи (при произволна наредба).
\end{itemize}

\item Да се дефинира структура \code{Point}, описваща точка в евклидовата равнина и клас  \code{Line}, описващ права в евклидовата равнина, зададена чрез две нейни точки.

Класът \code{Line} да съдържа методи, чрез които може да се извършват следните операции:

\begin{itemize}
	\item Проверка дали две прави са успоредни;
	\item Проверка дали дадена точка лежи на дадена права;
	\item Намиране на пресечната точка на две прави. Приемаме, че правите не са успоредни. Стойността на резултата може да е произволна в противен случай.
	\item Създаване на права, която е ъглополовяща на по-големия ъгъл, образуван от две прави. Стойността на резултата може да е произволна в противен случай.
\end{itemize}

Където е подходящо да се дефинират оператори вместо методи.

\end{enumerate}


	\vspace{20px}

	\small{Някои от задачите са от сборника \textit{Магдалина Тодорова, Петър Армянов, Калин Николов, ``Сборник от задачи по програмиране на C++. Част втора. Обектно-ориентирано програмиране''}. За тези задачи е запазена номерацията в сборника.}


\end{document}
