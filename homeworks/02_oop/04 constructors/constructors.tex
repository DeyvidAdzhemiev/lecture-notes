\documentclass[12pt,a4paper]{article}
\usepackage[utf8]{inputenc}
\usepackage{amsmath}
\usepackage{amsfonts}
\usepackage{amssymb}
\usepackage{listings}
\usepackage{url}
\usepackage[bulgarian]{babel}
\usepackage{listings}
\usepackage{enumerate}


\newcommand{\code}[1]{\texttt{#1}}

\lstset{breaklines=true} 


\author{\textit{email: kalin@fmi.uni-sofia.bg}}
\title{\textsc{Задачи за задължителна самоподготовка} \\
по \\
Обектно-ориентирано програмиране\\
\textit{Конструктори}}



\begin{document}
\maketitle


\begin{enumerate}

\item Да се дефинира клас \code{Rat}, описващ рационално число. За класа да се дефинират оператори за събиране и умножение на рационални числа, както и подходящи конструктори. Да се дефинира функция\\

\code{Rat poly (Rat coef[], int n, Rat x)}\\

където \code{coef} е масив с \code{n + 1} рационални коефициента $a_0, a_1, ..., a_{n-1}, a_n$, а \code{x} е рационално число. \\

Функцията да намира стойността на полинома $P(x) = a_0x_n + a_1x_{n-1} + ... a_{n-1}x + a_n$.

Да се реализира и изпълни подходящ тест.



\item Да се дефинира клас Word, описващ дума, съставена от не повече от 20 символа от тип char. Класът да съдържа следните операции:

\begin{itemize}
\item оператор \code{[]} за намиране на \code{i}-тия пореден символ в думата
\item оператори \code{+} и \code{+=} за добавяне на един символ в края на думата. Ако думата вече има 20 символа, операторите да нямат ефект
\item оператори \code{<} и \code{==} за сравнение на думи спрямо лексикографската наредба
\item подходящи конструктори
	
\end{itemize}

Да се реализира и изпълни подходящ тест за класа и неговите методи.


\item Да се реализира клас \code{NumbersSummator},  който поддържа сума на цели числа. При създаване на обект от класа,  съответната му сума да се инициализира с число, което се подава като аргумент на конструктора. За класа да се реализират следните методи:

\begin{itemize}
\item sum, който връща текущата стойност на сумата
\item add, увеличаващ сумата с дадено число
\item sub, намаляващ сумата с дадено число
\item num, връща колко пъти сумата е била променяна
\item average, връщащ средното аритметично на всички числа, с които сумата е била променяна.
	
\end{itemize}


\textit{Забележка:}Функционалност извън тези 4 метода, като например съхраняване на отделните числа от поредицата, не е необходима. 
Пример: 
\begin{lstlisting}

	NumbersSummator seq1 (10); 
	seq1.add (10); 
	seq1.add (5); 
	seq1.sub (15); 
	cout << seq1.sum() ; //->10 (10+10+5-15)
	cout << seq1.average(); //->0 (10+5-15)/3
	
\end{lstlisting}



\item Да се дефинира клас \code{BrowserHistory}, който съдържа информация за историята на посещението до най-много \code{N} Web сайта. \code{N} е параметър на конструктора на класа. За целта да се реализира структура \code{HistoryEntry}, описваща едно посещение на сайт чрез:

\begin{enumerate}
	\item Месец от годината, през който е посетен сайтът;
	\item Неговото URL.
\end{enumerate}

Класът code{BrowserHistory} да поддържа следните операции:
\begin{itemize}
\item Метод за добавяне на нов сайт към историята. Информацията за всеки сайт се въвежда от клавиатурата 
\item Оператор += с параметър \code{HistoryEntry}, добавящ сайт към историята 
\item Метод за отпечатване на информацията за всички сайтове в историята
\item Метод, който по даден месец от годината намира броя на сайтовете, посетени през този месец
\item Намиране на този месец от годината, в който има най-много посетени сайтове
\item Премахване на най-скоро добавеният сайт в историята
\item Оператор +, който обединява двете истории
\end{itemize}

Да се реализира и изпълни подходящ тест за класа и неговите методи.






\end{enumerate}


	\vspace{20px}


\end{document}

