\documentclass[12pt,a4paper]{article}
\usepackage[utf8]{inputenc}
\usepackage{amsmath}
\usepackage{amsfonts}
\usepackage{amssymb}
\usepackage{listings}
\usepackage{url}
\usepackage[bulgarian]{babel}
\usepackage{listings}
\usepackage{enumerate}


\newcommand{\code}[1]{\texttt{#1}}

\lstset{breaklines=true} 


\author{\textit{email: kalin@fmi.uni-sofia.bg}}
\title{\textsc{Задачи за задължителна самоподготовка} \\
по \\
Обектно-ориентирано програмиране\\
\textit{Конструктори за копиране}}



\begin{document}
\maketitle


\begin{enumerate}

\item За клас \code{BrowserHistory} от предишните домашни да се реализират конструктор за копиране, оператор за присвояване и деструктор.\\

Да се реализира подходящ тест на класа.

\item Клас \code{Dictionary} от предишните домашни да се реализира така, че максималният брой \code{N} на двойки ключ-стойност, които могат да бъдат добавени към речника, да се задава като параметър на конструктора на класа. За класа да се реализират конструктор за копиране, оператор за присвояване и дейструктор.\\

Да се реализира подходящ тест на класа.



\end{enumerate}


	\vspace{20px}


\end{document}

