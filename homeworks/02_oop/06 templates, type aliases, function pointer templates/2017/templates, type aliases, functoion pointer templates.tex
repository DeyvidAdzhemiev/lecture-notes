\documentclass[12pt,a4paper]{article}
\usepackage[utf8]{inputenc}
\usepackage{amsmath}
\usepackage{amsfonts}
\usepackage{amssymb}
\usepackage{listings}
\usepackage{url}
\usepackage[bulgarian]{babel}
\usepackage{listings}
\usepackage{enumerate}
\usepackage{hyperref}


\newcommand{\code}[1]{\texttt{#1}}

\lstset{breaklines=true}


\author{\textit{email: kalin@fmi.uni-sofia.bg}}
\title{\textsc{Задачи за задължителна самоподготовка} \\
по \\
Обектно-ориентирано програмиране\\
\textit{Предефиниране на типове, функции от високо ниво, шаблони на указатели към функции}}



\begin{document}
\maketitle


\begin{enumerate}

\item Инсталирайте програмата \href{http://meldmerge.org/}{\texttt{meld}} и сравнете версиите на файловете от примера за йерархията от фигури от миналата седмица и от тази седмица. Разучете как чрез \texttt{git} да получите достъп и до двете версии на файловете.

\item Като се използва шаблона \texttt{std::vector} да се създаде вектор \texttt{M} от 3 елемента, чиито елементи са вектори от по 3 числа от тип \texttt{double} (Матрица $M_{3x3}$ с елементи от тип \texttt{double}). Да се въведат елементите на \texttt{M} от клавиатурата.

\item Да се добави нов оператор \texttt{<{}<} за изход в поток на \texttt{std::vector} така, че при печатане на масива от масиви (матрицата) \texttt{M} от предишната задача, елементите да се отпечатат като правоъгълна таблица - т.е. три реда, като на всеки ред има по три числа, разделени с интервали. Операторът да е универсален и да може да се ползва за всякакви двумерни матрици, построени чрез \texttt{std::vector} с елементи от всякакви (``разумни'') типове.

\item По подобие на функцията \texttt{map} за масиви, обсъдена на лекции, да се дефинира функция \texttt{map} за \texttt{std::vector} с елементи от тип \texttt{T}. Да се подготвят подходящи помощни функции така, че чрез приложение на функцията \texttt{map} всички елементи на матрицата \texttt{M} да се увеличат с единица. \\

Да се напише подходящ тест.

\item Да се създаде \texttt{std::vector} от числа. Да се подготви подходяща помощна функция така, че чрез приложение на метода \texttt{map} всички елементи на масива да се отпечатат на екрана.

\item Нека е дадена следната структура \texttt{struct S \{int a; int b; int c;\}}. Да се дефинира и попълни примерен вектор \texttt{A} с елементи от тип \texttt{S}.

\begin{enumerate}
	\item Чрез подходяща помощна функция и използване на \texttt{map}, да се отпечата сумата на полетата \texttt{a}, \texttt{b} и \texttt{c} на всеки от елементите на \texttt{A}. Да се напише подходящ тест.
	\item Чрез подходяща помощна функция и използване на \texttt{map}, да се въведат нови стойности на висчки полета на елементите на \texttt{A} от клавиатурата.
	\item Чрез подходяща помощна функция и използване на \texttt{map}, да се увеличат с единица всички полета \texttt{a} на елементите на \texttt{A}. Да се напише подходящ тест.
	\item Чрез подходяща помощна функция и използване на \texttt{map}, да се разменят стойностите на \emph{полетата} \texttt{a} и \texttt{b} на всеки от елементите на \texttt{A}. Да се напише подходящ тест.


\end{enumerate}

\end{enumerate}


	\vspace{20px}


\end{document}
