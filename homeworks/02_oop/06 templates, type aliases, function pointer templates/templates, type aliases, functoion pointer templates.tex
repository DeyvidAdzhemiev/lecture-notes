\documentclass[12pt,a4paper]{article}
\usepackage[utf8]{inputenc}
\usepackage{amsmath}
\usepackage{amsfonts}
\usepackage{amssymb}
\usepackage{listings}
\usepackage{url}
\usepackage[bulgarian]{babel}
\usepackage{listings}
\usepackage{enumerate}
\usepackage{hyperref}


\newcommand{\code}[1]{\texttt{#1}}

\lstset{breaklines=true} 


\author{\textit{email: kalin@fmi.uni-sofia.bg}}
\title{\textsc{Задачи за задължителна самоподготовка} \\
по \\
Обектно-ориентирано програмиране\\
\textit{Предефиниране на типове, функции от високо ниво, шаблони на указатели към функции}}



\begin{document}
\maketitle


\begin{enumerate}

\item Инсталирайте програмата \href{http://meldmerge.org/}{\texttt{meld}} и сравнете файловете \href{https://github.com/stranxter/lecture-notes/blob/master/samples/02_oop/2016/03_DynArray/dynarray-template.cpp}{\texttt{dynarray-template.cpp}} и \href{https://github.com/stranxter/lecture-notes/blob/master/samples/02_oop/2016/03_DynArray/dynarray.cpp}{\texttt{dynarray.cpp}}.

\item Като се използва шаблона \texttt{DynArr} да се създаде масив \texttt{M} то 3 елемента, чиито елементи са масиви от по 3 числа от тип \texttt{double}. Да се въведат елементите на \texttt{M} от клавиатурата. 

\item Да се добави нов оператор \texttt{<<} за изход в поток към шаблона \texttt{DinArr} така, че при печатане на масива от масиви (матрицата) \texttt{M} от предишната задача, елементите да се отпечатат като правоъгълна таблица - т.е. три реда, като на всеки ред има по три числа, разделени с интервали. Операторът да е универсален и да може да се ползва за матрици, построени чрез \texttt{DynArr} във всякакви размерности и с елементи от всякакви типове.

\item Да се подготвят подходящи помощни функции така, че чрез приложение на метода \texttt{DynArr::map} всички елементи на матрицата \texttt{M} да се увеличат с единица. 

\item Като се използва шаблона \texttt{DynArrr} да се създаде масив \texttt{M} от числа. Да се подготви подходяща помощна функция така, че чрез приложение на метода \texttt{DynArr::map} всички елементи на масива да се отпечатат на екрана.

\item Нека е дадена следната структура \texttt{struct S \{int a; int b; int c;\}}. Да се дефинира и попълни примерен масив \texttt{A} с елементи от  \texttt{S}.

\begin{enumerate}
	\item Да се дефинира подходяща функция \texttt{map}, получаваща като аргументи (1)масив от структури \texttt{S}, (2) дължината \texttt{n} на масива и (3) функция $f:S \rightarrow S$. 
	\item Чрез подходяща помощна функция и използване на \texttt{map}, да се отпечата сумата на полетата \texttt{a}, \texttt{b} и \texttt{c} на всеки от елементите на \texttt{A}.
	\item Чрез подходяща помощна функция и използване на \texttt{map}, да се въведат елементите на \texttt{A}.
	\item Чрез подходяща помощна функция и използване на \texttt{map}, да се увеличи с единица всяко поле \texttt{a} на елементите на \texttt{A}.
	\item Чрез подходяща помощна функция и използване на \texttt{map}, да се разменят стойностите на полетата \texttt{a} и \texttt{b} на елементите на \texttt{A}.
	\item Да се тестват решенията на горните задачи.
	\item Функцията \texttt{map} да се преработи така, че да е универсална за всички типове масиви. Да се повторят тестовете на горните задачи с новата версия на функцията. Да се направи тест с друг тип масив.

	

\end{enumerate}

\end{enumerate}


	\vspace{20px}


\end{document}

