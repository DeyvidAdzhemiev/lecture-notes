\documentclass[12pt,a4paper]{article}
\usepackage[utf8]{inputenc}
\usepackage{amsmath}
\usepackage{amsfonts}
\usepackage{amssymb}
\usepackage{listings}
\usepackage{url}
\usepackage[bulgarian]{babel}
\usepackage{listings}
\usepackage{enumerate}
\usepackage{hyperref}
\usepackage[framemethod=tikz]{mdframed}
\usepackage{relsize}


\newcommand{\code}[1]{\texttt{#1}}

\lstset{breaklines=true}


\author{\textit{email: kalin@fmi.uni-sofia.bg}}
\title{\textsc{Задачи за задължителна самоподготовка} \\
по \\
Обектно-ориентирано програмиране\\
\textit{Линеен едносвързан списък}}



\begin{document}
\maketitle

Следните задачи да се решат като упражнение за директно боравене с указателите и двойните кутии, вместо да се свеждат до използването на вече готови методи от реализацията на класа. Т.е. решенията на задачите да не ползват други методи, освен ако не са помощни функции, специално написани за тях.

\begin{enumerate}

	\item Да се реализират следните операции в клас LinkedList, разработен на лекции:

	\begin{enumerate}
	  \item Метод \code{int LinkedList::count(int x)}, който преброява колко пъти елементът \code{x} се среща в списъка.
		\item Конструктор с два аргумента $x$ и $y$ от тип $int$. Конструкторът създава списък с елементи $x, x+1, ..., y$, при положение, че $x \leq y$.
		\item Метод \code{removeAll (int x)}, който изтрива всички срещания на елемента \code{x} от списъка.
		\item Метод \code{$l_1$.append($l_2$)}, която добавя към края на списъка $l_1$ всички елементи на списъка $l_2$.
		\item Метод \code{concat}, който съединява два списъка в нов, трети списък. Т.е. \code{$l_1$.concat($l_2$)} създава и връща нов списък от елементите на \code{$l_1$}, следвани от елементите на \code{$l_2$}.
		\item Да се дефинират оператори \code{+=} и \code{+}, съответни на методите \code{append} и \code{concat}.
		\item Метод \code{map}, който прилага едноаргументна функция $f:int \rightarrow int$ към всеки от елементите на списъка.
		\item Метод \code{reverse}, който обръща реда на елементите на списъка. Например, списъкът с елементи $1,2,3$ ще се преобразува до списъка с елементи $3,2,1$.
		\item Клас \code{LinkedList} да се преобразува до шаблон, позволяващ произволен тип на елементите.
	\end{enumerate}
\end{enumerate}


	\vspace{20px}


\end{document}
