\documentclass[12pt,a4paper]{article}
\usepackage[utf8]{inputenc}
\usepackage{amsmath}
\usepackage{amsfonts}
\usepackage{amssymb}
\usepackage{listings}
\usepackage{url}
\usepackage[bulgarian]{babel}
\usepackage{listings}
\usepackage{enumerate}
\usepackage{hyperref}
\usepackage[framemethod=tikz]{mdframed}
\usepackage{relsize}


\newcommand{\code}[1]{\texttt{#1}}

\lstset{breaklines=true} 


\author{\textit{email: kalin@fmi.uni-sofia.bg}}
\title{\textsc{Задачи за задължителна самоподготовка} \\
по \\
Обектно-ориентирано програмиране\\
\textit{Линеен едносвързан списък}}



\begin{document}
\maketitle


\begin{enumerate}

	\item Да се добавят следните методи и оператори към шаблона на клас List:

	\begin{enumerate}
		\item метод \code{push\_back} за дабавяне на елемент от тип \code{T} към \textit{края} на списъка
		\item оператор \code{+=} за дабавяне на елемент от тип \code{T} към \textit{края} на списъка
		\item метод \code{get\_ith(int n)} за намиране на \code{n}-тия поред елемент на списъка
		\item оператор за индексиране, позволяващ чете и писане
		\item метод \code{append(l2)}, който добавя към списъка всички елементи на списъка \code{l2} (т.е. \code{l1.append(l2)} ``добавя'' списъка \code{l2} към края на списъка \code{l1})
		\item метод \code{concat}, който съединява два списъка в трети (т.е. \code{l1.concat(l2)} създава нов списък от елементите на \code{l1}, следвани от елементите на \code{l2}
		\item оператори \code{+=} и \code{+}, съответни на методите \code{append} и \code{concat}
		\item функция \code{map}
		\item функция \code{reduce}
	\end{enumerate}
\end{enumerate}


	\vspace{20px}


\end{document}

