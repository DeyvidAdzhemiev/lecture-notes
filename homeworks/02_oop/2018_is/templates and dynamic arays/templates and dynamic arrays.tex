\documentclass[12pt,a4paper]{article}
\usepackage[utf8]{inputenc}
\usepackage{amsmath}
\usepackage{amsfonts}
\usepackage{amssymb}
\usepackage{listings}
\usepackage{url}
\usepackage[bulgarian]{babel}
\usepackage{listings}
\usepackage{enumerate}


\lstset{breaklines=true}


\author{\textit{email: kalin@fmi.uni-sofia.bg}}
\title{\textsc{Задачи за задължителна самоподготовка} \\
по \\
Обектно-ориентирано програмиране\\
\textit{Шаблони и динамични масиви}}



\begin{document}
\maketitle


\begin{enumerate}
	\item Да се реализира шаблон на функция \texttt{void input ([подходящ тип] array, int n)}, която въвежда от клавиатурата стойностите на елементите на масива \texttt{array} от произволен тип \texttt{T} с големина \texttt{n}. \\

	\textit{Какви са допустимите типове \texttt{T} за този шаблон? Защо функцията е от тип \texttt{void}?}\\

	Да се реализира и изпълни подходящ тест за функцията.

	\item Да се реализира шаблон на фукнция \texttt{bool ordered ([подходящ тип] array, int n)}, която проверява дали елементите на масива \texttt{array} от произволен тип \texttt{T} с големина \texttt{n} образуват монотонно-растяща редица спрямо релацията \texttt{<}.\\

	\textit{Какви са допустимите типове \texttt{T} за този шаблон?}\\

	Да се реализира и изпълни подходящ тест за функцията.

	\item Да се реализира шаблон на фукнция \texttt{bool member ([подходящ тип] array, int n, [подходящ тип]x)}, която проверява дали \texttt{x} е елемент на масива \texttt{array} от произволен тип \texttt{T} с големина \texttt{n}.\\

	\textit{Има ли в C++ тип \texttt{T}, който не е съвместим с този шаблон?}\\

	Да се реализира и изпълни подходящ тест за функцията.

	\item За разработения на лекции клас \texttt{Array} да се добави метод \texttt{bool member (T x)}, който проверява дали елемента \texttt{x} е елемент на масива.

	Да се реализира и изпълни подходящ тест за метода.

	\item За разработения на лекции клас \texttt{Array} да се добави метод \texttt{bool remove (T x)}, който изтрива първото срещане на елемента \texttt{x} от масива (ако има такова).

	Да се реализира и изпълни подходящ тест за метода.

	\item За разработения на лекции клас \texttt{Array} да се добави оператор за събиране, с който могат да се конкатенират два масива. Да се реализира съответен оператор +=.

	Да се реализира и изпълни подходящ тест за метода.

\end{enumerate}




\end{document}
