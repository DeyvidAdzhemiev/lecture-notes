\documentclass[12pt,a4paper]{article}
\usepackage[utf8]{inputenc}
\usepackage{amsmath}
\usepackage{amsfonts}
\usepackage{amssymb}
\usepackage{listings}
\usepackage{url}
\usepackage[bulgarian]{babel}
\usepackage{listings}
\usepackage{enumerate}
\usepackage{hyperref}


\lstset{breaklines=true} 


\author{\textit{email: kalin@fmi.uni-sofia.bg}}
\title{\textsc{Задачи за задължителна самоподготовка} \\
по \\
Структури от данни и програмиране\\
\textit{Задачи върху интерпретатора}}



\begin{document}
\maketitle

Част 1: Кодирания на редици

\begin{enumerate}
	\item Да се дефинира функция \texttt{long encode (long arr[],long n)}, която кодира еднозначно n-елементния масив arr с естествено число. Да се ползва Гьоделовото кодиране (\href{http://en.wikipedia.org/wiki/G%C3%B6del_numbering#G.C3.B6del.27s_encoding}{Gödel's encoding}), показано на лекции или друго кодиране по Ваш избор.
	\item Да се дефинира функция \texttt{long decode (long code,long n)}, която намира n-тия елемент на редица, кодирана в числото code чрез функцията от предната задача.
	\item Как може да се кодира дължината на редицата?
\end{enumerate}

Част 2: Задачи върху разработвания на лекции интерпретатор

\begin{enumerate}
	\item Да се създаде опростен вариант на оператора \texttt{printline}, наречен \texttt{print}, който отпечатва единствен израз.
	\item Да се добави поддръжка на оператора \texttt{loop}, описан на лекции.
	\item Да се тестват новите оператори с примерни програми.
\end{enumerate}






\end{document}

