\documentclass[12pt,a4paper]{article}
\usepackage[utf8]{inputenc}
\usepackage{amsmath}
\usepackage{amsfonts}
\usepackage{amssymb}
\usepackage{listings}
\usepackage{url}
\usepackage[bulgarian]{babel}
\usepackage{listings}


\lstset{breaklines=true} 


\author{\textit{email: kalin@fmi.uni-sofia.bg}}
\title{\textsc{Задачи за задължителна самоподготовкa} \\
по \\
Структури от данни и програмиране}



\begin{document}
\maketitle


\begin{enumerate}

	\item Да се дефинира метод \texttt{HashMap::efficiency()}, който изчислява ефективността на хеш таблицата като отношението  $\frac{all-coliding}{all}$, където $coliding$ е броят на ключовете, записани при колизия, а $all$ е броят на всички записани ключове.


	\item Да се дефинира оператор \texttt{<}\texttt{<} за клас \texttt{HashMap}, който отпечатва в поток всички двойки ключ-стойност в Хеш таблицата.

	\item Да се напише програма, която въвежда от клавиатурата две текста с произволна големина $t_1$ и $t_2$. Програмата да извежда броя на всички срещания на думи в $t_2$, които се срещат и в $t_1$.

	Пример: за следните текстове

	\textit{In computing, a hash table (hash map) is a data structure used to implement an associative array, a structure that can map keys to values. A hash table uses a hash function to compute an index into an array of buckets or slots, from which the correct value can be found.}

	и 

	\textit{Ideally, the hash function will assign each key to a unique bucket, but this situation is rarely achievable in practice (usually some keys will hash to the same bucket)}

	Този брой е 10, съставен от думите \textit {the} (2 срещания във втория текст), \textit{a} (1 срещане), \textit{hash} (2), \textit {function} (1), \textit{to} (2), \textit{is} (1), \textit{keys} (1).


	\item Да се напише програма, която въвежда от клавиатурата две текста с произволна големина $t_1$ и $t_2$. Програмата да извежда броя на уникалните думи в $t_2$, които се срещат и в $t_1$.

	Пример: за двата текста от предишната задача, този брой е 7, съставен от думите \textit {the}, \textit{a}, \textit{hash}, \textit {function}, \textit{to}, \textit{is}, \textit{keys}.


	\item Да се напише програма, която прочита от входа даден текст с произволна големина и намира такава дума с дължина повече от 3 букви, която се среща най-често в текста. Пример: за текста

	\textit{In computing, a hash table (hash map) is a data structure used to implement an associative array, a structure that can map keys to values. A hash table uses a hash function to compute an index into an array of buckets or slots, from which the correct value can be found.}

	Най-често срещаната дума е \textit{hash}.

	\item От клавиатурата да се въведе цялото положително число $n$, следвано от $2 \times n$ цели положителни числа $a_1, b_1, a_2, b_2, ..., a_n, b_n$. Програмата да печата на екрана \texttt{``Yes''}, ако изображението, дефинирано като $h(a_i)=b_i,i=1,...,n$ е добре дефинирана функция. Т.е. програмата да проверява дали има два различни индекса $i$ и $j$, за които е изпълнено $a_i=a_j$, но $b_i \neq b_j$.


\end{enumerate}




\end{document}

