\documentclass[12pt,a4paper]{article}
\usepackage[utf8]{inputenc}
\usepackage{amsmath}
\usepackage{amsfonts}
\usepackage{amssymb}
\usepackage{listings}
\usepackage{url}
\usepackage[bulgarian]{babel}
\usepackage{listings}
\usepackage[framemethod=tikz]{mdframed}
\usepackage{relsize}




\lstset{breaklines=true} 


\author{\textit{email: kalin@fmi.uni-sofia.bg}}
\title{\textsc{Задачи за задължителна самоподготовка} \\
по \\
Структури от данни и програмиране}



\begin{document}
\maketitle


\begin{enumerate}

	\item Да се дефинира \texttt{operator *} на шаблона на хеш-таблицата. Хеш-таблицата \texttt{c}, която се получава при \texttt{c = a * b}, да съдържа като множество от ключове сечението на множествата на ключове на \texttt{a} и \texttt{b}, като стойноста на всеки ключ е двойка (\texttt{std::pair}) от съответните стойности от \texttt{a} и \texttt{b}. Хеш-функцията на \texttt{c} да е същата като на \texttt{b}.

	\item Да се дефинира \texttt{operator +} на шаблона на хеш-таблицата. Хеш-таблицата \texttt{c}, която се получава при \texttt{c = a + b}, да съдържа като ключове симетричната разлика на ключовете на \texttt{a} и \texttt{b}, със съответните им стойности от \texttt{a} и \texttt{b}. Хеш-функцията на \texttt{c} да е същата като на \texttt{b}.

	\emph{Симетрична разлика на множествата $A$ и $B$ наричаме множеството $C = A \Delta B = A \cup B - A \cap B$, съдържащо тези елементи на $A$, които не са елементи на $B$ и тези елементи на $B$, които не са елементи на $A$.}

	\item Да се дефинира метод 

	\texttt{void map (void (*f) (ValueType\&))} 

	на хеш-таблицата, който прилага функцията \texttt{f} над всички стойности в хеш-таблицата.

	\item Да се дефинира метод 

	\texttt{void mapKeys (KeyType (*f) (const KeyType\&))} 

	на хеш-таблицата, който замества всеки ключ \texttt{key} на хеш-таблицата с \texttt{f(key)}, като се запазва старата му стойност. 

	\emph{Упътване: Да се извърши съответното ре-хеширане на елемента и той да се премести на съответния нов индекс в таблицата.}


	\item Методът \texttt{begin} на хеш-таблицата да се допълни така, че да може да получава и предикат $p:KeyType \rightarrow bool$. Резултатният итератор да итерира само през тези ключове от таблицата, които удовлетворяват $p$.


	\item Да се дефинира \texttt{operator *} на шаблона на хеш-таблицата. Хеш-таблицата \texttt{c}, която се получава при \texttt{c = a * b}, да съдържа като множество от ключове сечението на множествата на ключове на \texttt{a} и \texttt{b}, като стойноста на всеки ключ е:

	\begin{itemize}
		\item Вектор с два елемента -- стойността на ключа от \texttt{a} и стойността на ключа от \texttt{b}, ако тези стойност \emph{не са вектори}.
		\item Конкатенацията на стойността на ключа от \texttt{a} и стойността на ключа от \texttt{b}, ако тези стойност \emph{\underline{са} вектори}.
	\end{itemize}

	Хеш-функцията на \texttt{c} да е като хеш-функцията на \texttt{b}.
	\\
	\\

	\begin{mdframed}[hidealllines=true,backgroundcolor=gray!20]
	\relscale{0.75}
	Пример: Операторът да удовлетворява следния тест:

\begin{lstlisting}
HashMap<string,double> m(5,stringhash1);
m["Kalin"] = 1.85; m["Ivan"] = 1.86;
HashMap<string,double> m1(3,stringhash1);
m1["Kalin"] = 2; m1["Petar"] = 2;
HashMap<string,vector<double>> mult = m * m1;
mult = mult * mult;

assert (mult.containsKey("Kalin"));
assert (!mult.containsKey("Ivan"));
assert (!mult.containsKey("Petar"));
assert (mult["Kalin"].size() == 4);		
\end{lstlisting}

	\end{mdframed}

	\begin{mdframed}[hidealllines=true,backgroundcolor=gray!20]
	\relscale{0.75}
	Решение. 

	Можем да реализираме глобален шаблон на оператор за умножение на хеш-таблици:

\begin{verbatim}
template <class KeyType, class ValueType>
HashMap<KeyType,vector<ValueType>> 
  operator * (const HashMap<KeyType,ValueType> &hm1,
              const HashMap<KeyType,ValueType> &hm2)
	
\end{verbatim}

	Както се вижда, аргументите на оператора са две хеш-таблици с тип на стойностите \texttt{ValueType}, а стойността на оператора е от типа \texttt{vector<ValueType>}. Един лесен начин да реализираме тялото на оператора е:
	\\


\begin{verbatim}
HashMap<KeyType,vector<ValueType>> 
  result(hm2.size(),hm2.getHashFunction());
for (const KeyType &key : hm1)
{
    //key е в сечението на ключовете
    if (hm2.containsKey (key))
    {
        result[key].push_back(hm1[key]);
        result[key].push_back(hm2[key]);
    }
}
return result;

\end{verbatim}

	Очевидно е, че ако типа \texttt{ValueType} се случи да е самият той \emph{вектор}, ефектът от горната реализация ще бъдат стойности, които са \emph{вектори от вектори}, а това не е търсеният резултат. Следователно нашата реализация трябва да ``знае'' кога \texttt{ValueType} е вектор и да вземе съответните мерки да конкатенира векторите, които са стойности на ключа \texttt{key} в двете таблици \texttt{hm1} и \texttt{hm2}.

	От друга страна, няма как по време на изпълнението на програмата да определим в тялото на оператора дали \texttt{ValueType} е вектор или не е. Дори да използваме някакъв \texttt{RTTI} трик и все пак да разберем в \emph{run time}, че \texttt{ValueType} е вектор, ще се наложи да приложим още трикове, с които да преобразуваме \texttt{ValueType} така, че да можем да използваме методите на \texttt{std::vector} за достъп до елементите и да извършим конкатенацията. Би се получило неелегантно и неразбираемо решение.

	За щастие, шаблоните в \texttt{C++} ни позволяват да направим частен случай на шаблона на оператора \texttt{*}, който да се предпочете от компилатора тогава, когато хеш-таблицата е със стойност вектори. Необходимо е единствено да добавим още един шаблон:

\begin{verbatim}
template <class KeyType, class ValueType>
HashMap<KeyType,vector<ValueType>> 
  operator * (const HashMap<KeyType,vector<ValueType>> &hm1,
              const HashMap<KeyType,vector<ValueType>> &hm2)
\end{verbatim}


	Забележете, че хеш-таблиците са дефинирани с тип на стойностите \texttt{vector<ValueType>}, което е частен случай на простото \texttt{ValueType} в предишния оператор \texttt{*}, но е достатъчно на компилатора да подбере именно този оператор при опит да умножим две хеш-таблици, чиито стойности са от тип вектор. Съответната реализация на оператора е:
	\\
	\\

\begin{verbatim}
HashMap<KeyType,vector<ValueType>> 
  result(hm2.size(),hm2.getHashFunction());
for (const KeyType &key : hm1)
{
    //key е в сечението на ключовете
    if (hm2.containsKey (key))
    {
        result[key] = append (hm1[key], hm2[key]);
    }
}
return result;

\end{verbatim}

    Тук \texttt{append} е помощна функция за конкатенация на вектори.




	\end{mdframed}




\end{enumerate}




\end{document}

