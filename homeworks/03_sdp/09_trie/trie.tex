\documentclass[12pt,a4paper]{article}
\usepackage[utf8]{inputenc}
\usepackage{amsmath}
\usepackage{amsfonts}
\usepackage{amssymb}
\usepackage{listings}
\usepackage{url}
\usepackage[bulgarian]{babel}
\usepackage{listings}
\usepackage[framemethod=tikz]{mdframed}
\usepackage{relsize}




\lstset{breaklines=true} 


\author{\textit{email: kalin@fmi.uni-sofia.bg}}
\title{\textsc{Задачи за задължителна самоподготовка} \\
по \\
Структури от данни и програмиране}



\begin{document}
\maketitle


\begin{enumerate}

	\item За клас \texttt{Trie} да се разработи приятелски клас \texttt{TrieUtilities}, който да реализира методи за:

	\begin{itemize}
		\item Намиране на височината на дървото.
		\item Намиране на дължината на най-дългия ключ, записан в дървото.
		\item Намиране на броя на записаните стойности в дървото.
		\item Намиране на броя на буквите на латинската азбука, които \emph{не учстват} в никой ключ на дървото.
	\end{itemize}

	\item Да се дефинира итератор за \texttt{Trie}, позволяващ обхождането на ключовете му в нарастващ ред относно лексикографската наредба.

	\item Да се разработи оператор за индексиране (\texttt{[]}) на \texttt{Trie}, който да работи за четене и писане на ключове.

	\item Да се реализира отпечатване на \texttt{Trie} в \texttt{dotty} формат.



\end{enumerate}




\end{document}

