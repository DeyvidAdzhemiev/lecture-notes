\documentclass[12pt,a4paper]{article}
\usepackage[utf8]{inputenc}
\usepackage{amsmath}
\usepackage{amsfonts}
\usepackage{amssymb}
\usepackage{listings}
\usepackage{url}
\usepackage[bulgarian]{babel}
\usepackage{listings}
\usepackage[framemethod=tikz]{mdframed}
\usepackage{relsize}




\lstset{breaklines=true}


\author{\textit{email: kalin@fmi.uni-sofia.bg}}
\title{\textsc{Задачи за задължителна самоподготовка} \\
по \\
Структури от данни и програмиране}



\begin{document}
\maketitle


``Дясна регулярна граматика'' наричаме грматика с правила от вида $A \rightarrow aB$ или $A \rightarrow a$, където $A$ и $B$ са нетерминални символи, а $a$ е терминален символ от азбуката на граматиката.



\begin{enumerate}


	\item В текстов файл са записани правилата $P$ на дясна регулярна граматика $A=<A..Z,a..z,P,A>$ по следния начин:
	\begin{verbatim}
	A := aB
	\end{verbatim}

	или

	\begin{verbatim}
	A := a
	\end{verbatim}

	Където А и B са големи латински букви, а $a$ е малка латинска буква. Да се построи краен автомат с език, вквивалентен на езика на граматиката от файла.

	\emph{Упътване: За всеки нетерминален символ постройте състояние с индекс, който е поредния номер на символа в английската азбука. Направете едно специално финално състояние. Всички преходи от вида \texttt{A := a} се представят чрез преход от състоянието, съответно на нетерминалния символ \texttt{A} към специалното крайно състояние.}

	\item Да се реализира печатане на автомата в \texttt{dotty} формат.

	\item Да се реализира функция, която проверява дали дадена дума се разпознава от атомата или не.


\end{enumerate}




\end{document}
