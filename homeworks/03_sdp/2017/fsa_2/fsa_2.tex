\documentclass[12pt,a4paper]{article}
\usepackage[utf8]{inputenc}
\usepackage{amsmath}
\usepackage{amsfonts}
\usepackage{amssymb}
\usepackage{listings}
\usepackage{url}
\usepackage[bulgarian]{babel}
\usepackage{listings}
\usepackage[framemethod=tikz]{mdframed}
\usepackage{relsize}




\lstset{breaklines=true}


\author{\textit{email: kalin@fmi.uni-sofia.bg}}
\title{\textsc{Задачи за задължителна самоподготовка} \\
по \\
Структури от данни и програмиране}



\begin{document}
\maketitle



\begin{enumerate}


	\item (решена) Към разработения на лекции клас \texttt{DFSA} да се добави итератор за сътояния (\texttt{StatesIterator}), чрез който да могат да бъдат обхождани състоянията на автомата, например по следния начин:

\begin{verbatim}
for (uint state : A)
{
   std::cout << "State label =  "
             << state
             << std::endl;
}
\end{verbatim}


\item (решена) Към структурата \texttt{state} на разработения на лекции клас \texttt{DFSA} да се добави итератор за символи (\texttt{SymbolsIterator}), чрез който да могат да бъдат обхождани всички изходящи преходи от състоянието, например по следния начин:

\begin{verbatim}
for (uint state : A)
{
   for (char symbol : A[state])
   {
       std::cout << "Transition="
                 << state
                 << ":"
                 << symbol
                 << "->"
                 << A[state][symbol]
                 << std::endl;
}
\end{verbatim}

\item (решена) С така създадените методи за достъп да се релизира печатане в \texttt{Dotty} формат.

\item (ренеша) Да се подобри класа \texttt{DFSA}, така че да могат да бъдат обхождани преходите на константен автомат.

\emph{Упътване: Двата итератора трябва да осигурявят константен достъп. По-особената част е за всеки от операторите за индексиране \texttt{[]} (на автомата и на състоянието) да се добави константна версия. Обърнете внимание на разликата между оператор \texttt{map::operator []} и метода \texttt{map::at(...)}. Случаят е най-интересен за константната версия на \texttt{state::operator[]}, за която не се налага proxy.}

\item (решена) При условие, че в автомата няма цикли, да се намери най-дългата дума в езика му.

\item (не е решена) Да се намери най-дългата крайна дума в езика на автомат, в който може да има цикли.

\item (не е решена) Да се намери броя на крайните думи в езика на автомат, в който може да има цикли.


\end{enumerate}




\end{document}
