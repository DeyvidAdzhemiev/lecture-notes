\documentclass[12pt,a4paper]{article}
\usepackage[utf8]{inputenc}
\usepackage{amsmath}
\usepackage{amsfonts}
\usepackage{amssymb}
\usepackage{listings}
\usepackage{url}
\usepackage[bulgarian]{babel}
\usepackage{listings}
\usepackage[framemethod=tikz]{mdframed}
\usepackage{relsize}




\lstset{breaklines=true}


\author{\textit{email: kalin@fmi.uni-sofia.bg}}
\title{\textsc{Задачи за задължителна самоподготовка} \\
по \\
Структури от данни и програмиране}



\begin{document}
\maketitle



\begin{enumerate}


	\item Даден е граф $G:<V=\{0..n-1\},E\subseteq V \times V \times \{a..z\}>$ с етикети на ребрата $\{a..z\}$, представен с матрица $std::vector<char> G[n][n]$. Каваме, че думата $w=w_1,..,w_k$ може ``да се прочете'' в графа, ако в него има път между произволни два върха, състоящ се от последователни ребра с етикети $w_1,..,w_k$. Да се дефинира функция, която намира по колко различни начина може да се прочете дадена дума в даден граф (т.е. колко различни пътя в графа отговарят на това условие).

	\item  Даден е граф $G:<V=\{0..n-1\},E \subseteq V \times V >$, представен с матрица $bool$ $G[n][n]$. Нека $v \in V$ е връх в графа, a $k \in N$. Да се построи и отпечата в dotty формат дърво с корен $v$ и височина най-много $k$, за което е изпълнено:
	\begin{itemize}
		\item Всяко ниво $0 \leq l < k$ съдържа всички върхове $u$ от графа, до които има път от $v$ с дължина $l+1$ (в брой върхове).
		\item За всяко $0 < l < k$, eлементът $u$ на ниво $l-1$ е родител в дървото на елемента $w$ на ниво $l$ тогава и само тогава, когато $(u,w) \in E$.
	\end{itemize}

	\item Даден е ориентиран ацикличен граф $G:<V=\{0..n-1\},E \subseteq V \times V >$, представен с матрица $bool$ $G[n][n]$. Да се построи топологично сортиран вектор $(v_1,..,v_n)$ от всички върхове на графа.

	\item Даден е лабиринт, представен с матрица $bool$ $L[n][n]$. Елементите $L[i][j]==true$ считаме за проходими, а елементите $L[i][j]==false$ - за непроходими.

	Да се построи матрицата на съседтсво $bool$ $G[n \times n][n \times n]$ на графа $G:<V,E>$, за който множеството $V$ съдържа представители на всички проходими елементи на $L$, а $(u,v) \in E$ тогава и само тогава, когато $u$ и $v$ са проходими и са съседи в $L$.

	Да се отпечата графа в dotty формат и да се намери броя на свързаните му компоненти.



\end{enumerate}




\end{document}
