\documentclass[12pt,a4paper]{article}
\usepackage[utf8]{inputenc}
\usepackage{amsmath}
\usepackage{amsfonts}
\usepackage{amssymb}
\usepackage{listings}
\usepackage{url}
\usepackage[bulgarian]{babel}
\usepackage{listings}
\usepackage{enumerate}
\usepackage{hyperref}
\usepackage[framemethod=tikz]{mdframed}
\usepackage{relsize}


\newcommand{\code}[1]{\texttt{#1}}

\lstset{breaklines=true}


\author{\textit{email: kalin@fmi.uni-sofia.bg}}
\title{\textsc{Задачи за задължителна самоподготовка} \\
по \\
Структури от данни и програмиране\\
\textit{``Реактивни'' безкрайни потоци}}



\begin{document}
\maketitle


ВНИМАНИЕ: Решението на следите задачи е публикувано към кода от лекции. Разглeдайте решенията, осмислете ги и ги пресъздайте самостоятелно (или предложете друго решение на задачата за реализация на операции над безкрайни потоци).

\begin{enumerate}

	\item  Към класа \texttt{StreamBase} от реализираната на лекции йерархия да се добави метод за печатане на първите \texttt{n} елемента от потока. Методът да връща остатъка от потока.


	Следният пример отпечатва първите 5 нечетни числа:
	\begin{verbatim}
			ints.filter(odd).print(2).print(3);
	\end{verbatim}

	\item Да се дефинира поток \texttt{RepeatStream}, състоящ се от безкрайно повторение на дадено число.

	Следният пример отпечатва 5 единици:
	\begin{verbatim}
RepeatStream ones(1);
ones.print(5);
	\end{verbatim}


	\item Да се дефинира клас \texttt{SumStream}, който позволява по дадени два потока \texttt{A} и \texttt{B} да се генерира нов поток, всеки от елементите на който е сумата (получена с оператора \texttt{+}) на двата съответни елемента на \texttt{A} и \texttt{B}. В клас \texttt{StreamBase} да се добави метод \texttt{sum}, с който да може всеки поток да се сумира с друг поток.

	Следният пример отпечатва първите 5 четни числа:
	\begin{verbatim}
			ints.sum(ints).print (5);
	\end{verbatim}

	Следният пример отпечатва 5 двойки:
	\begin{verbatim}
			ones.sum(ones).print (5);
	\end{verbatim}


	\item Да се дефинира клас \texttt{ZipStream}, който позволява по дадени два потока $A=(a_i)$ и $B=(b_i)$ да се генерира нов поток $C=(c_i)$, всеки от елементите на който е резултат от приложението на някаква функция $f:int \times int \rightarrow int$ над двата съответни елемента на \texttt{A} и \texttt{B}, т.е. $c_i=f(a_i,b_i)$. В клас \texttt{StreamBase} да се добави метод \texttt{zip}, с който да може всеки поток да се комбинира с друг поток.

	Следният пример отпечатва първите 5 точни квадрата:
	\begin{verbatim}
			ints.zip(ints,mult).print (5);
	\end{verbatim}
	 Където $mult(x,y)=x*y$

	 \item Да се дефинира клас \texttt{ZigZagStream}, който позволява по дадени два потока $A=(a_i)$ и $B=(b_i)$ да се генрира нов поток $C=(c_i)$ от алтернативно разменянищите се последователни елементи на $A$ и $B$, т.е. $c_0=a_0, c_1=b_0, c_2=a_1, c_3=b_1,...$. В клас \texttt{StreamBase} да се добави метод \texttt{zigzag}, с който да може всеки поток да се комбинира с друг поток.

	 Следният пример отпечатва първите 5 естествени числа, последвани от вторите им степени:
	 \begin{verbatim}
	 		ints.zigzag(ints.zip(ints,mult)).print (5);
	 \end{verbatim}

	 \item Да се дефинира клас \texttt{AxisStream}, чрез който по дадена фунцкия $f:int\rightarrow int$ и число $y_0$ да се генерира поток от елементите $(f^i(y_0))$, където $f^i(y_0)=f(f..f(y_0))$ е $i$-кратното приложение на функцията $f$ над аргумента $y_0$ за $i=0...$, т.е. потокът се състои от елементите $y_0, f(y_0),f(f(y_0)),...$.

	 Следният пример отпечатва първите 5 степени на числото 2:
	 \begin{verbatim}
Axis powers2 (sqr,2);
powers2.print (5);
	 \end{verbatim}
	 Където $sqr(x)=x*x$


	 \item Какво трябва да се промени в йерархията така, че да може да се конструират потоци с елементи, различни от $int$?

\end{enumerate}


	\vspace{20px}


\end{document}
