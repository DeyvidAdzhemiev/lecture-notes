\documentclass[12pt,a4paper]{article}
\usepackage[utf8]{inputenc}
\usepackage{amsmath}
\usepackage{amsfonts}
\usepackage{amssymb}
\usepackage{listings}
\usepackage{url}
\usepackage[bulgarian]{babel}
\usepackage{listings}
\usepackage{enumerate}
\usepackage{hyperref}
\usepackage[framemethod=tikz]{mdframed}
\usepackage{relsize}


\newcommand{\code}[1]{\texttt{#1}}

\lstset{breaklines=true}


\author{\textit{email: kalin@fmi.uni-sofia.bg}}
\title{\textsc{Задачи за задължителна самоподготовка} \\
по \\
Структури от данни и програмиране\\
\textit{Линеен едносвързан списък}}



\begin{document}
\maketitle


\begin{enumerate}

	\item  Да се решат следните задачи от ЗЗС 1 чрез използване на итератор за списък: задачи 1, 6, 8, 10, 11. Т.е., да се дефинират функции, които получават итератор и извършват съответните проверки.
	\item Да се тестват същите функции с итератор на масив.

	\item \code{IteratorBase} и производните му да се обогатят с методи \code{getPrevious()} и съответно \code{hasPrevious()}, които навигират итераторите към предишния елемент на структурата от данни.

	\item Да се дефинира \code{StringIterator}, обхождащ символите в символен низ. Всички предишни решения да се тестват и с този итератор.

	\item Да се дефинират итератори, които обхождат:
		\begin{itemize}
			\item Редицата $\{2k\}_{k \in N}$ (редицата на четните естествени числа)
			\item Редицата $0,1,1,2,3,5,8,...$ от числата на Фибоначи: $a_0 = 0$, $a_1 = 1$, $a_i=a_{i-1}+a_{i-2}$, за $i>1$.
			\item Елементите на множеството $\{(i,j)|i,j \in N\}$, т.е. множеството от всички двойки естествени числа.

			\emph{Внимание:} Итераторът да се построи така, че \emph{всяка} двойка естествени числа да е достижима за краен брой стъпки. Например, това няма да е вярно, ако итераторът генерира последователно следните двойки: $(0,0), (0,1), (0,2), (0,3), ...$. В този случай, двойката $(1,0)$ никога няма да бъде достигната.
		\end{itemize}

		\item Да се дефинира итератор, който обхожда последователните стойности на дадена функция $f:N \rightarrow E$, където $E$ е произволен тип. Т.е., обхожда се редицата $\{f(k)\}_{k \in N	}$



\end{enumerate}


	\vspace{20px}


\end{document}
