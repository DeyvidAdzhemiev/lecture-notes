\documentclass[12pt,a4paper]{article}
\usepackage[utf8]{inputenc}
\usepackage{amsmath}
\usepackage{amsfonts}
\usepackage{amssymb}
\usepackage{listings}
\usepackage{url}
\usepackage[bulgarian]{babel}
\usepackage{listings}
\usepackage{enumerate}
\usepackage{hyperref}
\usepackage[framemethod=tikz]{mdframed}
\usepackage{relsize}


\newcommand{\code}[1]{\texttt{#1}}

\lstset{breaklines=true}


\author{\textit{email: kalin@fmi.uni-sofia.bg}}
\title{\textsc{Задачи за задължителна самоподготовка} \\
по \\
Структури от данни и програмиране\\
\textit{Линеен едносвързан списък}}



\begin{document}
\maketitle

Следните задачи да се решат като упражнение за директно боравене с указателите и двойните кутии на линеен едносвързан списък.

\begin{enumerate}

	\item  Функция \code{int count(box* l,int x)}, която преброява колко пъти елементът \code{x} се среща в списъка с първи елемент \code{l}.
	\item  Фунцкция \code{box* range (int x, int y)} която създава и връща първия елемент на списък с елементи $x, x+1, ..., y$, при положение, че $x \leq y$.
	\item  Функция \code{removeAll (box*\& l,int x)}, която изтрива всички срещания на елемента \code{x} от списъка \code{l}.
	\item  Функция \code{void append(box*\& l1, box* l2)}, която добавя към края на списъка $l_1$ всички елементи на списъка $l_2$.
	\item  Функция \code{box* concat(box *l1, box* l2)}, който съединява два списъка в нов, трети списък. Т.е. \code{concat($l_1,l_2$)} създава и връща нов списък от елементите на \code{$l_1$}, следвани от елементите на \code{$l_2$}.
	\item  Функция \code{map}, която прилага едноаргументна функция $f:int \rightarrow int$ към всеки от елементите на списък.
	\item  Функция \code{reverse}, която обръща реда на елементите на списък. Например, списъкът с елементи $1,2,3$ ще се преобразува до списъка с елементи $3,2,1$.
	\item Да се напише функция \code{bool duplicates (box *l)}, която проверява дали в списъка $l$ има дублиращи се елементи.
	\item Да се напише функция \code{void removeduplicates (box *\&l)}, която изтрива всички дублиращи се елементи от списъка $l$.
	\item Да се напише фунцкия \code{bool issorted (box *l)}, която проверява дали даден списък е подреден в нарастващ или в намаляващ ред.
	\item Да се напише фунцкия \code{bool palindrom (box *l)}, която проверява дали редицата от елементи на даден списък обрзува палиндром (т.е. дали се чете еднакво както отляво надясно така и отдяно наляво).


\end{enumerate}


	\vspace{20px}


\end{document}
