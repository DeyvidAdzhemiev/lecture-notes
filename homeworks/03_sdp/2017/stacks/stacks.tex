\documentclass[12pt,a4paper]{article}
\usepackage[utf8]{inputenc}
\usepackage{amsmath}
\usepackage{amsfonts}
\usepackage{amssymb}
\usepackage{listings}
\usepackage{url}
\usepackage[bulgarian]{babel}
\usepackage{listings}
\usepackage{enumerate}
\usepackage{hyperref}
\usepackage{relsize}
\usepackage{graphicx}


\lstset{breaklines=true}


\author{\textit{email: kalin@fmi.uni-sofia.bg}}
\title{\textsc{Задачи за задължителна самоподготовка} \\
по \\
Структури от данни и програмиране\\
\textit{Привеждане на рекурсивни решения към решения със стек}}




\begin{document}
\maketitle

 \underline{Упътване}:Решете задачите с рекурсия и след това преобразувайте решението в решение със стек.



\begin{enumerate}

	\item (*)Да се дефинира функция за намиране на стойността на полинома на Ермит $Hn(x)$ (x е реална променлива, а n неотрицателна цяла променлива), дефиниран по следния начин:

	$H_0(x)=1$

	$H_1(x)=2x$

	$H_n(x)=2xH_{n-1}(x)+2(n-1)H_{n-2}(x), n>1$,

	за дадени $n$ и $x$ \underline{с използване на стек}.


	\item Нека е дадена абстрактна шахматна дъска с размери $n \times n$, $4 \le n \le 8$ и число $k$, $0 \le k \le n$. Казваме, че разположени на дъската  $k$ коня образуват ``валидна конфигурация'', ако никоя фигура не е поставена на поле, което се ``бие'' от друга фигура според съответните шахматни правила.

	Да се дефинира клас \texttt{KnightConfig}, представящ ``конфигуратор'' на шахматни коне. Конструкторът на класа инициализира конфигуратора с числата $n$ и $k$. Класът позволява ``обхождането'' една по една на всички валидни конфигурации за дадените параметри, по подобие на \texttt{forward} итератор на структура от данни. Класът да притежава следните методи:

	\begin{itemize}
		\item \texttt{void KnightConfig::printCurrentConfig()}: Отпечатва текущо намерената конфигурация.
			Пример за отпечатана конфигурация с $n=5, k=2$:
			\begin{verbatim}
			_ _ _ _ _
			_ _ H _ _
			_ _ _ _ _
			_ _ _ _ H
			_ _ _ _ _

			\end{verbatim}
		\item \texttt{void KnightConfig::findNextConfig()}: Намира следваща конфигурация.
		\item \texttt{bool KnightConfig::noMoreConfigs()}: Показва дали всички възможни конфигурации са вече изчерпани.
	\end{itemize}



\end{enumerate}






\end{document}
