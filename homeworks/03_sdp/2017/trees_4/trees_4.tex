\documentclass[12pt,a4paper]{article}
\usepackage[utf8]{inputenc}
\usepackage{amsmath}
\usepackage{amsfonts}
\usepackage{amssymb}
\usepackage{listings}
\usepackage{url}
\usepackage[bulgarian]{babel}
\usepackage{listings}
\usepackage{enumerate}
\usepackage{hyperref}
\usepackage{relsize}
\usepackage{graphicx}


\lstset{breaklines=true}


\author{\textit{email: kalin@fmi.uni-sofia.bg}}
\title{\textsc{Задачи за задължителна самоподготовка} \\
по \\
Структури от данни и програмиране\\
\textit{Двоични дървета, стек, итератори}}




\begin{document}
\maketitle



\begin{enumerate}

	\item Разработеният на лекции итератор на шаблон \texttt{BTree<T>} да бъде видоизменен така, че са се позволи конфигуриране на типа на обхождането при създаване итератора. Да се реализират следните обхождания:

	\begin{enumerate}
		\item КДЛ и ЛДК.
    \item Обхождане в широчина.
		\item Обхождане само листата на дървото, от ляво надясно.
		\item Обхождане само на тези елементи на дървото (при коя да е от горните последователности), за които е удовлетворен даден предикат \texttt{bool pred (const T\&)}.
  \end{enumerate}

    \item Да се решат отново следните задачи чрез използване на итератор:

    \begin{enumerate}
      \item Проверка за изпълнена двоична наредба в дървото.
      \item Нека е дадено двоично дърво с елементи от тип \texttt{char}, за което е изпълнено, че:

      	\begin{itemize}
      		\item Дървото е непразно
      		\item Всеки от възлите му има точно 2 или 0 наследника
      		\item Елементите с 2 наследника съдържат един от символите $+$, $-$, $*$ и $/$
      		\item Елементите с 0 наследника съдържат цифра
      	\end{itemize}

      	Да се дефинира функция, която връща стойността на аритметичния израз, съответен на дървото.


	\end{enumerate}





\end{enumerate}






\end{document}
