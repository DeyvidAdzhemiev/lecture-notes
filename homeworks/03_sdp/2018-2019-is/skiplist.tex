\documentclass[12pt,a4paper]{article}
\usepackage[utf8]{inputenc}
\usepackage{amsmath}
\usepackage{amsfonts}
\usepackage{amssymb}
\usepackage{listings}
\usepackage{url}
\usepackage[bulgarian]{babel}
\usepackage{listings}
\usepackage{enumerate}
\usepackage{hyperref}
\usepackage[framemethod=tikz]{mdframed}
\usepackage{relsize}
\usepackage{enumitem}


\newcommand{\code}[1]{\texttt{#1}}

\lstset{breaklines=true}


\author{\textit{email: kalin@fmi.uni-sofia.bg}}
\title{\textsc{Задачи за задължителна самоподготовка} \\
по \\
Структури от данни и програмиране\\
\textit{Контейнери и итератори}}



\begin{document}
\maketitle


\begin{enumerate}

	\item  Да се реализира итераторът на \code{SkipList} така, че да се възползва от ``бързите връзки'' в списъка. \textit {Упътване: при обхождането извършвайте ``прескачане'' в случаите, в които има бърза връзка и в които няма да отидете твърде далеч напред в списъка. Класът на итератора трябва да се промени, за да позволява конструиране на итератор към конкретен елемент на списъка}.
	\item  Да се извърши времево измерване на проблема за търсене на елемент в подреден \code{SkipList} и да се изобрази чрез графика.
	\item Да се реализира копирането на \code{SkipList}.

\end{enumerate}


	\vspace{20px}

	\textit {Упътване: Инсталирайте библиотеката \code{graphviz} на компютъра си. Командата за създаване на PDF по Dotty изхода на програмата е:}

	\begin{verbatim}
	dot -Tpdf output.dot -o output.pdf
	\end{verbatim}

\end{document}
