\documentclass{beamer}
\usepackage{relsize}
\usepackage{color}

\usepackage{listings}
\usetheme{CambridgeUS}
%\usepackage{beamerthemesplit} % new 
\usepackage{enumitem}
\usepackage{amsmath}                    % See geometry.pdf to learn the layout options. 
\usepackage{amsthm}                   % See geometry.pdf to learn the layout options. There 
\usepackage{amssymb}                    % See geometry.pdf to learn the layout options. 
\usepackage[utf8]{inputenc} 
\usepackage{graphicx}
\usepackage[english,bulgarian]{babel}

\lstset{language=C++,
                basicstyle=\ttfamily,
                keywordstyle=\color{blue}\ttfamily,
                stringstyle=\color{red}\ttfamily,
                commentstyle=\color{green}\ttfamily,
                morecomment=[l][\color{magenta}]{\#}
}

\newtheorem{mydef}{Дефиниция}[section]
\newtheorem{lem}{Лема}[section]
\newtheorem{thm}{Твърдение}[section]

\DeclareMathOperator{\restrict}{\upharpoonright}

\setitemize{label=\usebeamerfont*{itemize item}%
  \usebeamercolor[fg]{itemize item}
  \usebeamertemplate{itemize item}}

\setbeamercovered{transparent}



\begin{document}
\title[Увод в програмирането]{Функции II} 
\author{Калин Георгиев} 
\frame{\titlepage} 


\section{Функции} 


\begin{frame}
\centerline{Отново функции vs. подпрограми}
\end{frame}



\begin{frame}[fragile]
\frametitle{Математически изображения}

\begin{itemize}
  \item Приличат на ``Формули'': $S=vt + \frac{1}{2}at^2$
\pause
  \item Съотвена фунцкия: $S:\mathcal{R}\times\mathcal{R}\times\mathcal{R}\rightarrow\mathcal{R}$, $S(v,t,a)=vt + \frac{1}{2}at^2$
\pause
  \item Могат да учстават в изрази: $S(10,60,0) + S(10,60,20)$
\pause
  \item Не ``правят'' нищо 
\end{itemize}
\pause

\begin{flushleft}
\relscale{0.8}
\begin{lstlisting}
double displacement (double speed, 
                     double time, 
                    double acceleration)
{
  double S = speed*time + acceleration*time*time/2;
  return S;
}
void main ()
{
  //....
  cout << displacement (10,60,0) + displacement (10,60,20);
}

\end{lstlisting}
  
\end{flushleft}
\end{frame}


\begin{frame}[fragile]
\frametitle{Подпрограми. Процедури}

\begin{itemize}
  \item ``Правят'' нещо: Страничен ефект
  \item ``Стойността'' им няма значение
\end{itemize}



\begin{flushleft}
\relscale{0.7}
\begin{lstlisting}
void pritnSequence (long start, long end, long step)
{
  for (long element = start; element <= end; element += step)
  {
    cout << element;
    if (element < end)
      cout << ",";
  }  
  cout << endl;
}

void main ()
{
  pritnSequence (1,10,1);
  pritnSequence (10,30,2);
  pritnSequence (30,80,5);
}
\end{lstlisting}

\end{flushleft}

\end{frame}


\begin{frame}
\centerline{Процес на изпълнение. Програмен стек}
\end{frame}


\begin{frame}[fragile]
\frametitle{Формални vs. Фактически параметри}



\begin{columns}[t]
  \begin{column}{0.75\textwidth}

\relscale{0.72}
\begin{lstlisting}
void pritnSequence (long start, long end, long step)
{
  for (long element = start; 
       element <= end;  
       element += step)
  {
    cout << element;
    if (element < end)
      cout << ",";
  }  
  cout << endl;
}

void main ()
{ long step = 1;
  pritnSequence (1,10,step); //(1)
  step = 2;
  pritnSequence (10,30,step); //(2)
  step = 15;
  pritnSequence (30,80,5); //(3)
}
\end{lstlisting}


  \end{column}
  \begin{column}{0.25\textwidth}

    \alert{(1)}

    \begin{tabular}{|c|c|}
    step & 1 \\\hline

    \end{tabular}
    \pause
    \begin{tabular}{|c|c|}
    start & 1 \\\hline
    end & 10 \\\hline
    step & 1 \\\hline

    \end{tabular}
  \end{column}
\end{columns}

\end{frame}




\begin{frame}[fragile]
\frametitle{Формални vs. Фактически параметри}



\begin{columns}[t]
  \begin{column}{0.75\textwidth}

\relscale{0.72}
\begin{lstlisting}
void pritnSequence (long start, long end, long step)
{
  for (long element = start; 
       element <= end;  
       element += step)
  {
    cout << element;
    if (element < end)
      cout << ",";
  }  
  cout << endl;
}

void main ()
{ long step = 1;
  pritnSequence (1,10,step); //(1)
  step = 2;
  pritnSequence (10,30,step); //(2)
  step = 15;
  pritnSequence (30,80,5); //(3)
}
\end{lstlisting}


  \end{column}
  \begin{column}{0.25\textwidth}

    \alert{(2)}

    \begin{tabular}{|c|c|}
    step & 2 \\\hline

    \end{tabular}
    \pause
    \begin{tabular}{|c|c|}
    start & 10 \\\hline
    end & 30 \\\hline
    step & 2 \\\hline

    \end{tabular}
  \end{column}
\end{columns}

\end{frame}






\begin{frame}[fragile]
\frametitle{Формални vs. Фактически параметри}



\begin{columns}[t]
  \begin{column}{0.75\textwidth}

\relscale{0.72}
\begin{lstlisting}
void pritnSequence (long start, long end, long step)
{
  for (long element = start; 
       element <= end;  
       element += step)
  {
    cout << element;
    if (element < end)
      cout << ",";
  }  
  cout << endl;
}

void main ()
{ long step = 1;
  pritnSequence (1,10,step); //(1)
  step = 2;
  pritnSequence (10,30,step); //(2)
  step = 15;
  pritnSequence (30,80,5); //(3)
}
\end{lstlisting}


  \end{column}
  \begin{column}{0.25\textwidth}

    \alert{(3)}

    \begin{tabular}{|c|c|}
    step & 15 \\\hline

    \end{tabular}
    \pause
    \begin{tabular}{|c|c|}
    start & 30 \\\hline
    end & 80 \\\hline
    step & 5 \\\hline

    \end{tabular}
  \end{column}
\end{columns}

\end{frame}

\begin{frame}[fragile]
\frametitle{Взаимни извиквания}

\begin{columns}[t]
  \begin{column}{0.8\textwidth}
\relscale{0.72}
\begin{lstlisting}
void g (long x)
{cout << x;}

void f (long x)
{
  x = x + 10;
  g (x);
}

void main ()
{
  long x = 0;
  f (x);
  cout << x;
}

\end{lstlisting}
  \end{column}
  \begin{column}{0.2\textwidth}

    \begin{tabular}{c|c|c|}
    main: & x & 0 \\\hline

    \end{tabular}
    \pause
    \begin{tabular}{c|c|c|}
    f: & x & 0 \\\hline

    \end{tabular}
    \pause
    \begin{tabular}{c|c|c|}
    f: & x & 10 \\\hline

    \end{tabular}
    \pause
    \begin{tabular}{c|c|c|}
    g: & x & 10 \\\hline

    \end{tabular}


  \end{column}
\end{columns}


\end{frame}

\begin{frame}[fragile]
\frametitle{Взаимни извиквания}

\begin{columns}[t]
  \begin{column}{0.8\textwidth}
\relscale{0.72}
\begin{lstlisting}
void g (long x)
{cout << x;}

void f (long x)
{
  x = x + 10;
  g (x);
}

void main ()
{
  long x = 0;
  f (x);
  cout << x;
}

\end{lstlisting}
  \end{column}
  \begin{column}{0.2\textwidth}

    \begin{tabular}{c|c|c|}
    main: & x & 0 \\\hline

    \end{tabular}



  \end{column}
\end{columns}


\end{frame}


\begin{frame}[fragile]
\frametitle{Самоизвиквания}



\begin{columns}[t]
  \begin{column}{0.8\textwidth}
\relscale{0.72}
\begin{lstlisting}
void printSequence (long N)
{
  if (N > 0)
  {
    printSequence (N-1);
  }
  cout << N << " ";
}
void main ()
{
  printSequence(4);
}
\end{lstlisting}
  \end{column}
  \begin{column}{0.2\textwidth}

\pause
    \begin{tabular}{|c|c|}
    N & 4 \\\hline
    \end{tabular}
    \pause
    \begin{tabular}{|c|c|}
    N & 3 \\\hline
    \end{tabular}
    \pause
    \begin{tabular}{|c|c|}
    N & 2 \\\hline
    \end{tabular}
    \pause
    \begin{tabular}{|c|c|}
    N & 1 \\\hline
    \end{tabular}
    \pause
    \begin{tabular}{|c|c|}
    N & 0 \\\hline
    \end{tabular}


  \end{column}
\end{columns}

\end{frame}


\begin{frame}[fragile]
\frametitle{Размяна}



\relscale{0.72}
\begin{lstlisting}
void printSequence (long N)
{
  cout << N << " ";
  if (N > 0)
  {
    printSequence (N-1);
  }
}
void main ()
{
  printSequence(4);
}
\end{lstlisting}
\end{frame}


\begin{frame}[fragile]
\centerline{Пример:}
\begin{itemize}
  \item Въвеждане на число във фиксиран интервал
\end{itemize}

\begin{flushleft}
\relscale{0.7}
\begin{lstlisting}
void main ()
{
  cout << enterNumber (0,100) / enterNumber (1,100);
}
\end{lstlisting}
  
\end{flushleft}

\begin{itemize}
  \item Отпечатване на цифри
\end{itemize}


\end{frame}



\begin{frame}
\centerline{Благодаря за вниманието!}
\end{frame}



\end{document}




\begin{columns}[t]
  \begin{column}{0.2\textwidth}

  \end{column}
  \begin{column}{0.8\textwidth}

  \end{column}
\end{columns}


