\documentclass{beamer}
\usepackage{relsize}
\usepackage{color}

\usepackage{listings}
\usetheme{CambridgeUS}
%\usepackage{beamerthemesplit} % new 
\usepackage{enumitem}
\usepackage{amsmath}                    % See geometry.pdf to learn the layout options. 
\usepackage{amsthm}                   % See geometry.pdf to learn the layout options. There 
\usepackage{amssymb}                    % See geometry.pdf to learn the layout options. 
\usepackage[utf8]{inputenc} 
\usepackage{graphicx}
\usepackage[english,bulgarian]{babel}
\usepackage[framemethod=tikz]{mdframed}
\usepackage{caption}
\usepackage{tikz}
\usepackage{forest}
\usetikzlibrary{shapes,arrows,positioning,calc,chains}

\lstset{language=C++,
                basicstyle=\ttfamily,
                keywordstyle=\color{blue}\ttfamily,
                stringstyle=\color{red}\ttfamily,
                commentstyle=\color{green}\ttfamily,
                morecomment=[l][\color{magenta}]{\#}
}

\newcommand{\code}[1]{\texttt{#1}}

\newtheorem{mydef}{Дефиниция}[section]
\newtheorem{lem}{Лема}[section]
\newtheorem{thm}{Твърдение}[section]

\DeclareMathOperator{\restrict}{\upharpoonright}

\setitemize{label=\usebeamerfont*{itemize item}%
  \usebeamercolor[fg]{itemize item}
  \usebeamertemplate{itemize item}}

\setbeamercovered{transparent}

\captionsetup{font=footnotesize}

\lstset{breaklines=true}
\tikzset{
block/.style = {draw, fill=white, rectangle,align = center},
entry/.style = {draw, fill=black, circle, radius=3em},
condition/.style = {draw, fill=white, diamond, align = center,node distance=3cm},
fork/.style = {draw, fill=black, circle,inner sep=1pt},
lnode/.style={rectangle split, rectangle split parts=3,draw, rectangle split horizontal},
treenode/.style = {align=center, inner sep=0pt, text centered, circle, font=\sffamily\bfseries, draw=black, fill=white, text width=1.5em}
}


\begin{document}
\title[Увод в програмирането]{Машини с неограничени регистри} 
\author{Калин Георгиев} 
\frame{\titlepage} 

\section{Машини с неограничени регистри} 


\begin{frame}
\centerline{Машини с неограничени регистри}
\end{frame}


\begin{frame}[fragile]
\frametitle{Машини с неограничени регистри}
\begin{figure}
  \centering
    \begin{tikzpicture}
        \foreach \x in {0,0.5,...,5} {
            \pgfmathtruncatemacro{\label}{2*\x}
            \node (l\x) [left] at (\x,1) {\tiny \label};
            \node (n\x)[draw, minimum width=0.5cm] at (\x, 0.5) {0};
        }
        \node [minimum width=0.5cm](next) at (6,0.5) {... ... ...};
    \end{tikzpicture}
  \caption{Памет}
  \label{fig:mnr}
\end{figure}

\relscale{0.75}
Инструкции:
\begin{itemize}
    \item \texttt{ZERO n}: Записва стойността 0 в клетката с адрес $n$
    \item \texttt{INC n}: Увеличава с единица стойността, записана в клетката с адрес $n$
    \item \texttt{MOVE x y}: Присвоява на клетката с адрес $y$ стойността на клетката с адрес $x$
    \item \texttt{JUMP x}: Изпълнението на програмата ``прескача'' и продължава от инструкцията с пореден номер $x$. Ако програмата има по-малко от $x+1$ инструкции, изпълнението ѝ се прекратява
    \item \texttt{JUMP x y z}: Ако съдържанията на клетките  $x$ и $y$ съвпадат, изпълнението на програмата ``прескача'' и продължава от инструкцията с пореден номер $z$. В противен случай, програмата продължава със следващата инструкция. Ако програмата има по-малко от $z+1$ инструкции, изпълнението ѝ се прекратява
\end{itemize}


\end{frame}


\begin{frame}[fragile]
\frametitle{Програма}
\begin{figure}
    \centering
    \begin{tikzpicture}
        \foreach \x in {1,1.5,...,5} {
            \pgfmathtruncatemacro{\label}{2*\x}
            \node (l\x) [left] at (\x,1) {\tiny \label};
            \node (n\x)[draw, minimum width=0.5cm] at (\x, 0.5) {0};
        }
        \node [minimum width=0.5cm](next) at (6,0.5) {... ... ...};
        \node (l0) [left] at (0,1) {\tiny 0};
        \node (n0)[draw, red, minimum width=0.5cm] at (0, 0.5) {1};
        \node (l0.5) [left] at (0.5,1) {\tiny 1};
        \node (n0.5)[draw, red, minimum width=0.5cm] at (0.5, 0.5) {2};

    \end{tikzpicture}
    \caption{Памет}
\end{figure}

Програма:
\begin{verbatim}
0: INC 0
1: INC 1
2: INC 1
\end{verbatim}

\end{frame}


\begin{frame}[fragile]
    \frametitle{Умножение по 2}
    \relscale{0.7}

\begin{columns}[t]
    \begin{column}{0.75\textwidth}
        \begin{figure}
            \centering
            \begin{tikzpicture}
                \foreach \x in {0,0.5,...,5} {
                    \pgfmathtruncatemacro{\label}{2*\x}
                    \node (l\x) [left] at (\x,5.5 ) {\tiny \label};
                }    
                \foreach \iter in {1,...,6}{
                    \pgfmathsetmacro{\y}{.55*(10-\iter)} 
                    \foreach \x in {1.5,2,...,5} {
                        \node (n\x)[draw, minimum width=0.5cm] at (\x, \y) {0};
                    }   
                    \node (n0) [draw, minimum width=0.5cm] at (0, \y) {5};
                    \pgfmathtruncatemacro{\itm}{\iter-1}
                    \node (n1) [draw, minimum width=0.5cm] at (0.5, \y) {\itm};
                    \pgfmathtruncatemacro{\mtwo}{2*\itm}
                    \node (n2) [draw, minimum width=0.5cm] at (1.0, \y) {\mtwo};
                }
                    
            \end{tikzpicture}
            \caption{Памет}
        \end{figure}
        
    \end{column}
    \begin{column}{0.25\textwidth}
        \relscale{0.5}
        \begin{tikzpicture}[auto, node distance=1.5cm,>=latex']
            \node [entry, name=start](start){};
            \node [fork,name=test1fork,below of = start,node distance = 0.75cm]{};
            \node [condition,name=test1, below of = test1fork,node distance = 1.5cm] (test1) {\code{0:JUMP 0 1 5}};
            \node [block,name=inc,right of = test1, node distance = 2cm,align = left] (inc) {\code{1:INC 1}\\\code{2:INC 2}\\\code{3:JUMP 2}\\\code{4:JUMP 0}};
            \node [entry, name=end, below of = test1, node distance = 1.5cm](end){};
            \draw [->] (start) -- (test1fork);
            \draw [->] (test1fork) -- (test1);
            \draw [->] (test1) -- node{} (inc);
            \draw [->] (inc) |- (test1fork);
            \draw [->] (test1) -- node []{}(end);
            \end{tikzpicture}


    \end{column}
    \end{columns}
      

    
    Програма:
    \begin{verbatim}
        0: JUMP 0 1 5
        1: INC 1
        2: INC 2
        3: INC 2
        4: JUMP 0    
    \end{verbatim}
    
    \end{frame}
    


\begin{frame}
    \centerline{Благодаря за вниманието!}
\end{frame}


\end{document}



