\documentclass{beamer}
\usepackage{relsize}
\usepackage{color}

\usepackage{listings}
\usetheme{CambridgeUS}
%\usepackage{beamerthemesplit} % new 
\usepackage{enumitem}
\usepackage{amsmath}                    % See geometry.pdf to learn the layout options. 
\usepackage{amsthm}                   % See geometry.pdf to learn the layout options. There 
\usepackage{amssymb}                    % See geometry.pdf to learn the layout options. 
\usepackage[utf8]{inputenc} 
\usepackage{graphicx}
\usepackage[english,bulgarian]{babel}

\lstset{language=C++,
                basicstyle=\ttfamily,
                keywordstyle=\color{blue}\ttfamily,
                stringstyle=\color{red}\ttfamily,
                commentstyle=\color{green}\ttfamily,
                morecomment=[l][\color{magenta}]{\#}
}

\newtheorem{mydef}{Дефиниция}[section]
\newtheorem{lem}{Лема}[section]
\newtheorem{thm}{Твърдение}[section]

\DeclareMathOperator{\restrict}{\upharpoonright}

\setitemize{label=\usebeamerfont*{itemize item}%
  \usebeamercolor[fg]{itemize item}
  \usebeamertemplate{itemize item}}

\setbeamercovered{transparent}



\begin{document}
\title[Обектно ориентирано програмиране]{Типове II: Шаблони на функции. Указатели към функции} 
\author{Калин Георгиев} 
\frame{\titlepage} 

\section{Шаблони} 


\begin{frame}
\centerline{Шаблони (на функции)}
\end{frame}


\begin{frame}[fragile]
\frametitle{Еднообразни функции за различни типове}



\begin{columns}[t]
  \begin{column}{0.5\textwidth}

\begin{flushleft}
\relscale{0.75}
\begin{lstlisting}
int findIndexMax 
   (int arr[], int arrSize)
{
  int indexMax = 0;
  for (int i = 1; i < arrSize; i++)
    if (arr[indexMax] < arr[i])
      indexMax = i;

  return indexMax;
}
int findIndexMax 
   (char arr[], int arrSize)
{
  int indexMax = 0;
  for (int i = 1; i < arrSize; i++)
    if (arr[indexMax] < arr[i])
      indexMax = i;

  return indexMax;
}
\end{lstlisting}  
\end{flushleft}

  \end{column}
  \begin{column}{0.5\textwidth}
\begin{flushleft}
\relscale{0.75}
\begin{lstlisting}
int findIndexMax 
   (double arr[], int arrSize)
{
  int indexMax = 0;
  for (int i = 1; i < arrSize; i++)
    if (arr[indexMax] < arr[i])
      indexMax = i;

  return indexMax;
}

  ...

\end{lstlisting}  
\end{flushleft}

  \end{column}
\end{columns}


\end{frame}


\begin{frame}[fragile]
\frametitle{Създаване на ``Шаблон на функция''}


\begin{flushleft}
\relscale{0.7}
\begin{lstlisting}
template <typename T>
int findIndexMax  (T arr[], int arrSize)
{
  int indexMax = 0;
  for (int i = 1; i < arrSize; i++)
    if (arr[indexMax] < arr[i])
      indexMax = i;

  return indexMax;
}

\end{lstlisting}  
\end{flushleft}

\end{frame}


\begin{frame}[fragile]
\frametitle{Използване на шаблона на функция}

\begin{columns}[t]
  \begin{column}{0.6\textwidth}

\begin{flushleft}
\relscale{0.7}
\begin{lstlisting}
int main ()
{
  int arri[] = {1,5,6,7};
  cout << findIndexMax<int> (arri,4);

  double arrd[] = {2.1,17.5,6.0};
  cout << findIndexMax<double> (arrd,3);

  char arrc[] = "Hello";
  cout << findIndexMax<char> (arrc,5);

  char* arrstr[] = {"Hello", "World", "!"};
  cout << findIndexMax<char*> (arrstr,3); //!!!
}
\end{lstlisting}  
\end{flushleft}

  \end{column}
  \begin{column}{0.4\textwidth}

\begin{flushleft}
\relscale{0.1}
\begin{lstlisting}
template <typename T>
int findIndexMax  
   (T arr[], int arrSize)
{
  int indexMax = 0;
  for (int i = 1; i < arrSize; i++)
    if (arr[indexMax] < arr[i])
      indexMax = i;

  return indexMax;
}

\end{lstlisting}  
\end{flushleft}
  \end{column}
\end{columns}

\begin{itemize}
  \item Конретния тип трябва да е съвместим с всички операции в шаблона (в горния пример - $<$)!
\end{itemize}

\end{frame}

\begin{frame}[fragile]
\frametitle{Още един пример: печатане на ``всякакви'' масиви}


\begin{flushleft}
\relscale{0.7}
\begin{lstlisting}
template <typename T>
void printArray  (T arr[], int arrSize)
{
  cout << "{"
  for (int i = 0; i < arrSize-1; i++)
    cout << arr[i] << ",";

  if (arrSize > 0) //no comma
    cout << arr[arrSize-1];

  cout << "}";
}
\end{lstlisting}  
\end{flushleft}

\end{frame}

\section{Указатели} 

\begin{frame}
\centerline{Още по-дълбока параметризация: функции като параметри}
\end{frame}

\begin{frame}[fragile]
\frametitle{Пример за еднотипни фунцкии}

\begin{flushleft}
\relscale{0.7}
\begin{lstlisting}
int findIndexMax  (int arr[], int arrSize)
{
  int indexMax = 0;
  for (int i = 1; i < arrSize; i++)
    if (arr[indexMax] < arr[i])
      indexMax = i;

  return indexMax;
}

int findIndexMin  (int arr[], int arrSize)
{
  int indexMax = 0;
  for (int i = 1; i < arrSize; i++)
    if (arr[indexMax] > arr[i])
      indexMax = i;

  return indexMax;
}

\end{lstlisting}  
\end{flushleft}

\end{frame}


\begin{frame}[fragile]
\frametitle{Функции вместо операторите < и >}

\begin{columns}[t]
  \begin{column}{0.5\textwidth}

\begin{flushleft}
\relscale{0.7}
\begin{lstlisting}
bool compareGt (int a, int b)
{
  return a > b;
}
bool compareLt (int a, int b)
{
  return a < b;
}
\end{lstlisting}  
\end{flushleft}

  \end{column}
  \begin{column}{0.5\textwidth}

\begin{flushleft}
\relscale{0.1}
\begin{lstlisting}
int findIndexMax  
   (int arr[], int arrSize)
{
  int indexMax = 0;
  for (int i = 1; i < arrSize; i++)
    if (compareLt (arr[indexMax],arr[i]))
      indexMax = i;

  return indexMax;
}

int findIndexMin 
   (int arr[], int arrSize)
{
  int indexMax = 0;
  for (int i = 1; i < arrSize; i++)
    if (compareGt (arr[indexMax],arr[i]))
      indexMax = i;

  return indexMax;
}
\end{lstlisting}  
\end{flushleft}
  \end{column}
\end{columns}

\end{frame}

\begin{frame}[fragile]
\frametitle{Функциите имат тип}

\begin{columns}[t]
  \begin{column}{0.5\textwidth}

\begin{flushleft}
\relscale{0.7}
\begin{lstlisting}
bool compareGt (int a, int b)
{return a > b;}
bool compareLt (int a, int b)
{return a < b;}
\end{lstlisting}  
\end{flushleft}
\pause
\begin{flushleft}
\relscale{0.7}
\begin{lstlisting}
int main (){
  //variable definition: 
  //pComparator
  bool (*pComparator) (int,int); 

  //pointer assignment
  pComparator = compareLt; 
  cout << pComparator (1,2);

  pComparator = compareGt; 
  cout << pComparator (1,2);
}
\end{lstlisting}  
\end{flushleft}

  \end{column}
  \begin{column}{0.5\textwidth}

$comparator: int \times int \rightarrow bool$

\pause
\vspace{50px}
$ptrFn: T_1 \times T_2 \times ... \times T_k \rightarrow T_{res}$
\pause
\begin{flushleft}
\relscale{0.7}
\begin{lstlisting}
Tres (*ptrFn) (T1,T2,...,Tk);
\end{lstlisting}  
\end{flushleft}

  \end{column}
\end{columns}

\end{frame}


\begin{frame}[fragile]
\frametitle{Предаване на функции като параметри}

\begin{columns}[t]
  \begin{column}{0.5\textwidth}

\begin{flushleft}
\relscale{0.7}
\begin{lstlisting}
int findExtremum 
   (int arr[], 
    int arrSize, 
    bool (*pComparator)(int,int))
{
  int indexMax = 0;
  for (int i = 1; i < arrSize; i++)
    if (pComparator (arr[indexMax],arr[i]))
      indexMax = i;

  return indexMax;
}
\end{lstlisting}  
\end{flushleft}

  \end{column}
  \begin{column}{0.5\textwidth}

\begin{flushleft}
\relscale{0.5}
\begin{lstlisting}
bool compareGt (int a, int b)
{return a > b;}
bool compareLt (int a, int b)
{return a < b;}
\end{lstlisting}  
\end{flushleft}

  \end{column}
\end{columns}

\end{frame}





\begin{frame}[fragile]
\frametitle{Предаване на функции като параметри}

\begin{flushleft}
\relscale{0.7}
\begin{lstlisting}
void sort (int arr[], 
           int arrSize,
           bool (*pComparator)(int,int))
{
  for (int i = 0; i < arrSize-1; i++)
  {
    //find subarray extremum and
    //swap with a[i]
    swap (arr[i],
          arr[i+findExtremum(arr+i,
                             arrSize-i,
                             pComparator)]);
  }

}
\end{lstlisting}  
\end{flushleft}


\end{frame}


\begin{frame}[fragile]
\frametitle{Предаване на функции като параметри}



\vspace{-20px}

\begin{columns}[t]
  \begin{column}{0.5\textwidth}
\begin{flushleft}
\relscale{0.7}
\begin{lstlisting}
int main ()
{
  int arr[] = {1,7,3,5,2,3,2,4};

  sort (arr,8,compareLt);
  printArray (arr,8);

  sort (arr,8,compareGt);
  printArray (arr,8);

  return 0;
}
\end{lstlisting}  
\end{flushleft}
  \end{column}
  \begin{column}{0.5\textwidth}
\begin{flushleft}
\relscale{0.5}
\begin{lstlisting}
bool compareGt (int a, int b)
{return a > b;}
bool compareLt (int a, int b)
{return a < b;}

int findExtremum 
   (int arr[], 
    int arrSize, 
    bool (*pComparator)(int,int))
{
  int indexMax = 0;
  for (int i = 1; i < arrSize; i++)
    if (pComparator (arr[indexMax],arr[i]))
      indexMax = i;
  return indexMax;
}


void sort (int arr[], 
           int arrSize,
           bool (*pComparator)(int,int))
{
  for (int i = 0; i < arrSize-1; i++)
  {
    swap (arr[i],
          arr[i+findExtremum(arr+i,
                             arrSize-i,
                             pComparator)]);
  }
}
\end{lstlisting}  
\end{flushleft}
  \end{column}
\end{columns}


\end{frame}








\begin{frame}
\centerline{Благодаря за вниманието!}
\end{frame}



\end{document}



\begin{columns}[t]
  \begin{column}{0.55\textwidth}

  \end{column}
  \begin{column}{0.45\textwidth}

  \end{column}
\end{columns}
