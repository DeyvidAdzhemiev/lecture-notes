\documentclass{beamer}
\usepackage{relsize}
\usepackage{color}

\usepackage{listings}
\usetheme{CambridgeUS}
%\usepackage{beamerthemesplit} % new 
\usepackage{enumitem}
\usepackage{amsmath}                    % See geometry.pdf to learn the layout options. 
\usepackage{amsthm}                   % See geometry.pdf to learn the layout options. There 
\usepackage{amssymb}                    % See geometry.pdf to learn the layout options. 
\usepackage[utf8]{inputenc} 
\usepackage{graphicx}
\usepackage[english,bulgarian]{babel}

\lstset{language=C++,
                basicstyle=\ttfamily,
                keywordstyle=\color{blue}\ttfamily,
                stringstyle=\color{red}\ttfamily,
                commentstyle=\color{green}\ttfamily,
                morecomment=[l][\color{magenta}]{\#}
}

\newtheorem{mydef}{Дефиниция}[section]
\newtheorem{lem}{Лема}[section]
\newtheorem{thm}{Твърдение}[section]

\DeclareMathOperator{\restrict}{\upharpoonright}

\setitemize{label=\usebeamerfont*{itemize item}%
  \usebeamercolor[fg]{itemize item}
  \usebeamertemplate{itemize item}}

\setbeamercovered{transparent}



\begin{document}
\title[Обектно ориентирано програмиране]{Типове III: Предифиниране на типове. Шаблони на указатели към функции} 
\author{Калин Георгиев} 
\frame{\titlepage} 


\section{Шаблони + Указатели} 




\begin{frame}
\centerline{Внимание: \textbf{-std=c++11}}
\end{frame}

\begin{frame}
\centerline{Предефиниране на типове}
\end{frame}

\begin{frame}[fragile]
\frametitle{Предефиниране на типове}

\begin{itemize}
  \item Използване на ``сложен'' тип:
\end{itemize}

\begin{flushleft}
\relscale{0.7}
\begin{lstlisting}
void doSomething (int myMatrix[10][20])

int main ()
{
  int m[10][20] = {...};
  doSomething (m);
}
\end{lstlisting}	
\end{flushleft}

\end{frame}

\begin{frame}[fragile]
\frametitle{Предефиниране на типове}

\begin{itemize}
  \item ``Полагане'' на ново име на тип:
\end{itemize}



\begin{columns}[t]
  \begin{column}{0.5\textwidth}
\begin{flushleft}
\relscale{0.7}
\begin{lstlisting}
using myarr = int[10][20];

void doSomething (myarr myMatrix)

int main ()
{
  myarr m = {...};
  doSomething (m);

  //???
  myarr x[10];
}
\end{lstlisting}  
\end{flushleft}


  \end{column}
  \begin{column}{0.5\textwidth}
\begin{flushleft}
\relscale{0.7}
\begin{lstlisting}
void doSomething 
     (int myMatrix[10][20])

int main ()
{
  int m[10][20] = {...};
  doSomething (m);
}
\end{lstlisting}  
\end{flushleft}

  \end{column}
\end{columns}


\end{frame}

\begin{frame}
\centerline{Тип на указател към фунцкия}
\end{frame}


\begin{frame}[fragile]
\frametitle{Указател към функция}

\begin{columns}[t]
  \begin{column}{0.5\textwidth}

\begin{flushleft}
\relscale{0.8}
\begin{lstlisting}
using Comparator = 
    bool (*)(int,int);
\end{lstlisting}  
\end{flushleft}

\vspace{15px}
$comparator: int \times int \rightarrow bool$

  \end{column}
  \begin{column}{0.5\textwidth}

\begin{flushleft}
\relscale{0.5}
\begin{lstlisting}
bool compareGt (int a, int b)
{return a > b;}
bool compareLt (int a, int b)
{return a < b;}
\end{lstlisting}  
\end{flushleft}

  \end{column}
\end{columns}

\end{frame}


\begin{frame}[fragile]
\frametitle{Предаване на функции като параметри}

\begin{flushleft}
\relscale{0.7}
\begin{lstlisting}
using Comparator = bool (*)(int,int);

//int findExtremum (int arr[], 
//                  int arrSize, 
//                  bool (*pComparator)(int,int));

int findExtremum (int arr[], 
                  int arrSize, 
                  Comparator pComparator);
{
  int indexMax = 0;
  for (int i = 1; i < arrSize; i++)
    if (pComparator (arr[indexMax],arr[i]))
      indexMax = i;

  return indexMax;
}
\end{lstlisting}  
\end{flushleft}

  
\end{frame}

\begin{frame}
\centerline{Шаблони на указатели към функции}
\end{frame}

\begin{frame}[fragile]
\frametitle{Шаблон на указател}





\begin{columns}[t]
  \begin{column}{0.55\textwidth}
\begin{flushleft}
\relscale{0.7}
\begin{lstlisting}


template <typename T>
using Comparator = bool (*)(T,T);

template <typename T>
int findExtremum (T arr[], 
                  int arrSize, 
                  Comparator<T> pComparator);
{
  int indexMax = 0;
  for (int i = 1; i < arrSize; i++)
    if (pComparator (arr[indexMax],arr[i]))
      indexMax = i;

  return indexMax;
}
\end{lstlisting}  
\end{flushleft}
  \end{column}
  \begin{column}{0.45\textwidth}
\begin{flushleft}
\relscale{0.5}
\begin{lstlisting}

//int findExtremum 
//    (int arr[], 
//     int arrSize, 
//     bool (*pComparator)(int,int));

using Comparator = bool (*)(int,int);

int findExtremum (int arr[], 
                  int arrSize, 
                  Comparator pComparator);

\end{lstlisting}  
\end{flushleft}
  \end{column}
\end{columns}


  


\end{frame}



\begin{frame}[fragile]
\frametitle{Пример}



\begin{columns}[t]
  \begin{column}{0.55\textwidth}
\begin{flushleft}
\relscale{0.7}
\begin{lstlisting}
template <typename T>
bool compareGt (T a, T b) 
{return a > b;}

bool compareGt (char a, char b) 
{return a < b;}

template <typename T>
bool compareLt (T a, T b) 
{return a < b;}

int main ()
{
  int ia[] = {1,3,5};
  double da[] = {1.7,6.5,3.4,5.8};
  char ca = "abz";

  cout << findExtremum<int> (ia,3,compareGt<int>);
  cout << findExtremum<double> (da,4,compareGt<double>);
  cout << findExtremum<char> (ca,3,compareGt<char>);
}

\end{lstlisting}  
\end{flushleft}
  \end{column}
  \begin{column}{0.45\textwidth}
\begin{flushleft}
\relscale{0.5}
\begin{lstlisting}
template <typename T>
using Comparator = bool (*)(T,T);

template <typename T>
int findExtremum (T arr[], 
                  int arrSize, 
                  Comparator<T> pComparator);
{
  int indexMax = 0;
  for (int i = 1; i < arrSize; i++)
    if (pComparator (arr[indexMax],arr[i]))
      indexMax = i;

  return indexMax;
}
\end{lstlisting}  
\end{flushleft}
  \end{column}
\end{columns}
  
\end{frame}


\begin{frame}
\centerline{Благодаря за вниманието!}
\end{frame}



\end{document}



\begin{columns}[t]
  \begin{column}{0.55\textwidth}

  \end{column}
  \begin{column}{0.45\textwidth}

  \end{column}
\end{columns}
