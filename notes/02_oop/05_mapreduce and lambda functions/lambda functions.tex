\documentclass{beamer}
\usepackage{relsize}
\usepackage{color}

\usepackage{listings}
\usetheme{CambridgeUS}
%\usepackage{beamerthemesplit} % new
\usepackage{enumitem}
\usepackage{amsmath}                    % See geometry.pdf to learn the layout options.
\usepackage{amsthm}                   % See geometry.pdf to learn the layout options. There
\usepackage{amssymb}                    % See geometry.pdf to learn the layout options.
\usepackage[utf8]{inputenc}
\usepackage{graphicx}
\usepackage[english,bulgarian]{babel}

\lstset{language=C++,
                basicstyle=\ttfamily,
                keywordstyle=\color{blue}\ttfamily,
                stringstyle=\color{red}\ttfamily,
                commentstyle=\color{green}\ttfamily,
                morecomment=[l][\color{magenta}]{\#}
}

\newtheorem{mydef}{Дефиниция}[section]
\newtheorem{lem}{Лема}[section]
\newtheorem{thm}{Твърдение}[section]

\DeclareMathOperator{\restrict}{\upharpoonright}

\setitemize{label=\usebeamerfont*{itemize item}%
  \usebeamercolor[fg]{itemize item}
  \usebeamertemplate{itemize item}}

\setbeamercovered{transparent}



\begin{document}
\title[Обектно ориентирано програмиране]{$\lambda$ функции и std::function}
\author{Калин Георгиев}
\frame{\titlepage}

\section{$\lambda$}


\begin{frame}
    \centerline{$\lambda$ функции}
\end{frame}

\begin{frame}[fragile]
    \frametitle{Недостатъци на \texttt{(*)}}
    
\begin{lstlisting}[basicstyle=\tiny]
template <class T>
void map (T a[], size_t n, T(*f)(T))
{ 
    for (size_t i = 0; i < n; i++)
    {
        a[i] = f(a[i]);
    }
}

int plus1 (int x)
{return x+1};
int plus2 (int x)
{return x+2};
int mult2 (int x)
{return x*2};

int main ()
{
    int a[] = {1,2,3};
    map (a,3,plus1);
    map (a,3,plus2);
    map (a,3,mult2);
}
\end{lstlisting}
\end{frame}



\begin{frame}[fragile]
    \frametitle{Недостатъци на \texttt{(*)}}
    
    \begin{itemize}
        \item необходимо е да имаме компилиран код, чиито адрес да вземем
        \item не можем да създаваме функции ``в движение'', които да зависят от контекста
    \end{itemize}
    
\begin{lstlisting}[basicstyle=\tiny]
using doublefn = double (*) (double);

doublefn squared (doublefn f)
{ 
    return ???x->f(f(x))???
}

int main ()
{
    std::cout << squared(f)(3);
}
\end{lstlisting}
\end{frame}
    


\begin{frame}[fragile]
    \frametitle{Прости ламбда функции}
    
\begin{lstlisting}[]

    map (a,3,[](int x)->int{return x+1;});
    map (a,3,[](int x)->int{return x+2;});
    map (a,3,[](int x)->int{return x*2;});

\end{lstlisting}
\end{frame}

\begin{frame}
    \centerline{\texttt{std::function}}
\end{frame}


\begin{frame}[fragile]
\frametitle{Извикваем (callable) обект}

Всичко, над което може да се приложи оператор \texttt{()}.

\begin{itemize}
    \item функции
    \item $\lambda$ функции
    \item структури/класове с оператор ()
    \item \texttt{std::bind} изрази
\end{itemize}

\end{frame}

\begin{frame}[fragile]
    \frametitle{Извикваем (callable) обект}

\begin{lstlisting}[basicstyle=\small]
class increase
{
    public:
    increase (int _inc):inc (_inc){}
    int operator () (int x)
    {
        return x+inc;
    }
    private:
    int inc;
};
int main ()
{
    increase inc1{5}, inc2{10};
    std::cout << inc1(1) << std::endl;
    std::cout << inc2(1) << std::endl;
}    
\end{lstlisting}    
\end{frame}
    

\begin{frame}[fragile]
    \frametitle{\texttt{std::function}}

Обвива всякакви callable обекти.
\bigskip

\begin{lstlisting}[basicstyle=\small]
std::function<int(int)> f;
f = plus1; 
std::cout << f(1);

f = increase{1}; 
std::cout << f(1);

f = [](int x)->int{return x+1;}; 
std::cout << f(1);
\end{lstlisting}    
\end{frame}



\begin{frame}[fragile]
    \frametitle{Capture list}
    
\begin{lstlisting}[basicstyle=\small]
using doublefn = std::function<double(double)>;

doublefn squared (doublefn f)
{ 
    return [f](double x)
                ->double{return f(x)*f(x);};
}

int main ()
{
    std::cout << squared ([](double x)
                ->double{return x*x;}) (10);
}
\end{lstlisting}
\end{frame}



\begin{frame}[fragile]
    \frametitle{Примери}
    
\begin{itemize} 
    \item $f(g(x))$
    \item $f_1(f_2(...f_k(x)...))$
    \item $f^k(x)$
    \item Функция по списък от двойки
\end{itemize}

\end{frame}

\begin{frame}
    \centerline{\texttt{std::bind} изрази}
\end{frame}


\begin{frame}[fragile]
    \frametitle{Фиксиране на аргументи на фукция}
    
\begin{lstlisting}[basicstyle=\small]
double quadratic (double a, double b, double c, double x)
{return a*x*x + b*x + c;}

int main ()
{
   using namespace std::placeholders; 
   std::function<double(double)> 
        qf = std::bind(quadratic,4,2,1,_1);

   std::cout << qf (0);
}

\end{lstlisting}
\end{frame}


\begin{frame}[fragile]
    \frametitle{Метод като функция}
    
\begin{lstlisting}[basicstyle=\small]
struct power
{
    double operator () (double x)
    {return pow (x,exp);}
    double exp;
};
    
int main ()
{
   using namespace std::placeholders; 
   power pw2 = power{2};
   std::function<double(double)> 
       squared = std::bind (&power::operator(),&pw2,_1);

   std::cout << squared(4);
}

\end{lstlisting}
\end{frame}


\begin{frame}[fragile]
    \frametitle{Алтернативен подход с $\lambda$}
    
\begin{lstlisting}[basicstyle=\tiny]
double quadratic (double a, double b, double c, double x)
{return a*x*x + b*x + c;}

struct power
{
    double operator () (double x)
    {return pow (x,exp);}
    double exp;
};

int main ()
{
   std::function<double(double)> 
      qf2 = [](double x)->double{return quadratic(4,2,1,x);};

   power pw2 = power{2};

   std::function<double(double)> 
      squared2 = [&pw2](double x)->double{return pw2(x);};

   std::cout << qf2 (0) << std::endl;
   std::cout << squared2 (4) << std::endl;
}
\end{lstlisting}
\end{frame}



\begin{frame}
\centerline{Въпроси?}
\end{frame}


\end{document}

\begin{columns}[t]
  \begin{column}{0.55\textwidth}

  \end{column}
  \begin{column}{0.45\textwidth}

  \end{column}
\end{columns}
