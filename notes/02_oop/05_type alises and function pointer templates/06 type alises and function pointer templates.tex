\documentclass{beamer}
\usepackage{relsize}
\usepackage{color}

\usepackage{listings}
\usetheme{CambridgeUS}
%\usepackage{beamerthemesplit} % new 
\usepackage{enumitem}
\usepackage{amsmath}                    % See geometry.pdf to learn the layout options. 
\usepackage{amsthm}                   % See geometry.pdf to learn the layout options. There 
\usepackage{amssymb}                    % See geometry.pdf to learn the layout options. 
\usepackage[utf8]{inputenc} 
\usepackage{graphicx}
\usepackage[english,bulgarian]{babel}

\lstset{language=C++,
                basicstyle=\ttfamily,
                keywordstyle=\color{blue}\ttfamily,
                stringstyle=\color{red}\ttfamily,
                commentstyle=\color{green}\ttfamily,
                morecomment=[l][\color{magenta}]{\#}
}

\newtheorem{mydef}{Дефиниция}[section]
\newtheorem{lem}{Лема}[section]
\newtheorem{thm}{Твърдение}[section]

\DeclareMathOperator{\restrict}{\upharpoonright}

\setitemize{label=\usebeamerfont*{itemize item}%
  \usebeamercolor[fg]{itemize item}
  \usebeamertemplate{itemize item}}

\setbeamercovered{transparent}



\begin{document}
\title[Обектно ориентирано програмиране]{Типове II: Шаблони на функции. Указатели към функции} 
\author{Калин Георгиев} 
\frame{\titlepage} 


\section{Шаблони + Указатели} 




\begin{frame}
\centerline{Внимание: \textbf{-std=c++11}}
\end{frame}

\begin{frame}
\centerline{Предефиниране на типове}
\end{frame}

\begin{frame}[fragile]
\frametitle{Предефиниране на типове}

\begin{itemize}
  \item Използване на ``сложен'' тип:
\end{itemize}

\begin{flushleft}
\relscale{0.7}
\begin{lstlisting}
void doSomething (int myMatrix[10][20])

int main ()
{
  int m[10][20] = {...};
  doSomething (m);
}
\end{lstlisting}	
\end{flushleft}

\end{frame}

\begin{frame}[fragile]
\frametitle{Предефиниране на типове}

\begin{itemize}
  \item ``Полагане'' на ново име на тип:
\end{itemize}



\begin{columns}[t]
  \begin{column}{0.5\textwidth}
\begin{flushleft}
\relscale{0.7}
\begin{lstlisting}
using myarr = int[10][20];

void doSomething (myarr myMatrix)

int main ()
{
  myarr m = {...};
  doSomething (m);

  //???
  myarr x[10];
}
\end{lstlisting}  
\end{flushleft}


  \end{column}
  \begin{column}{0.5\textwidth}
\begin{flushleft}
\relscale{0.7}
\begin{lstlisting}
void doSomething 
     (int myMatrix[10][20])

int main ()
{
  int m[10][20] = {...};
  doSomething (m);
}
\end{lstlisting}  
\end{flushleft}

  \end{column}
\end{columns}


\end{frame}

\begin{frame}
\centerline{Тип на указател към фунцкия}
\end{frame}


\begin{frame}[fragile]
\frametitle{Указател към функция}

\begin{columns}[t]
  \begin{column}{0.5\textwidth}

\begin{flushleft}
\relscale{0.8}
\begin{lstlisting}
using Comparator = 
    bool (*)(int,int);
\end{lstlisting}  
\end{flushleft}

\vspace{15px}
$comparator: int \times int \rightarrow bool$

  \end{column}
  \begin{column}{0.5\textwidth}

\begin{flushleft}
\relscale{0.5}
\begin{lstlisting}
bool compareGt (int a, int b)
{return a > b;}
bool compareLt (int a, int b)
{return a < b;}
\end{lstlisting}  
\end{flushleft}

  \end{column}
\end{columns}

\end{frame}


\begin{frame}[fragile]
\frametitle{Предаване на функции като параметри}

\begin{flushleft}
\relscale{0.7}
\begin{lstlisting}
using Comparator = bool (*)(int,int);

//int findExtremum (int arr[], 
//                  int arrSize, 
//                  bool (*pComparator)(int,int));

int findExtremum (int arr[], 
                  int arrSize, 
                  Comparator pComparator);
{
  int indexMax = 0;
  for (int i = 1; i < arrSize; i++)
    if (pComparator (arr[indexMax],arr[i]))
      indexMax = i;

  return indexMax;
}
\end{lstlisting}  
\end{flushleft}

  
\end{frame}

\begin{frame}
\centerline{Шаблони на указатели към функции}
\end{frame}

\begin{frame}[fragile]
\frametitle{Шаблон на указател}





\begin{columns}[t]
  \begin{column}{0.55\textwidth}
\begin{flushleft}
\relscale{0.7}
\begin{lstlisting}


template <typename T>
using Comparator = bool (*)(T,T);

template <typename T>
int findExtremum (T arr[], 
                  int arrSize, 
                  Comparator<T> pComparator);
{
  int indexMax = 0;
  for (int i = 1; i < arrSize; i++)
    if (pComparator (arr[indexMax],arr[i]))
      indexMax = i;

  return indexMax;
}
\end{lstlisting}  
\end{flushleft}
  \end{column}
  \begin{column}{0.45\textwidth}
\begin{flushleft}
\relscale{0.5}
\begin{lstlisting}

//int findExtremum 
//    (int arr[], 
//     int arrSize, 
//     bool (*pComparator)(int,int));

using Comparator = bool (*)(int,int);

int findExtremum (int arr[], 
                  int arrSize, 
                  Comparator pComparator);

\end{lstlisting}  
\end{flushleft}
  \end{column}
\end{columns}


  


\end{frame}



\begin{frame}[fragile]
\frametitle{Пример}



\begin{columns}[t]
  \begin{column}{0.55\textwidth}
\begin{flushleft}
\relscale{0.7}
\begin{lstlisting}
template <typename T>
bool compareGt (T a, T b) 
{return a > b;}

bool compareGt (char a, char b) 
{return a < b;}

template <typename T>
bool compareLt (T a, T b) 
{return a < b;}

int main ()
{
  int ia[] = {1,3,5};
  double da[] = {1.7,6.5,3.4,5.8};
  char ca = "abz";

  cout << findExtremum<int> (ia,3,compareGt<int>);
  cout << findExtremum<double> (da,4,compareGt<double>);
  cout << findExtremum<char> (ca,3,compareGt<char>);
}

\end{lstlisting}  
\end{flushleft}
  \end{column}
  \begin{column}{0.45\textwidth}
\begin{flushleft}
\relscale{0.5}
\begin{lstlisting}
template <typename T>
using Comparator = bool (*)(T,T);

template <typename T>
int findExtremum (T arr[], 
                  int arrSize, 
                  Comparator<T> pComparator);
{
  int indexMax = 0;
  for (int i = 1; i < arrSize; i++)
    if (pComparator (arr[indexMax],arr[i]))
      indexMax = i;

  return indexMax;
}
\end{lstlisting}  
\end{flushleft}
  \end{column}
\end{columns}
  
\end{frame}




\section{Map.Reduce} 

\begin{frame}
\centerline{Map.Reduce}
\end{frame}

\begin{frame}[fragile]
\frametitle{Задача: еднотипна промяна на всеки елемент на масив}

\begin{flushleft}
\relscale{0.7}
\begin{lstlisting}
void increase (int arr[], int arrsize)
{
  for (int i = 0; i < arrsize; i++)
    arr[i] = arr[i]+1;
}
void multiply (int arr[], int arrsize)
{
  for (int i = 0; i < arrsize; i++)
    arr[i] = arr[i]*2;
}
void increaseEvens (int arr[], int arrsize)
{
  for (int i = 0; i < arrsize; i++)
    if (arr[i] %2 == 0)
      arr[i] = arr[i] + 1;
}
\end{lstlisting}  
\end{flushleft}

  
\end{frame}


\begin{frame}[fragile]
\frametitle{Map}

\begin{center}
$map: T \rightarrow T$
\end{center}

\begin{itemize}
  \item еднотипна обработка на всеки от елементите на масив
\end{itemize}

\begin{flushleft}
\relscale{0.7}
\begin{lstlisting}
template <typename T>
using mapFn = T (*) (T);

template <typename T>
void map (T arr[], int arrsize, mapFn<T> f)
{
  for (int i = 0; i < arrsize; i++)
    arr[i] = f(arr[i]);
}
\end{lstlisting}  
\end{flushleft}

  
\end{frame}



\begin{frame}[fragile]
\frametitle{Пример: добавяне на единица}


\begin{columns}[t]
  \begin{column}{0.5\textwidth}

\begin{flushleft}
\relscale{0.7}
\begin{lstlisting}
int plusOne (int x)
{return x+1;}

int multTwo (int x)
{return x*2;}

int main ()
{
  int arr[] = {1,2,3};
  map<int> (arr,3,plusOne);
  map<int> (arr,3,multTwo);

  printArray<int> (arr,3);
}

\end{lstlisting}  
\end{flushleft}
  \end{column}
  \begin{column}{0.5\textwidth}
\begin{flushleft}
\relscale{0.5}
\begin{lstlisting}
template <typename T>
using mapFn = T (*) (T);

template <typename T>
void map (T arr[], int arrsize, mapFn<T> f)
{
  for (int i = 0; i < arrsize; i++)
    arr[i] = f(arr[i]);
}
\end{lstlisting}  
\end{flushleft}

  \end{column}
\end{columns}


\end{frame}

\begin{frame}[fragile]
\frametitle{Пример: добавяне на единица само на четните елементи}


\begin{columns}[t]
  \begin{column}{0.5\textwidth}

\begin{flushleft}
\relscale{0.7}
\begin{lstlisting}
int evenPlusOne (int x)
{
  if (x%2 == 0)
    return x+1;
  return x;  
}

int main ()
{
  int arr[] = {1,2,3};
  map<int> (arr,3,evenPlusOne);

  printArray (arr,3);
}

\end{lstlisting}  
\end{flushleft}
  \end{column}
  \begin{column}{0.5\textwidth}
\begin{flushleft}
\relscale{0.5}
\begin{lstlisting}
template <typename T>
using mapFn = T (*) (T);

template <typename T>
void map (T arr[], int arrsize, mapFn<T> f)
{
  for (int i = 0; i < arrsize; i++)
    arr[i] = f(arr[i]);
}
\end{lstlisting}  
\end{flushleft}

  \end{column}
\end{columns}


\end{frame}



\begin{frame}[fragile]
\frametitle{Задача: намиране на сума, произведение, брой и пр.}

\begin{flushleft}
\relscale{0.7}
\begin{lstlisting}
int sum (int arr[], int arrsize)
{
  int result = arr[0];
  for (int i = 1; i < arrsize; i++)
    result = result + arr[i];
  return result;
}
int prod (int arr[], int arrsize)
{
  int result = arr[0];
  for (int i = 1; i < arrsize; i++)
    result = result * arr[i];
  return result;
}
int countEvens (int arr[], int arrsize)
{
  int result = 0;
  for (int i = 1; i < arrsize; i++)
    if (arr[i] % 2 == 0)
      result = result + 1;
  return result;
}
\end{lstlisting}  
\end{flushleft}

  
\end{frame}



\begin{frame}[fragile]
\frametitle{Reduce}

\begin{center}
$OP: R \times E \rightarrow R$
\end{center}

\begin{itemize}
  \item Сумиране (``акумулиране'', ``обединяване'') на всички елементи в един резултат
\end{itemize}

\begin{flushleft}
\relscale{0.7}
\begin{lstlisting}
template <typename ResT, typename ElemT>
using reduceFn = ResT (*) (ResT, ElemT);

template <typename ResT,typename ElemT>
ResT reduce (ElemT arr[], 
             int arrsize, 
             reduceFn<ResT,ElemT> f, 
             ResT init)
{
  ResT result = init;

  for (int i = 0; i < arrsize; i++)
    result = f (result,arr[i]);

  return result;
}
\end{lstlisting}  
\end{flushleft}

  
\end{frame}



\begin{frame}[fragile]
\frametitle{Пример: Събиране и умножение}


\begin{columns}[t]
  \begin{column}{0.55\textwidth}

\begin{flushleft}
\relscale{0.7}
\begin{lstlisting}
int sum (int accumulated, int x)
{return accumulated + x;}

int prod (int accumulated, int x)
{return accumulated * x;}

int main ()
{
  int arr[] = {1,2,3};
  cout << reduce<int,int> (arr,3,sum,0);
  cout << reduce<int,int> (arr,3,prod,1);
}

\end{lstlisting}  
\end{flushleft}
  \end{column}
  \begin{column}{0.45\textwidth}
\begin{flushleft}
\vspace{-30px}
\relscale{0.45}
\begin{lstlisting}
template <typename ResT, typename ElemT>
using reduceFn = ResT (*) (ResT, ElemT);

template <typename ResT,typename ElemT>
ResT reduce (ElemT arr[], 
             int arrsize, 
             reduceFn<ResT,ElemT> f, 
             ResT init)
{
  ResT result = init;

  for (int i = 1; i < arrsize; i++)
    result = f (result,arr[i]);

  return result;
}
\end{lstlisting}  
\end{flushleft}

  \end{column}
\end{columns}


\end{frame}


\begin{frame}[fragile]
\frametitle{Пример: Събиране само на четните числа}


\begin{columns}[t]
  \begin{column}{0.55\textwidth}

\begin{flushleft}
\relscale{0.7}
\begin{lstlisting}
int sumEvens (int accumulated, int x)
{
  if (x % 2 == 0)
    return accumulated + x;
  return accumulated;
}


int main ()
{
  int arr[] = {1,2,3};
  cout << reduce<int,int> (arr,3,sumEvens,0);
}

\end{lstlisting}  
\end{flushleft}
  \end{column}
  \begin{column}{0.45\textwidth}
\begin{flushleft}
\vspace{-30px}
\relscale{0.5}
\begin{lstlisting}
template <typename ResT, typename ElemT>
using reduceFn = ResT (*) (ResT, ElemT);

template <typename ResT,typename ElemT>
ResT reduce (ElemT arr[], 
             int arrsize, 
             reduceFn<ResT,ElemT> f, 
             ResT init)
{
  ResT result = init;

  for (int i = 1; i < arrsize; i++)
    result = f (result,arr[i]);

  return result;
}
\end{lstlisting}  
\end{flushleft}

  \end{column}
\end{columns}

\end{frame}

\begin{frame}[fragile]
\frametitle{Пример: Проверка дали има четни числа}


\begin{columns}[t]
  \begin{column}{0.55\textwidth}

\begin{flushleft}
\relscale{0.7}
\begin{lstlisting}
bool isEven (bool accumulated, int x)
{
  if (x % 2 == 0)
    return true;
  return accumulated;
}


int main ()
{
  int arr[] = {1,2,3};
  cout << reduce<bool,int> (arr,3,isEven,false);
}

\end{lstlisting}  
\end{flushleft}
  \end{column}
  \begin{column}{0.45\textwidth}
\begin{flushleft}
\vspace{-30px}
\relscale{0.5}
\begin{lstlisting}
template <typename ResT, typename ElemT>
using reduceFn = ResT (*) (ResT, ElemT);

template <typename ResT,typename ElemT>
ResT reduce (ElemT arr[], 
             int arrsize, 
             reduceFn<ResT,ElemT> f, 
             ResT init)
{
  ResT result = init;

  for (int i = 1; i < arrsize; i++)
    result = f (result,arr[i]);

  return result;
}
\end{lstlisting}  
\end{flushleft}

  \end{column}
\end{columns}


\end{frame}




\begin{frame}[fragile]
\frametitle{Пример: Брой срещания на символ}


\begin{columns}[t]
  \begin{column}{0.55\textwidth}

\begin{flushleft}
\relscale{0.7}
\begin{lstlisting}
int countLs (int accumulated, char x)
{
  if (x == 'l')
    return accumulated + 1;
  return accumulated;
}


int main ()
{
  cout << reduce<int,char> ("Hello World!",12,countLs,0);
}

\end{lstlisting}  
\end{flushleft}
  \end{column}
  \begin{column}{0.45\textwidth}
\begin{flushleft}
\vspace{-30px}
\relscale{0.5}
\begin{lstlisting}
template <typename ResT, typename ElemT>
using reduceFn = ResT (*) (ResT, ElemT);

template <typename ResT,typename ElemT>
ResT reduce (ElemT arr[], 
             int arrsize, 
             reduceFn<ResT,ElemT> f, 
             ResT init)
{
  ResT result = init;

  for (int i = 1; i < arrsize; i++)
    result = f (result,arr[i]);

  return result;
}
\end{lstlisting}  
\end{flushleft}

  \end{column}
\end{columns}


\end{frame}

\begin{frame}
\centerline{Благодаря за вниманието!}
\end{frame}



\end{document}



\begin{columns}[t]
  \begin{column}{0.55\textwidth}

  \end{column}
  \begin{column}{0.45\textwidth}

  \end{column}
\end{columns}
