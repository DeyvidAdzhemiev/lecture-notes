\documentclass{beamer}
\usepackage{relsize}
\usepackage{color}

\usepackage{listings}
\usetheme{CambridgeUS}
%\usepackage{beamerthemesplit} % new 
\usepackage{enumitem}
\usepackage{amsmath}                    % See geometry.pdf to learn the layout options. 
\usepackage{amsthm}                   % See geometry.pdf to learn the layout options. There 
\usepackage{amssymb}                    % See geometry.pdf to learn the layout options. 
\usepackage[utf8]{inputenc} 
\usepackage{graphicx}
\usepackage[english,bulgarian]{babel}
\usepackage[framemethod=tikz]{mdframed}
\usepackage{caption}
\usepackage{tikz}
\usepackage{forest}
\usetikzlibrary{shapes,arrows,positioning,calc,chains}

\lstset{language=C++,
                basicstyle=\ttfamily,
                keywordstyle=\color{blue}\ttfamily,
                stringstyle=\color{red}\ttfamily,
                commentstyle=\color{green}\ttfamily,
                morecomment=[l][\color{magenta}]{\#}
}

\newtheorem{mydef}{Дефиниция}[section]
\newtheorem{lem}{Лема}[section]
\newtheorem{thm}{Твърдение}[section]

\DeclareMathOperator{\restrict}{\upharpoonright}

\setitemize{label=\usebeamerfont*{itemize item}%
  \usebeamercolor[fg]{itemize item}
  \usebeamertemplate{itemize item}}

\setbeamercovered{transparent}

\captionsetup{font=footnotesize}

\lstset{breaklines=true}
\tikzset{
block/.style = {draw, rectangle,align = center},
entry/.style = {draw, fill=black, circle, radius=3em},
condition/.style = {draw, fill=white, diamond, align = center,node distance=3cm},
fork/.style = {draw, fill=black, circle,inner sep=1pt},
lnode/.style={rectangle split, rectangle split parts=2,draw, rectangle split horizontal},
treenode/.style = {align=center, inner sep=0pt, text centered, circle, font=\sffamily\bfseries, draw=black, fill=white, text width=1.5em},
token/.style={rectangle split, rectangle split parts=2,draw, rectangle split horizontal=false}
}


\begin{document}
\title[Структури от данни и програмиране]{Персистентни СД. Умни указатели} 
\author{Калин Георгиев} 
\frame{\titlepage} 

\section{Персистентни СД} 


\begin{frame}
\centerline{Персистентни СД}
\end{frame}

\begin{frame}[fragile]
  \frametitle{Персистентни СД}
  
\begin{verbatim}
  let list = [3,2,1]
      list1 = 4:list
      list2 = 6:5:list
\end{verbatim}

\begin{figure}
  \centering

    \begin{tikzpicture}[auto, node distance=1.5cm,>=latex']
    \node[lnode] (l1) {\nodepart{one}3};
    \node[lnode, right of = l1] (l2) {\nodepart{one}2};
    \node[lnode, right of = l2] (l3) {\nodepart{one}1};

    \draw[*->] let \p1 = (l2.two), \p2 = (l2.center) in (\x1,\y2) -- (l3);
    \draw[*->] let \p1 = (l1.two), \p2 = (l1.center) in (\x1,\y2) -- (l2);

    \node[lnode, above left = of l1] (l11) {\nodepart{one}4};
    \node[lnode, below left = of l1] (l21) {\nodepart{one}5};
    \node[lnode, left of = l21] (l22) {\nodepart{one}6};

    \draw[*->] let \p1 = (l11.two), \p2 = (l11.center) in (\x1,\y2) -- (l1);
    \draw[*->] let \p1 = (l21.two), \p2 = (l21.center) in (\x1,\y2) -- (l1);
    \draw[*->] let \p1 = (l22.two), \p2 = (l22.center) in (\x1,\y2) -- (l21);


    \end{tikzpicture}
\end{figure}
  
\end{frame}
  
\begin{frame}[fragile]
  \frametitle{Модификация на данните}
  
\begin{verbatim}
  list<int> list ({1,2,3,4,5});
  //list.set(2,10); -невъзможно
  list<int> result = list.set(2,10);
\end{verbatim}

\begin{figure}
  \centering

    \begin{tikzpicture}[auto, node distance=1.5cm,>=latex']
    \node[lnode] (l1) {\nodepart{one}1};
    \node[lnode, right of = l1] (l2) {\nodepart{one}2};
    \node[lnode, right of = l2] (l3) {\nodepart{one}3};
    \node[lnode, right of = l3] (l4) {\nodepart{one}4};
    \node[lnode, right of = l4] (l5) {\nodepart{one}5};

    \draw[*->] let \p1 = (l4.two), \p2 = (l4.center) in (\x1,\y2) -- (l5);
    \draw[*->] let \p1 = (l3.two), \p2 = (l3.center) in (\x1,\y2) -- (l4);
    \draw[*->] let \p1 = (l2.two), \p2 = (l2.center) in (\x1,\y2) -- (l3);
    \draw[*->] let \p1 = (l1.two), \p2 = (l1.center) in (\x1,\y2) -- (l2);

    \node[lnode, below left = of l4, color = red] (l23) {\nodepart{one}10};
    \node[lnode, left of = l23, color = red] (l22) {\nodepart{one}2};
    \node[lnode, left of = l22, color = red] (l21) {\nodepart{one}1};

    \draw[*->,color = red] let \p1 = (l21.two), \p2 = (l21.center) in (\x1,\y2) -- (l22);
    \draw[*->,color = red] let \p1 = (l22.two), \p2 = (l22.center) in (\x1,\y2) -- (l23);
    \draw[*->,color = red] let \p1 = (l23.two), \p2 = (l23.center) in (\x1,\y2) -- (l4);


    \end{tikzpicture}
\end{figure}
  
\end{frame}

\section{Жизнен цикъл на паметта}

\begin{frame}
  \centerline{Потребител / собственик на памет}
\end{frame}


\begin{frame}[fragile]
  \frametitle{Създател, потребител, собственик}

  \begin{columns}[t]
    \begin{column}{0.50\textwidth}
      \begin{flushleft}
        \relscale{0.7}
        \begin{verbatim}
        struct node
        {int data; node* next};

        node* cons (int head, node* tail)
        {return new node{head,tail};}

        node* makelist ()
        {return cons (3, cons (2, cons (1, nullptr));}

        void uselists ()
        {
          node* list = makelist();
          node* list1 = cons (4, list);
          node* list2 = cons (6, cons (5, list));
        
          anotheruser1(list1);
          anotheruser2(list2);
        
          //delete list, list1, list2?
        }
        \end{verbatim} 
      \end{flushleft}       
    \end{column}
    \begin{column}{0.5\textwidth}
      \relscale{0.6}
      \begin{figure}
        \centering
          \begin{tikzpicture}[auto, node distance=1.4cm,>=latex']
          \node[lnode] (l1) {\nodepart{one}3};
          \node[lnode, right of = l1] (l2) {\nodepart{one}2};
          \node[lnode, right of = l2] (l3) {\nodepart{one}1};
      
          \draw[*->] let \p1 = (l2.two), \p2 = (l2.center) in (\x1,\y2) -- (l3);
          \draw[*->] let \p1 = (l1.two), \p2 = (l1.center) in (\x1,\y2) -- (l2);
      
          \node[lnode, above left = of l1] (l11) {\nodepart{one}4};
          \node[lnode, below left = of l1] (l21) {\nodepart{one}5};
          \node[lnode, left of = l21] (l22) {\nodepart{one}6};
      
          \draw[*->] let \p1 = (l11.two), \p2 = (l11.center) in (\x1,\y2) -- (l1);
          \draw[*->] let \p1 = (l21.two), \p2 = (l21.center) in (\x1,\y2) -- (l1);
          \draw[*->] let \p1 = (l22.two), \p2 = (l22.center) in (\x1,\y2) -- (l21);
          \end{tikzpicture}
      \end{figure}
        
    \end{column}
  \end{columns}
  
\end{frame}

\begin{frame}
  \centerline{Умни указатели}
\end{frame}


\begin{frame}[fragile]
  \frametitle{Reference counting. std::shared\_ptr<T>}
  
  \texttt{node*} $\rightarrow$ \texttt{std::shared\_ptr<node<T>{}>}
  
  \texttt{new} $\rightarrow$ \texttt{std::make\_shared}

\begin{figure}
  \centering

    \begin{tikzpicture}[auto, node distance=1.5cm,>=latex']
     \node[block] (data) {data};
     \node[token, left of = data] (control) {\nodepart{two}3};
  
      \node[block, above left = of control] (r1) {\nodepart{one}reference 1};
      \node[block, below of = r1] (r2) {\nodepart{one}reference 2};
      \node[block, below left = of control] (r3) {\nodepart{one}reference 3};

      \draw[->] (r1) -- (control);
      \draw[->] (r2) -- (control);
      \draw[->] (r3) -- (control);
 
      \draw[*->] let \p1 = (control.one), \p2 = (control.one) in (\x1,\y2) -- (data);

  
    \end{tikzpicture}
\end{figure}
  
\end{frame}


\begin{frame}[fragile]
  \frametitle{Reference counting}

  \relscale{0.6}
  \begin{figure}
    \centering
      \begin{tikzpicture}[auto, node distance=0.9cm,>=latex']
        \node[lnode] (l1) {\nodepart{one}3};
        \node[lnode, right of = l1] (l2) {\nodepart{one}2};
        \node[lnode, right of = l2] (l3) {\nodepart{one}1};
    
        \draw[*->] let \p1 = (l2.two), \p2 = (l2.center) in (\x1,\y2) -- (l3);
        \draw[*->] let \p1 = (l1.two), \p2 = (l1.center) in (\x1,\y2) -- (l2);
    
        \node[lnode, above left = of l1] (l11) {\nodepart{one}4};
        \node[lnode, below left = of l1] (l21) {\nodepart{one}5};
        \node[lnode, left of = l21] (l22) {\nodepart{one}6};
    
        \draw[*->] let \p1 = (l11.two), \p2 = (l11.center) in (\x1,\y2) -- (l1);
        \draw[*->] let \p1 = (l21.two), \p2 = (l21.center) in (\x1,\y2) -- (l1);
        \draw[*->] let \p1 = (l22.two), \p2 = (l22.center) in (\x1,\y2) -- (l21);

        \node[block,left of = l1,color = red] (rl) {list};
        \node[block,left of = l11, color = red] (rl1) {list1};
        \node[block,left of = l22, color = red] (rl2) {list2};

        \draw[->,color=red] (rl) -- (l1);
        \draw[->,color=red] (rl1) -- (l11);
        \draw[->,color=red] (rl2) -- (l22);


      \end{tikzpicture}
    \end{figure}

  \begin{flushleft}
    \relscale{1}
    \begin{verbatim}
    struct node
    {int data; shared_ptr<node> next};

    shared_prt<node> cons (int head, shared_ptr<node> tail)
    {return make_shared(head,tail);}

    shared_prt<node> makelist ()
    {return cons (3, cons (2, cons (1, nullptr));}

    void uselists ()
    {
      shared_prt<node> list = makelist();
      shared_prt<node> list1 = cons (4, list);
      shared_prt<node> list2 = cons (6, cons (5, list));
    
      anotheruser1(list1);
      anotheruser2(list2);
    }//~list, ~list1, ~list2
    \end{verbatim} 
  \end{flushleft}       
  
\end{frame}

\begin{frame}[fragile]
  \frametitle{Някои проблеми}
  
\begin{figure}
  \centering

    \begin{tikzpicture}[auto, node distance=1.5cm,>=latex']
    \node[lnode] (l1) {\nodepart{one}1};
    \node[lnode, above right = of l1] (l2) {\nodepart{one}2};
    \node[lnode, above left = of l1] (l3) {\nodepart{one}3};

    \draw[*->] let \p1 = (l2.two), \p2 = (l2.center) in (\x1,\y2) -- (l1);
    \draw[*->] let \p1 = (l3.two), \p2 = (l3.center) in (\x1,\y2) -- (l2);
    \draw[*->] let \p1 = (l1.two), \p2 = (l1.center) in (\x1,\y2) -- (l3);


    \end{tikzpicture}
\end{figure}
  
\end{frame}


\begin{frame}
\centerline{Благодаря за вниманието!}
\end{frame}


\end{document}

\begin{columns}[t]
  \begin{column}{0.55\textwidth}

  \end{column}
  \begin{column}{0.45\textwidth}

  \end{column}
\end{columns}


