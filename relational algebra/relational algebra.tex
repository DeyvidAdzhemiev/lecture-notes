\documentclass{beamer}
\usepackage{relsize}
\usepackage{color}

\usepackage{listings}
\usetheme{CambridgeUS}
%\usepackage{beamerthemesplit} % new 
\usepackage{enumitem}
\usepackage{amsmath}                    % See geometry.pdf to learn the layout options. 
\usepackage{amsthm}                   % See geometry.pdf to learn the layout options. There 
\usepackage{amssymb}                    % See geometry.pdf to learn the layout options. 
\usepackage[utf8]{inputenc} 
\usepackage{graphicx}
\usepackage[english,bulgarian]{babel}

\lstset{language=C++,
                basicstyle=\ttfamily,
                keywordstyle=\color{blue}\ttfamily,
                stringstyle=\color{red}\ttfamily,
                commentstyle=\color{green}\ttfamily,
                morecomment=[l][\color{magenta}]{\#}
}

\newtheorem{mydef}{Дефиниция}[section]
\newtheorem{lem}{Лема}[section]
\newtheorem{thm}{Твърдение}[section]

\DeclareMathOperator{\restrict}{\upharpoonright}

\setitemize{label=\usebeamerfont*{itemize item}%
  \usebeamercolor[fg]{itemize item}
  \usebeamertemplate{itemize item}}

\setbeamercovered{transparent}



\begin{document}
\title[Увод в програмирането]{\textit{Малко теория}\\ Релационна алгебра} 
\author{Калин Георгиев} 
\frame{\titlepage} 

\section{Релационнен модел} 
\subsection{Релации}


\begin{frame}
\centerline{Релации}
\end{frame}

\begin{frame}[fragile]
\frametitle{Какво е релация?}

\begin{itemize}
  \item Задава ``отношения'' между елементите на две множества 
\end{itemize}

\vspace{1em}

$\mathcal{A}nimals=\{cat,dog,crab\}$

$\mathcal{N}=\{0,1,2,...\}$

\vspace{1em}

$legs=\{(cat,2),(dog,2),(crab,8)\} \subseteq \mathcal{A}nimals \times \mathcal{N}$

\vspace{1em}

$eyes=\{(cat,2),(dog,2),(crab,2)\} \subseteq \mathcal{A}nimals \times \mathcal{N}$


\vspace{1em}

$eyesANDlegs=\{(cat,2,2),(dog,2,2),(crab,2,8)\} \subseteq \mathcal{A}nimals \times \mathcal{N} \times \mathcal{N}$

\vspace{1em}

\end{frame}


\begin{frame}[fragile]
\frametitle{Какво е релация?}
Коя е тази релация?

$\{(x,y)|x \in \mathcal{N}, y \in \mathcal{N}, \exists z \in \mathcal{N}-\{0\}:y=x+z\} \subseteq \mathcal{N} \times \mathcal {N}$


\pause

\begin{center}
  $\leq \subseteq \mathcal{N} \times \mathcal{N}$
\end{center}

\end{frame}


\subsection{Релационен модел}
\begin{frame}
\centerline{Релационен модел на данни}
\end{frame}

\begin{frame}[fragile]
\frametitle{Релационен модел}

$eyesANDlegs=\{(cat,2,2),(dog,2,2),(crab,2,8)\} \subseteq \mathcal{A}nimals \times \mathcal{N} \times \mathcal{N}$

\vspace{1em}

\begin{itemize}
  \item ``Човешки четимо'' задаване на релация
  \item Атрибути на елемент
  \item Схема на данните
\end{itemize}

\vspace{1em}

$eyesANDlegs=(animal:\mathcal{A}nimals,eyes:\mathcal{N},legs:\mathcal{N})$

\pause


\begin{center}
  
\begin{tabular}{ c | c | c }
  
  animal  & eyes  & legs \\ \hline  
  cat  & 2  & 2 \\
  dog & 2 & 2 \\
  crab & 2 & 8 \\
  
\end{tabular} 

\end{center}


\end{frame}

\section{Релационна алгебра}
\subsection{Операции}


\begin{frame}
\centerline{Някои операции в релационната алгебра}
\end{frame}

\begin{frame}[fragile]
\frametitle{Селекция}

\begin{center}
\begin{tabular}{ c | c | c }
  animal  & eyes  & legs \\ \hline  
  cat  & 2  & 2 \\
  dog & 2 & 2 \\
  crab & 2 & 8 \\
\end{tabular} 
\end{center}

\pause

\begin{center}
  $\sigma_p(r) = \{t|t \in r,p(r)\}$
\end{center}

\pause

\begin{center}
  $twolegs(r):legs=2$

  $\sigma_{twolegs}(eyesANDlegs)=\{t|t \in eyesANDlegs,twolegs(r)\}$
\end{center}

\begin{center}
\begin{tabular}{ c | c | c }
  animal  & eyes  & legs \\ \hline  
  cat  & 2  & 2 \\
  dog & 2 & 2 \\
\end{tabular} 
\end{center}


\end{frame}



\begin{frame}[fragile]
\frametitle{Проекция}

\begin{center}
\begin{tabular}{ c | c | c }
  animal  & eyes  & legs \\ \hline  
  cat  & 2  & 2 \\
  dog & 2 & 2 \\
  crab & 2 & 8 \\
\end{tabular} 
\end{center}

\pause

\begin{center}
  $\pi_{A_1,A_2,...,A_k}(r)$
\end{center}

\pause

\begin{center}
  $\pi_{animal,eyes}(eyesANDlegs)$
\end{center}

\begin{center}
\begin{tabular}{ c | c | c }
  animal  & eyes \\ \hline  
  cat  & 2   \\
  dog & 2 \\
  crab & 2 \\
\end{tabular} 
\end{center}


\end{frame}



\begin{frame}[fragile]
\frametitle{Проекция по селекция}

\begin{center}
\begin{tabular}{ c | c | c }
  animal  & eyes  & legs \\ \hline  
  cat  & 2  & 2 \\
  dog & 2 & 2 \\
  crab & 2 & 8 \\
\end{tabular} 
\end{center}

\pause

\begin{center}
  $\pi_{animal,eyes}(\sigma_{twolegs}(eyesANDlegs))$
\end{center}

\pause


\begin{center}
\begin{tabular}{ c | c }
  animal  & eyes \\ \hline  
  cat  & 2   \\
  dog & 2 \\
\end{tabular} 
\end{center}


\end{frame}


\begin{frame}[fragile]
\frametitle{Natural join, $\Join$}

\begin{center}
\begin{tabular}{ c c }

  \begin{tabular}{ c | c  }
    legs &  \\\hline
    animal  & legs \\ \hline  
    cat  & 2  \\
    dog & 2 \\
    crab &  8 \\
  \end{tabular} &

  \begin{tabular}{ c | c  }
    eyes &  \\\hline
    animal  & eyes \\ \hline  
    cat  & 2  \\
    dog & 2 \\
    crab &  2 \\
  \end{tabular}  
\end{tabular} 
\end{center}

\begin{center}
  \begin{tabular}{ c | c  c }
    $eyes \Join legs$&  & \\\hline
    animal  & eyes & legs\\ \hline  
    cat  & 2 & 2  \\
    dog & 2 & 2\\
    crab &  2 & 8 \\
  \end{tabular}  
\end{center}

\end{frame}



\section{Разни}
\subsection {Релации vs. функции}

\begin{frame}
\centerline{Релации и функции}
\end{frame}


\begin{frame}[fragile]
\frametitle{Релации vs. функции}
\begin{itemize}
  \item Нека е дадена релация $\mathcal{R} \subseteq \mathcal{N} \times \mathcal{N}$
  \item Нека $\forall x \in \mathcal{N}: f(x) \stackrel{def}{=}y \Leftrightarrow (x,y) \in \mathcal{R}$
  \vspace{1em}
  \pause
  \item Например $\forall a \in \mathcal{A}nimals:getLegs(a)\stackrel{def}{=}l \Leftrightarrow (a,l) \in legs$
  \item Тогава $getLegs(crab)=8$
  \vspace{1em}
  \pause
  \item Но нека $\forall x \in \mathcal{N}:leq(x)\stackrel{def}{=}y \Leftrightarrow (x,y) \in \leq$
  \begin{flushleft}
  \relscale{0.5}
  Да напомним $\leq=\{(x,y)|x \in \mathcal{N}, y \in \mathcal{N}, \exists z \in \mathcal{N}-\{0\}:y=x+z\} \subseteq \mathcal{N} \times \mathcal {N}$
  \end{flushleft}
  \item Знаем, че $0 \leq 0, 0 \leq 1,...$ изобщо $\forall y \in \mathcal{N}$ знаем, че $0 \leq y$
  \item Тогава колко е $leq(0)$?
   \vspace{1em}
  \pause
  \item Последно, можем ли $\forall$ функция $f:A\rightarrow B$ да построим релация $R_f=\{(x,y)|f(x)=y\}$?
\end{itemize}
\end{frame}

\begin{frame}
\centerline{Благодаря за вниманието!}
\end{frame}

\end{document}



\begin{frame}[fragile]
\end{frame}



\begin{columns}[t]
  \begin{column}{0.2\textwidth}

\relscale{0.63}
\begin{lstlisting}
\end{lstlisting}
\relscale{1}

  \end{column}
  \begin{column}{0.8\textwidth}

  \end{column}
\end{columns}



\begin{center}
  
\begin{tabular}{ c c }

\begin{tabular}{ c | c  c }
  \hline                        
  \&\&  & true  & false \\ \hline  
  true  & true  & false \\
  false & false & false \\
  
\end{tabular} &

\begin{tabular}{ c | c  c }
  \hline                        
  ||  & true  & false \\ \hline  
  true  & true  & true \\
  false & true & false \\
  
\end{tabular}

\end{tabular}
\end{center}

\begin{frame}
\centerline{Език за програмиране C++}
\end{frame}

